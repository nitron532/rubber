
	\item \text[5pts] What is the output of the following code? Explain your answer. Draw the memory model.
    % Provide comments for the following code. More specifically, label the high-level idea of each section of code. As an example, the first section is done for you.
	\begin{tcolorbox} [colback=white, colframe=black, width=6.7in, arc=0mm, boxrule=1pt]
	\text 
\#\code{include <iostream>}\\
\code{using namespace std;}\\
\code{int main()} \\
\code{\{}\\
    // Declare two integer variables\\
  \code{int a = 2;}\\
  \code{int b = 4;}\\\\
\begin{Solution}
  // Initialize pointers to the addresses of 'a' and 'b'\\
\end{Solution} \code{int *ptr1 = \&a;}\\
  \code{int *ptr2 = \&b;}\\\\
\begin{Solution}
  // Output the values pointed by ptr1 and ptr2 (2 and 4 respectively)\\
\end{Solution}
  \code{cout << *ptr1 << endl;}\\
  \code{cout << *ptr2 << endl;}\\\\
\begin{Solution}
    // Swap the pointers ptr1 and ptr2\\
\end{Solution}
  \code{int *temp = ptr1;}\begin{Solution}// temp now points to what ptr1 points to (a) \end{Solution}\\
  \code{ptr1 = ptr2;}       \begin{Solution}// ptr1 now points to what ptr2 points to (b) \end{Solution}\\
  \code{ptr2 = temp;}      \begin{Solution}// ptr2 now points to what temp points to (a) \end{Solution} \\\\
\begin{Solution}
  // Output the values pointed by ptr1 and ptr2 after the swap (4 and 2 respectively)\\
\end{Solution}
  \code{cout << *ptr1 << endl;}\\
  \code{cout << *ptr2 << endl;}\\\\
  \code{return 0;} \\
\code{\}}
	\end{tcolorbox}
% \text [5pts] What is the output? Explain your answer.
	\begin{tcolorbox} [colback=white, colframe=black, width=6.7in, arc=0mm, boxrule=1pt]
	\begin{Question}\vspace{3.5in}\end{Question}
	\begin{Solution}
	2\\
	4\\
	4\\
	2\\
	Initially, ptr1 points to a (value 2) and ptr2 points to b (value 4), so the first two lines of output are 2 and 4. The pointers ptr1 and ptr2 are then swapped, making ptr1 point to b (value 4) and ptr2 point to a (value 2), resulting in the last two lines of output being 4 and 2.
	\end{Solution}
	\end{tcolorbox}

% \item \text [5pts] Draw the memory model.
	% \begin{tcolorbox} [colback=white, colframe=black, width=6.7in, arc=0mm, boxrule=1pt]
	% \begin{Question}\vspace{3in}\end{Question}
	% \begin{Solution}
   
	% \end{Solution}
	% \end{tcolorbox}
