% \iffalse meta-comment
%
% File: classlist.dtx
% Version: 2016/05/16 v1.5
% Info: Record classes used in a document
%
% Copyright (C)
%    2005, 2006, 2008, 2011 Heiko Oberdiek
%    2016-2019 Oberdiek Package Support Group
%    https://github.com/ho-tex/oberdiek/issues
%
% This work may be distributed and/or modified under the
% conditions of the LaTeX Project Public License, either
% version 1.3c of this license or (at your option) any later
% version. This version of this license is in
%    https://www.latex-project.org/lppl/lppl-1-3c.txt
% and the latest version of this license is in
%    https://www.latex-project.org/lppl.txt
% and version 1.3 or later is part of all distributions of
% LaTeX version 2005/12/01 or later.
%
% This work has the LPPL maintenance status "maintained".
%
% The Current Maintainers of this work are
% Heiko Oberdiek and the Oberdiek Package Support Group
% https://github.com/ho-tex/oberdiek/issues
%
% This work consists of the main source file classlist.dtx
% and the derived files
%    classlist.sty, classlist.pdf, classlist.ins, classlist.drv.
%
% Distribution:
%    CTAN:macros/latex/contrib/oberdiek/classlist.dtx
%    CTAN:macros/latex/contrib/oberdiek/classlist.pdf
%
% Unpacking:
%    (a) If classlist.ins is present:
%           tex classlist.ins
%    (b) Without classlist.ins:
%           tex classlist.dtx
%    (c) If you insist on using LaTeX
%           latex \let\install=y% \iffalse meta-comment
%
% File: classlist.dtx
% Version: 2016/05/16 v1.5
% Info: Record classes used in a document
%
% Copyright (C)
%    2005, 2006, 2008, 2011 Heiko Oberdiek
%    2016-2019 Oberdiek Package Support Group
%    https://github.com/ho-tex/oberdiek/issues
%
% This work may be distributed and/or modified under the
% conditions of the LaTeX Project Public License, either
% version 1.3c of this license or (at your option) any later
% version. This version of this license is in
%    https://www.latex-project.org/lppl/lppl-1-3c.txt
% and the latest version of this license is in
%    https://www.latex-project.org/lppl.txt
% and version 1.3 or later is part of all distributions of
% LaTeX version 2005/12/01 or later.
%
% This work has the LPPL maintenance status "maintained".
%
% The Current Maintainers of this work are
% Heiko Oberdiek and the Oberdiek Package Support Group
% https://github.com/ho-tex/oberdiek/issues
%
% This work consists of the main source file classlist.dtx
% and the derived files
%    classlist.sty, classlist.pdf, classlist.ins, classlist.drv.
%
% Distribution:
%    CTAN:macros/latex/contrib/oberdiek/classlist.dtx
%    CTAN:macros/latex/contrib/oberdiek/classlist.pdf
%
% Unpacking:
%    (a) If classlist.ins is present:
%           tex classlist.ins
%    (b) Without classlist.ins:
%           tex classlist.dtx
%    (c) If you insist on using LaTeX
%           latex \let\install=y% \iffalse meta-comment
%
% File: classlist.dtx
% Version: 2016/05/16 v1.5
% Info: Record classes used in a document
%
% Copyright (C)
%    2005, 2006, 2008, 2011 Heiko Oberdiek
%    2016-2019 Oberdiek Package Support Group
%    https://github.com/ho-tex/oberdiek/issues
%
% This work may be distributed and/or modified under the
% conditions of the LaTeX Project Public License, either
% version 1.3c of this license or (at your option) any later
% version. This version of this license is in
%    https://www.latex-project.org/lppl/lppl-1-3c.txt
% and the latest version of this license is in
%    https://www.latex-project.org/lppl.txt
% and version 1.3 or later is part of all distributions of
% LaTeX version 2005/12/01 or later.
%
% This work has the LPPL maintenance status "maintained".
%
% The Current Maintainers of this work are
% Heiko Oberdiek and the Oberdiek Package Support Group
% https://github.com/ho-tex/oberdiek/issues
%
% This work consists of the main source file classlist.dtx
% and the derived files
%    classlist.sty, classlist.pdf, classlist.ins, classlist.drv.
%
% Distribution:
%    CTAN:macros/latex/contrib/oberdiek/classlist.dtx
%    CTAN:macros/latex/contrib/oberdiek/classlist.pdf
%
% Unpacking:
%    (a) If classlist.ins is present:
%           tex classlist.ins
%    (b) Without classlist.ins:
%           tex classlist.dtx
%    (c) If you insist on using LaTeX
%           latex \let\install=y% \iffalse meta-comment
%
% File: classlist.dtx
% Version: 2016/05/16 v1.5
% Info: Record classes used in a document
%
% Copyright (C)
%    2005, 2006, 2008, 2011 Heiko Oberdiek
%    2016-2019 Oberdiek Package Support Group
%    https://github.com/ho-tex/oberdiek/issues
%
% This work may be distributed and/or modified under the
% conditions of the LaTeX Project Public License, either
% version 1.3c of this license or (at your option) any later
% version. This version of this license is in
%    https://www.latex-project.org/lppl/lppl-1-3c.txt
% and the latest version of this license is in
%    https://www.latex-project.org/lppl.txt
% and version 1.3 or later is part of all distributions of
% LaTeX version 2005/12/01 or later.
%
% This work has the LPPL maintenance status "maintained".
%
% The Current Maintainers of this work are
% Heiko Oberdiek and the Oberdiek Package Support Group
% https://github.com/ho-tex/oberdiek/issues
%
% This work consists of the main source file classlist.dtx
% and the derived files
%    classlist.sty, classlist.pdf, classlist.ins, classlist.drv.
%
% Distribution:
%    CTAN:macros/latex/contrib/oberdiek/classlist.dtx
%    CTAN:macros/latex/contrib/oberdiek/classlist.pdf
%
% Unpacking:
%    (a) If classlist.ins is present:
%           tex classlist.ins
%    (b) Without classlist.ins:
%           tex classlist.dtx
%    (c) If you insist on using LaTeX
%           latex \let\install=y\input{classlist.dtx}
%        (quote the arguments according to the demands of your shell)
%
% Documentation:
%    (a) If classlist.drv is present:
%           latex classlist.drv
%    (b) Without classlist.drv:
%           latex classlist.dtx; ...
%    The class ltxdoc loads the configuration file ltxdoc.cfg
%    if available. Here you can specify further options, e.g.
%    use A4 as paper format:
%       \PassOptionsToClass{a4paper}{article}
%
%    Program calls to get the documentation (example):
%       pdflatex classlist.dtx
%       makeindex -s gind.ist classlist.idx
%       pdflatex classlist.dtx
%       makeindex -s gind.ist classlist.idx
%       pdflatex classlist.dtx
%
% Installation:
%    TDS:tex/latex/oberdiek/classlist.sty
%    TDS:doc/latex/oberdiek/classlist.pdf
%    TDS:source/latex/oberdiek/classlist.dtx
%
%<*ignore>
\begingroup
  \catcode123=1 %
  \catcode125=2 %
  \def\x{LaTeX2e}%
\expandafter\endgroup
\ifcase 0\ifx\install y1\fi\expandafter
         \ifx\csname processbatchFile\endcsname\relax\else1\fi
         \ifx\fmtname\x\else 1\fi\relax
\else\csname fi\endcsname
%</ignore>
%<*install>
\input docstrip.tex
\Msg{************************************************************************}
\Msg{* Installation}
\Msg{* Package: classlist 2016/05/16 v1.5 Record classes used in a document (HO)}
\Msg{************************************************************************}

\keepsilent
\askforoverwritefalse

\let\MetaPrefix\relax
\preamble

This is a generated file.

Project: classlist
Version: 2016/05/16 v1.5

Copyright (C)
   2005, 2006, 2008, 2011 Heiko Oberdiek
   2016-2019 Oberdiek Package Support Group

This work may be distributed and/or modified under the
conditions of the LaTeX Project Public License, either
version 1.3c of this license or (at your option) any later
version. This version of this license is in
   https://www.latex-project.org/lppl/lppl-1-3c.txt
and the latest version of this license is in
   https://www.latex-project.org/lppl.txt
and version 1.3 or later is part of all distributions of
LaTeX version 2005/12/01 or later.

This work has the LPPL maintenance status "maintained".

The Current Maintainers of this work are
Heiko Oberdiek and the Oberdiek Package Support Group
https://github.com/ho-tex/oberdiek/issues


This work consists of the main source file classlist.dtx
and the derived files
   classlist.sty, classlist.pdf, classlist.ins, classlist.drv.

\endpreamble
\let\MetaPrefix\DoubleperCent

\generate{%
  \file{classlist.ins}{\from{classlist.dtx}{install}}%
  \file{classlist.drv}{\from{classlist.dtx}{driver}}%
  \usedir{tex/latex/oberdiek}%
  \file{classlist.sty}{\from{classlist.dtx}{package}}%
}

\catcode32=13\relax% active space
\let =\space%
\Msg{************************************************************************}
\Msg{*}
\Msg{* To finish the installation you have to move the following}
\Msg{* file into a directory searched by TeX:}
\Msg{*}
\Msg{*     classlist.sty}
\Msg{*}
\Msg{* To produce the documentation run the file `classlist.drv'}
\Msg{* through LaTeX.}
\Msg{*}
\Msg{* Happy TeXing!}
\Msg{*}
\Msg{************************************************************************}

\endbatchfile
%</install>
%<*ignore>
\fi
%</ignore>
%<*driver>
\NeedsTeXFormat{LaTeX2e}
\ProvidesFile{classlist.drv}%
  [2016/05/16 v1.5 Record classes used in a document (HO)]%
\documentclass{ltxdoc}
\usepackage{holtxdoc}[2011/11/22]
\begin{document}
  \DocInput{classlist.dtx}%
\end{document}
%</driver>
% \fi
%
%
%
% \GetFileInfo{classlist.drv}
%
% \title{The \xpackage{classlist} package}
% \date{2016/05/16 v1.5}
% \author{Heiko Oberdiek\thanks
% {Please report any issues at \url{https://github.com/ho-tex/oberdiek/issues}}}
%
% \maketitle
%
% \begin{abstract}
% This package records the loaded classes and stores
% them in a list.
% \end{abstract}
%
% \tableofcontents
%
% \section{Documentation}
%
% \subsection{Background}
%
% This packages is an answer of a newsgroup question:
% \begin{quote}
% \begin{tabular}{@{}ll@{}}
%   Newsgroup: & comp.text.tex\\
%   Subject: & Finding the Document Class\\
%   From: & Herber Schulz\\
%   Date: & 18 Jun 2005 13:16:49 -0500\\
%   Message-ID: &
%    \textless
%    \texttt{herbs-D55DB9.13170418062005@news.isp.giganews.com}^^A
%    \textgreater
% \end{tabular}
% \end{quote}
%
% \subsection{Usage}
%
% Load this package before \cs{documentclass}:
% \begin{quote}
%   |\RequirePackage{classlist}|\\
%   |\documentclass[some,options]{whatever}|
% \end{quote}
% It then records the classes with options.
%
% If used after \cs{documentclass}, \cs{@filelist} is
% parsed for classes. The additional data
% specified options and requested version is no
% longer available here.
%
% \begin{description}
% \item[\cs{MainClassName}] contains the first loaded class.
% \item[\cs{ClassList}] stores the class entries, eg.
%   \begin{quote}
%   \begin{tabular}{@{}l@{ }l@{}}
%     \cs{ClassList} $\rightarrow$&
%     |\ClassListEntry{myarticle}{a4paper}{}|\\
%     &|\ClassListEntry{article}{}{}|
%   \end{tabular}
%   \end{quote}
% \item[\cs{ClassListEntry}] has three arguments:
%   \begin{quote}
%   \begin{tabular}{@{}ll@{}}
%     |#1|: & class name\\
%     |#2|: & options given in \cs{documentclass}/\cs{LoadClass}\\
%     |#3|: & requested version, not the version of class
%   \end{tabular}
%   \end{quote}
% \item[\cs{PrintClassList}] prints the list on screen it can be
%    configured by
% \item[\cs{PrintClassListTitle}] for the title and
% \item[\cs{PrintClassListEntry}] for formatting the entries.
%    See the implementation for how to use these.
% \end{description}
%
% \StopEventually{
% }
%
% \section{Implementation}
%
%    \begin{macrocode}
%<*package>
%    \end{macrocode}
%    Package identification.
%    \begin{macrocode}
\NeedsTeXFormat{LaTeX2e}
\ProvidesPackage{classlist}%
  [2016/05/16 v1.5 Record classes used in a document (HO)]
%    \end{macrocode}
%
%    \begin{macrocode}
\let\ClassList\@empty
\let\MainClassName\relax
%    \end{macrocode}
%
%    Test, whether we are called before \cs{documentclass}.
%    \begin{macrocode}
\ifx\@classoptionslist\relax
  \let\CL@org@fileswith@pti@ns\@fileswith@pti@ns
  \def\@fileswith@pti@ns#1[#2]#3[#4]{%
%    \end{macrocode}
%    \begin{tabular}{@{}ll@{}}
%      |#1|: & \cs{@clsextension}\\
%      |#2|: & options of \cs{documentclass}/\cs{LoadClass}\\
%      |#3|: & class name\\
%      |#4|: & requested version
%    \end{tabular}
%    \begin{macrocode}
    \ifx#1\@clsextension
      \@ifl@aded#1{#3}{%
        \PackageInfo{classlist}{%
          Skipping class `#3', because\MessageBreak
          this class is already loaded%
        }%
      }{%
        \@ifundefined{MainClassName}{%
          \def\MainClassName{#3}%
        }{}%
        \@temptokena\expandafter{%
          \ClassList
          \ClassListEntry{#3}{#2}{#4}%
        }%
        \edef\ClassList{\the\@temptokena}%
      }%
    \fi
    \CL@org@fileswith@pti@ns{#1}[{#2}]{#3}[{#4}]%
  }%
  \let\@@fileswith@pti@ns\@fileswith@pti@ns
\else
%    \end{macrocode}
%    Called after \cs{documentclass}.
%    \begin{macrocode}
  \PackageInfo{classlist}{Use \string\@filelist\space method}%

  \let\ClassListEntry\relax
  \expandafter\def\expandafter\CL@test
      \expandafter#\expandafter1\@clsextension#2\@nil{%
    \ifx\\#2\\%
%    \end{macrocode}
%    Name does not contain \cs{@clsextension}
%    \begin{macrocode}
    \else
      \expandafter\CL@test@i\CL@entry\@nil
    \fi
  }%
  \expandafter\def\expandafter\CL@test@i
      \expandafter#\expandafter1\@clsextension#2\@nil{%
    \ifx\\#2\\%
      \@ifundefined{opt@\CL@entry}{%
      }{%
        \@ifundefined{MainClassName}{%
          \let\MainClassName\CL@entry
        }{%
        }%
        \edef\ClassList{%
          \ClassList
          \ClassListEntry{\CL@entry}{}{}%
        }%
      }%
    \else
%    \end{macrocode}
%    Names with more than one \cs{@clsextension} are not supported.
%    \begin{macrocode}
    \fi
  }%
  \@for\CL@entry:=\@filelist\do{%
    \expandafter\expandafter\expandafter\CL@test\expandafter
        \CL@entry\@clsextension\@nil
  }%
\fi
%    \end{macrocode}
%
%    \begin{macro}{\PrintClassListEntry}
%    \begin{macrocode}
\providecommand*{\PrintClassListEntry}[3]{%
  \toks@{* #1}%
  \typeout{\the\toks@}%
}
%    \end{macrocode}
%    \end{macro}
%    \begin{macro}{\PrintClassListTitle}
%    \begin{macrocode}
\providecommand*{\PrintClassListTitle}{%
  \typeout{Class list:}%
}
%    \end{macrocode}
%    \end{macro}
%    \begin{macro}{\PrintClassList}
%    \begin{macrocode}
\providecommand*{\PrintClassList}{%
  \begingroup
    \let\ClassListEntry\PrintClassListEntry
    \PrintClassListTitle
    \ClassList
  \endgroup
}
%    \end{macrocode}
%    \end{macro}
%    \begin{macro}{\CL@InfoEntry}
%    \begin{macrocode}
\def\CL@InfoEntry#1#2#3{%
  \advance\count@ by \@ne
  \def\x{#2}%
  \@onelevel@sanitize\x
  \edef\CL@Info{%
    \CL@Info
    \noexpand\MessageBreak
    (\the\count@) %
    #1 [\x]%
    \ifx\\#3\\%
    \else
      \space[#3]% hash-ok
    \fi
  }%
}
%    \end{macrocode}
%    \end{macro}
%    \begin{macrocode}
\AtBeginDocument{%
  \begingroup
    \count@=\z@
    \def\CL@Info{Class List:}%
    \let\ClassListEntry\CL@InfoEntry
    \ClassList
    \let\on@line\@empty
    \PackageInfo{classlist}{\CL@Info}%
  \endgroup
}
%    \end{macrocode}
%
%    \begin{macrocode}
%</package>
%    \end{macrocode}
%
% \section{Installation}
%
% \subsection{Download}
%
% \paragraph{Package.} This package is available on
% CTAN\footnote{\CTANpkg{classlist}}:
% \begin{description}
% \item[\CTAN{macros/latex/contrib/oberdiek/classlist.dtx}] The source file.
% \item[\CTAN{macros/latex/contrib/oberdiek/classlist.pdf}] Documentation.
% \end{description}
%
%
% \paragraph{Bundle.} All the packages of the bundle `oberdiek'
% are also available in a TDS compliant ZIP archive. There
% the packages are already unpacked and the documentation files
% are generated. The files and directories obey the TDS standard.
% \begin{description}
% \item[\CTANinstall{install/macros/latex/contrib/oberdiek.tds.zip}]
% \end{description}
% \emph{TDS} refers to the standard ``A Directory Structure
% for \TeX\ Files'' (\CTANpkg{tds}). Directories
% with \xfile{texmf} in their name are usually organized this way.
%
% \subsection{Bundle installation}
%
% \paragraph{Unpacking.} Unpack the \xfile{oberdiek.tds.zip} in the
% TDS tree (also known as \xfile{texmf} tree) of your choice.
% Example (linux):
% \begin{quote}
%   |unzip oberdiek.tds.zip -d ~/texmf|
% \end{quote}
%
% \subsection{Package installation}
%
% \paragraph{Unpacking.} The \xfile{.dtx} file is a self-extracting
% \docstrip\ archive. The files are extracted by running the
% \xfile{.dtx} through \plainTeX:
% \begin{quote}
%   \verb|tex classlist.dtx|
% \end{quote}
%
% \paragraph{TDS.} Now the different files must be moved into
% the different directories in your installation TDS tree
% (also known as \xfile{texmf} tree):
% \begin{quote}
% \def\t{^^A
% \begin{tabular}{@{}>{\ttfamily}l@{ $\rightarrow$ }>{\ttfamily}l@{}}
%   classlist.sty & tex/latex/oberdiek/classlist.sty\\
%   classlist.pdf & doc/latex/oberdiek/classlist.pdf\\
%   classlist.dtx & source/latex/oberdiek/classlist.dtx\\
% \end{tabular}^^A
% }^^A
% \sbox0{\t}^^A
% \ifdim\wd0>\linewidth
%   \begingroup
%     \advance\linewidth by\leftmargin
%     \advance\linewidth by\rightmargin
%   \edef\x{\endgroup
%     \def\noexpand\lw{\the\linewidth}^^A
%   }\x
%   \def\lwbox{^^A
%     \leavevmode
%     \hbox to \linewidth{^^A
%       \kern-\leftmargin\relax
%       \hss
%       \usebox0
%       \hss
%       \kern-\rightmargin\relax
%     }^^A
%   }^^A
%   \ifdim\wd0>\lw
%     \sbox0{\small\t}^^A
%     \ifdim\wd0>\linewidth
%       \ifdim\wd0>\lw
%         \sbox0{\footnotesize\t}^^A
%         \ifdim\wd0>\linewidth
%           \ifdim\wd0>\lw
%             \sbox0{\scriptsize\t}^^A
%             \ifdim\wd0>\linewidth
%               \ifdim\wd0>\lw
%                 \sbox0{\tiny\t}^^A
%                 \ifdim\wd0>\linewidth
%                   \lwbox
%                 \else
%                   \usebox0
%                 \fi
%               \else
%                 \lwbox
%               \fi
%             \else
%               \usebox0
%             \fi
%           \else
%             \lwbox
%           \fi
%         \else
%           \usebox0
%         \fi
%       \else
%         \lwbox
%       \fi
%     \else
%       \usebox0
%     \fi
%   \else
%     \lwbox
%   \fi
% \else
%   \usebox0
% \fi
% \end{quote}
% If you have a \xfile{docstrip.cfg} that configures and enables \docstrip's
% TDS installing feature, then some files can already be in the right
% place, see the documentation of \docstrip.
%
% \subsection{Refresh file name databases}
%
% If your \TeX~distribution
% (\TeX\,Live, \mikTeX, \dots) relies on file name databases, you must refresh
% these. For example, \TeX\,Live\ users run \verb|texhash| or
% \verb|mktexlsr|.
%
% \subsection{Some details for the interested}
%
% \paragraph{Unpacking with \LaTeX.}
% The \xfile{.dtx} chooses its action depending on the format:
% \begin{description}
% \item[\plainTeX:] Run \docstrip\ and extract the files.
% \item[\LaTeX:] Generate the documentation.
% \end{description}
% If you insist on using \LaTeX\ for \docstrip\ (really,
% \docstrip\ does not need \LaTeX), then inform the autodetect routine
% about your intention:
% \begin{quote}
%   \verb|latex \let\install=y\input{classlist.dtx}|
% \end{quote}
% Do not forget to quote the argument according to the demands
% of your shell.
%
% \paragraph{Generating the documentation.}
% You can use both the \xfile{.dtx} or the \xfile{.drv} to generate
% the documentation. The process can be configured by the
% configuration file \xfile{ltxdoc.cfg}. For instance, put this
% line into this file, if you want to have A4 as paper format:
% \begin{quote}
%   \verb|\PassOptionsToClass{a4paper}{article}|
% \end{quote}
% An example follows how to generate the
% documentation with pdf\LaTeX:
% \begin{quote}
%\begin{verbatim}
%pdflatex classlist.dtx
%makeindex -s gind.ist classlist.idx
%pdflatex classlist.dtx
%makeindex -s gind.ist classlist.idx
%pdflatex classlist.dtx
%\end{verbatim}
% \end{quote}
%
% \begin{History}
%   \begin{Version}{2005/06/19 v1.0}
%   \item
%     First published version: CTAN and newsgroup \xnewsgroup{comp.text.tex}:
%     \URL{``\link{Re: Finding the Document Class}''}^^A
%     {https://groups.google.com/group/comp.text.tex/msg/8ee9523c2dc13666}
%   \end{Version}
%   \begin{Version}{2005/06/19 v1.1}
%   \item
%     After \cs{documentclass} the package looks
%     at \cs{@filelist} instead of aborting with error.
%   \end{Version}
%   \begin{Version}{2006/02/20 v1.2}
%   \item
%     DTX framework.
%   \item
%     Fix for \cs{@@fileswith@pti@ns}.
%   \end{Version}
%   \begin{Version}{2008/08/11 v1.3}
%   \item
%     Code is not changed.
%   \item
%     URLs updated.
%   \end{Version}
%   \begin{Version}{2011/10/17 v1.4}
%   \item
%     Documentation fix: \cs{MainClass} $\rightarrow$ \cs{MainClassName}.
%   \end{Version}
%   \begin{Version}{2016/05/16 v1.5}
%   \item
%     Documentation updates.
%   \end{Version}
% \end{History}
%
% \PrintIndex
%
% \Finale
\endinput

%        (quote the arguments according to the demands of your shell)
%
% Documentation:
%    (a) If classlist.drv is present:
%           latex classlist.drv
%    (b) Without classlist.drv:
%           latex classlist.dtx; ...
%    The class ltxdoc loads the configuration file ltxdoc.cfg
%    if available. Here you can specify further options, e.g.
%    use A4 as paper format:
%       \PassOptionsToClass{a4paper}{article}
%
%    Program calls to get the documentation (example):
%       pdflatex classlist.dtx
%       makeindex -s gind.ist classlist.idx
%       pdflatex classlist.dtx
%       makeindex -s gind.ist classlist.idx
%       pdflatex classlist.dtx
%
% Installation:
%    TDS:tex/latex/oberdiek/classlist.sty
%    TDS:doc/latex/oberdiek/classlist.pdf
%    TDS:source/latex/oberdiek/classlist.dtx
%
%<*ignore>
\begingroup
  \catcode123=1 %
  \catcode125=2 %
  \def\x{LaTeX2e}%
\expandafter\endgroup
\ifcase 0\ifx\install y1\fi\expandafter
         \ifx\csname processbatchFile\endcsname\relax\else1\fi
         \ifx\fmtname\x\else 1\fi\relax
\else\csname fi\endcsname
%</ignore>
%<*install>
\input docstrip.tex
\Msg{************************************************************************}
\Msg{* Installation}
\Msg{* Package: classlist 2016/05/16 v1.5 Record classes used in a document (HO)}
\Msg{************************************************************************}

\keepsilent
\askforoverwritefalse

\let\MetaPrefix\relax
\preamble

This is a generated file.

Project: classlist
Version: 2016/05/16 v1.5

Copyright (C)
   2005, 2006, 2008, 2011 Heiko Oberdiek
   2016-2019 Oberdiek Package Support Group

This work may be distributed and/or modified under the
conditions of the LaTeX Project Public License, either
version 1.3c of this license or (at your option) any later
version. This version of this license is in
   https://www.latex-project.org/lppl/lppl-1-3c.txt
and the latest version of this license is in
   https://www.latex-project.org/lppl.txt
and version 1.3 or later is part of all distributions of
LaTeX version 2005/12/01 or later.

This work has the LPPL maintenance status "maintained".

The Current Maintainers of this work are
Heiko Oberdiek and the Oberdiek Package Support Group
https://github.com/ho-tex/oberdiek/issues


This work consists of the main source file classlist.dtx
and the derived files
   classlist.sty, classlist.pdf, classlist.ins, classlist.drv.

\endpreamble
\let\MetaPrefix\DoubleperCent

\generate{%
  \file{classlist.ins}{\from{classlist.dtx}{install}}%
  \file{classlist.drv}{\from{classlist.dtx}{driver}}%
  \usedir{tex/latex/oberdiek}%
  \file{classlist.sty}{\from{classlist.dtx}{package}}%
}

\catcode32=13\relax% active space
\let =\space%
\Msg{************************************************************************}
\Msg{*}
\Msg{* To finish the installation you have to move the following}
\Msg{* file into a directory searched by TeX:}
\Msg{*}
\Msg{*     classlist.sty}
\Msg{*}
\Msg{* To produce the documentation run the file `classlist.drv'}
\Msg{* through LaTeX.}
\Msg{*}
\Msg{* Happy TeXing!}
\Msg{*}
\Msg{************************************************************************}

\endbatchfile
%</install>
%<*ignore>
\fi
%</ignore>
%<*driver>
\NeedsTeXFormat{LaTeX2e}
\ProvidesFile{classlist.drv}%
  [2016/05/16 v1.5 Record classes used in a document (HO)]%
\documentclass{ltxdoc}
\usepackage{holtxdoc}[2011/11/22]
\begin{document}
  \DocInput{classlist.dtx}%
\end{document}
%</driver>
% \fi
%
%
%
% \GetFileInfo{classlist.drv}
%
% \title{The \xpackage{classlist} package}
% \date{2016/05/16 v1.5}
% \author{Heiko Oberdiek\thanks
% {Please report any issues at \url{https://github.com/ho-tex/oberdiek/issues}}}
%
% \maketitle
%
% \begin{abstract}
% This package records the loaded classes and stores
% them in a list.
% \end{abstract}
%
% \tableofcontents
%
% \section{Documentation}
%
% \subsection{Background}
%
% This packages is an answer of a newsgroup question:
% \begin{quote}
% \begin{tabular}{@{}ll@{}}
%   Newsgroup: & comp.text.tex\\
%   Subject: & Finding the Document Class\\
%   From: & Herber Schulz\\
%   Date: & 18 Jun 2005 13:16:49 -0500\\
%   Message-ID: &
%    \textless
%    \texttt{herbs-D55DB9.13170418062005@news.isp.giganews.com}^^A
%    \textgreater
% \end{tabular}
% \end{quote}
%
% \subsection{Usage}
%
% Load this package before \cs{documentclass}:
% \begin{quote}
%   |\RequirePackage{classlist}|\\
%   |\documentclass[some,options]{whatever}|
% \end{quote}
% It then records the classes with options.
%
% If used after \cs{documentclass}, \cs{@filelist} is
% parsed for classes. The additional data
% specified options and requested version is no
% longer available here.
%
% \begin{description}
% \item[\cs{MainClassName}] contains the first loaded class.
% \item[\cs{ClassList}] stores the class entries, eg.
%   \begin{quote}
%   \begin{tabular}{@{}l@{ }l@{}}
%     \cs{ClassList} $\rightarrow$&
%     |\ClassListEntry{myarticle}{a4paper}{}|\\
%     &|\ClassListEntry{article}{}{}|
%   \end{tabular}
%   \end{quote}
% \item[\cs{ClassListEntry}] has three arguments:
%   \begin{quote}
%   \begin{tabular}{@{}ll@{}}
%     |#1|: & class name\\
%     |#2|: & options given in \cs{documentclass}/\cs{LoadClass}\\
%     |#3|: & requested version, not the version of class
%   \end{tabular}
%   \end{quote}
% \item[\cs{PrintClassList}] prints the list on screen it can be
%    configured by
% \item[\cs{PrintClassListTitle}] for the title and
% \item[\cs{PrintClassListEntry}] for formatting the entries.
%    See the implementation for how to use these.
% \end{description}
%
% \StopEventually{
% }
%
% \section{Implementation}
%
%    \begin{macrocode}
%<*package>
%    \end{macrocode}
%    Package identification.
%    \begin{macrocode}
\NeedsTeXFormat{LaTeX2e}
\ProvidesPackage{classlist}%
  [2016/05/16 v1.5 Record classes used in a document (HO)]
%    \end{macrocode}
%
%    \begin{macrocode}
\let\ClassList\@empty
\let\MainClassName\relax
%    \end{macrocode}
%
%    Test, whether we are called before \cs{documentclass}.
%    \begin{macrocode}
\ifx\@classoptionslist\relax
  \let\CL@org@fileswith@pti@ns\@fileswith@pti@ns
  \def\@fileswith@pti@ns#1[#2]#3[#4]{%
%    \end{macrocode}
%    \begin{tabular}{@{}ll@{}}
%      |#1|: & \cs{@clsextension}\\
%      |#2|: & options of \cs{documentclass}/\cs{LoadClass}\\
%      |#3|: & class name\\
%      |#4|: & requested version
%    \end{tabular}
%    \begin{macrocode}
    \ifx#1\@clsextension
      \@ifl@aded#1{#3}{%
        \PackageInfo{classlist}{%
          Skipping class `#3', because\MessageBreak
          this class is already loaded%
        }%
      }{%
        \@ifundefined{MainClassName}{%
          \def\MainClassName{#3}%
        }{}%
        \@temptokena\expandafter{%
          \ClassList
          \ClassListEntry{#3}{#2}{#4}%
        }%
        \edef\ClassList{\the\@temptokena}%
      }%
    \fi
    \CL@org@fileswith@pti@ns{#1}[{#2}]{#3}[{#4}]%
  }%
  \let\@@fileswith@pti@ns\@fileswith@pti@ns
\else
%    \end{macrocode}
%    Called after \cs{documentclass}.
%    \begin{macrocode}
  \PackageInfo{classlist}{Use \string\@filelist\space method}%

  \let\ClassListEntry\relax
  \expandafter\def\expandafter\CL@test
      \expandafter#\expandafter1\@clsextension#2\@nil{%
    \ifx\\#2\\%
%    \end{macrocode}
%    Name does not contain \cs{@clsextension}
%    \begin{macrocode}
    \else
      \expandafter\CL@test@i\CL@entry\@nil
    \fi
  }%
  \expandafter\def\expandafter\CL@test@i
      \expandafter#\expandafter1\@clsextension#2\@nil{%
    \ifx\\#2\\%
      \@ifundefined{opt@\CL@entry}{%
      }{%
        \@ifundefined{MainClassName}{%
          \let\MainClassName\CL@entry
        }{%
        }%
        \edef\ClassList{%
          \ClassList
          \ClassListEntry{\CL@entry}{}{}%
        }%
      }%
    \else
%    \end{macrocode}
%    Names with more than one \cs{@clsextension} are not supported.
%    \begin{macrocode}
    \fi
  }%
  \@for\CL@entry:=\@filelist\do{%
    \expandafter\expandafter\expandafter\CL@test\expandafter
        \CL@entry\@clsextension\@nil
  }%
\fi
%    \end{macrocode}
%
%    \begin{macro}{\PrintClassListEntry}
%    \begin{macrocode}
\providecommand*{\PrintClassListEntry}[3]{%
  \toks@{* #1}%
  \typeout{\the\toks@}%
}
%    \end{macrocode}
%    \end{macro}
%    \begin{macro}{\PrintClassListTitle}
%    \begin{macrocode}
\providecommand*{\PrintClassListTitle}{%
  \typeout{Class list:}%
}
%    \end{macrocode}
%    \end{macro}
%    \begin{macro}{\PrintClassList}
%    \begin{macrocode}
\providecommand*{\PrintClassList}{%
  \begingroup
    \let\ClassListEntry\PrintClassListEntry
    \PrintClassListTitle
    \ClassList
  \endgroup
}
%    \end{macrocode}
%    \end{macro}
%    \begin{macro}{\CL@InfoEntry}
%    \begin{macrocode}
\def\CL@InfoEntry#1#2#3{%
  \advance\count@ by \@ne
  \def\x{#2}%
  \@onelevel@sanitize\x
  \edef\CL@Info{%
    \CL@Info
    \noexpand\MessageBreak
    (\the\count@) %
    #1 [\x]%
    \ifx\\#3\\%
    \else
      \space[#3]% hash-ok
    \fi
  }%
}
%    \end{macrocode}
%    \end{macro}
%    \begin{macrocode}
\AtBeginDocument{%
  \begingroup
    \count@=\z@
    \def\CL@Info{Class List:}%
    \let\ClassListEntry\CL@InfoEntry
    \ClassList
    \let\on@line\@empty
    \PackageInfo{classlist}{\CL@Info}%
  \endgroup
}
%    \end{macrocode}
%
%    \begin{macrocode}
%</package>
%    \end{macrocode}
%
% \section{Installation}
%
% \subsection{Download}
%
% \paragraph{Package.} This package is available on
% CTAN\footnote{\CTANpkg{classlist}}:
% \begin{description}
% \item[\CTAN{macros/latex/contrib/oberdiek/classlist.dtx}] The source file.
% \item[\CTAN{macros/latex/contrib/oberdiek/classlist.pdf}] Documentation.
% \end{description}
%
%
% \paragraph{Bundle.} All the packages of the bundle `oberdiek'
% are also available in a TDS compliant ZIP archive. There
% the packages are already unpacked and the documentation files
% are generated. The files and directories obey the TDS standard.
% \begin{description}
% \item[\CTANinstall{install/macros/latex/contrib/oberdiek.tds.zip}]
% \end{description}
% \emph{TDS} refers to the standard ``A Directory Structure
% for \TeX\ Files'' (\CTANpkg{tds}). Directories
% with \xfile{texmf} in their name are usually organized this way.
%
% \subsection{Bundle installation}
%
% \paragraph{Unpacking.} Unpack the \xfile{oberdiek.tds.zip} in the
% TDS tree (also known as \xfile{texmf} tree) of your choice.
% Example (linux):
% \begin{quote}
%   |unzip oberdiek.tds.zip -d ~/texmf|
% \end{quote}
%
% \subsection{Package installation}
%
% \paragraph{Unpacking.} The \xfile{.dtx} file is a self-extracting
% \docstrip\ archive. The files are extracted by running the
% \xfile{.dtx} through \plainTeX:
% \begin{quote}
%   \verb|tex classlist.dtx|
% \end{quote}
%
% \paragraph{TDS.} Now the different files must be moved into
% the different directories in your installation TDS tree
% (also known as \xfile{texmf} tree):
% \begin{quote}
% \def\t{^^A
% \begin{tabular}{@{}>{\ttfamily}l@{ $\rightarrow$ }>{\ttfamily}l@{}}
%   classlist.sty & tex/latex/oberdiek/classlist.sty\\
%   classlist.pdf & doc/latex/oberdiek/classlist.pdf\\
%   classlist.dtx & source/latex/oberdiek/classlist.dtx\\
% \end{tabular}^^A
% }^^A
% \sbox0{\t}^^A
% \ifdim\wd0>\linewidth
%   \begingroup
%     \advance\linewidth by\leftmargin
%     \advance\linewidth by\rightmargin
%   \edef\x{\endgroup
%     \def\noexpand\lw{\the\linewidth}^^A
%   }\x
%   \def\lwbox{^^A
%     \leavevmode
%     \hbox to \linewidth{^^A
%       \kern-\leftmargin\relax
%       \hss
%       \usebox0
%       \hss
%       \kern-\rightmargin\relax
%     }^^A
%   }^^A
%   \ifdim\wd0>\lw
%     \sbox0{\small\t}^^A
%     \ifdim\wd0>\linewidth
%       \ifdim\wd0>\lw
%         \sbox0{\footnotesize\t}^^A
%         \ifdim\wd0>\linewidth
%           \ifdim\wd0>\lw
%             \sbox0{\scriptsize\t}^^A
%             \ifdim\wd0>\linewidth
%               \ifdim\wd0>\lw
%                 \sbox0{\tiny\t}^^A
%                 \ifdim\wd0>\linewidth
%                   \lwbox
%                 \else
%                   \usebox0
%                 \fi
%               \else
%                 \lwbox
%               \fi
%             \else
%               \usebox0
%             \fi
%           \else
%             \lwbox
%           \fi
%         \else
%           \usebox0
%         \fi
%       \else
%         \lwbox
%       \fi
%     \else
%       \usebox0
%     \fi
%   \else
%     \lwbox
%   \fi
% \else
%   \usebox0
% \fi
% \end{quote}
% If you have a \xfile{docstrip.cfg} that configures and enables \docstrip's
% TDS installing feature, then some files can already be in the right
% place, see the documentation of \docstrip.
%
% \subsection{Refresh file name databases}
%
% If your \TeX~distribution
% (\TeX\,Live, \mikTeX, \dots) relies on file name databases, you must refresh
% these. For example, \TeX\,Live\ users run \verb|texhash| or
% \verb|mktexlsr|.
%
% \subsection{Some details for the interested}
%
% \paragraph{Unpacking with \LaTeX.}
% The \xfile{.dtx} chooses its action depending on the format:
% \begin{description}
% \item[\plainTeX:] Run \docstrip\ and extract the files.
% \item[\LaTeX:] Generate the documentation.
% \end{description}
% If you insist on using \LaTeX\ for \docstrip\ (really,
% \docstrip\ does not need \LaTeX), then inform the autodetect routine
% about your intention:
% \begin{quote}
%   \verb|latex \let\install=y% \iffalse meta-comment
%
% File: classlist.dtx
% Version: 2016/05/16 v1.5
% Info: Record classes used in a document
%
% Copyright (C)
%    2005, 2006, 2008, 2011 Heiko Oberdiek
%    2016-2019 Oberdiek Package Support Group
%    https://github.com/ho-tex/oberdiek/issues
%
% This work may be distributed and/or modified under the
% conditions of the LaTeX Project Public License, either
% version 1.3c of this license or (at your option) any later
% version. This version of this license is in
%    https://www.latex-project.org/lppl/lppl-1-3c.txt
% and the latest version of this license is in
%    https://www.latex-project.org/lppl.txt
% and version 1.3 or later is part of all distributions of
% LaTeX version 2005/12/01 or later.
%
% This work has the LPPL maintenance status "maintained".
%
% The Current Maintainers of this work are
% Heiko Oberdiek and the Oberdiek Package Support Group
% https://github.com/ho-tex/oberdiek/issues
%
% This work consists of the main source file classlist.dtx
% and the derived files
%    classlist.sty, classlist.pdf, classlist.ins, classlist.drv.
%
% Distribution:
%    CTAN:macros/latex/contrib/oberdiek/classlist.dtx
%    CTAN:macros/latex/contrib/oberdiek/classlist.pdf
%
% Unpacking:
%    (a) If classlist.ins is present:
%           tex classlist.ins
%    (b) Without classlist.ins:
%           tex classlist.dtx
%    (c) If you insist on using LaTeX
%           latex \let\install=y\input{classlist.dtx}
%        (quote the arguments according to the demands of your shell)
%
% Documentation:
%    (a) If classlist.drv is present:
%           latex classlist.drv
%    (b) Without classlist.drv:
%           latex classlist.dtx; ...
%    The class ltxdoc loads the configuration file ltxdoc.cfg
%    if available. Here you can specify further options, e.g.
%    use A4 as paper format:
%       \PassOptionsToClass{a4paper}{article}
%
%    Program calls to get the documentation (example):
%       pdflatex classlist.dtx
%       makeindex -s gind.ist classlist.idx
%       pdflatex classlist.dtx
%       makeindex -s gind.ist classlist.idx
%       pdflatex classlist.dtx
%
% Installation:
%    TDS:tex/latex/oberdiek/classlist.sty
%    TDS:doc/latex/oberdiek/classlist.pdf
%    TDS:source/latex/oberdiek/classlist.dtx
%
%<*ignore>
\begingroup
  \catcode123=1 %
  \catcode125=2 %
  \def\x{LaTeX2e}%
\expandafter\endgroup
\ifcase 0\ifx\install y1\fi\expandafter
         \ifx\csname processbatchFile\endcsname\relax\else1\fi
         \ifx\fmtname\x\else 1\fi\relax
\else\csname fi\endcsname
%</ignore>
%<*install>
\input docstrip.tex
\Msg{************************************************************************}
\Msg{* Installation}
\Msg{* Package: classlist 2016/05/16 v1.5 Record classes used in a document (HO)}
\Msg{************************************************************************}

\keepsilent
\askforoverwritefalse

\let\MetaPrefix\relax
\preamble

This is a generated file.

Project: classlist
Version: 2016/05/16 v1.5

Copyright (C)
   2005, 2006, 2008, 2011 Heiko Oberdiek
   2016-2019 Oberdiek Package Support Group

This work may be distributed and/or modified under the
conditions of the LaTeX Project Public License, either
version 1.3c of this license or (at your option) any later
version. This version of this license is in
   https://www.latex-project.org/lppl/lppl-1-3c.txt
and the latest version of this license is in
   https://www.latex-project.org/lppl.txt
and version 1.3 or later is part of all distributions of
LaTeX version 2005/12/01 or later.

This work has the LPPL maintenance status "maintained".

The Current Maintainers of this work are
Heiko Oberdiek and the Oberdiek Package Support Group
https://github.com/ho-tex/oberdiek/issues


This work consists of the main source file classlist.dtx
and the derived files
   classlist.sty, classlist.pdf, classlist.ins, classlist.drv.

\endpreamble
\let\MetaPrefix\DoubleperCent

\generate{%
  \file{classlist.ins}{\from{classlist.dtx}{install}}%
  \file{classlist.drv}{\from{classlist.dtx}{driver}}%
  \usedir{tex/latex/oberdiek}%
  \file{classlist.sty}{\from{classlist.dtx}{package}}%
}

\catcode32=13\relax% active space
\let =\space%
\Msg{************************************************************************}
\Msg{*}
\Msg{* To finish the installation you have to move the following}
\Msg{* file into a directory searched by TeX:}
\Msg{*}
\Msg{*     classlist.sty}
\Msg{*}
\Msg{* To produce the documentation run the file `classlist.drv'}
\Msg{* through LaTeX.}
\Msg{*}
\Msg{* Happy TeXing!}
\Msg{*}
\Msg{************************************************************************}

\endbatchfile
%</install>
%<*ignore>
\fi
%</ignore>
%<*driver>
\NeedsTeXFormat{LaTeX2e}
\ProvidesFile{classlist.drv}%
  [2016/05/16 v1.5 Record classes used in a document (HO)]%
\documentclass{ltxdoc}
\usepackage{holtxdoc}[2011/11/22]
\begin{document}
  \DocInput{classlist.dtx}%
\end{document}
%</driver>
% \fi
%
%
%
% \GetFileInfo{classlist.drv}
%
% \title{The \xpackage{classlist} package}
% \date{2016/05/16 v1.5}
% \author{Heiko Oberdiek\thanks
% {Please report any issues at \url{https://github.com/ho-tex/oberdiek/issues}}}
%
% \maketitle
%
% \begin{abstract}
% This package records the loaded classes and stores
% them in a list.
% \end{abstract}
%
% \tableofcontents
%
% \section{Documentation}
%
% \subsection{Background}
%
% This packages is an answer of a newsgroup question:
% \begin{quote}
% \begin{tabular}{@{}ll@{}}
%   Newsgroup: & comp.text.tex\\
%   Subject: & Finding the Document Class\\
%   From: & Herber Schulz\\
%   Date: & 18 Jun 2005 13:16:49 -0500\\
%   Message-ID: &
%    \textless
%    \texttt{herbs-D55DB9.13170418062005@news.isp.giganews.com}^^A
%    \textgreater
% \end{tabular}
% \end{quote}
%
% \subsection{Usage}
%
% Load this package before \cs{documentclass}:
% \begin{quote}
%   |\RequirePackage{classlist}|\\
%   |\documentclass[some,options]{whatever}|
% \end{quote}
% It then records the classes with options.
%
% If used after \cs{documentclass}, \cs{@filelist} is
% parsed for classes. The additional data
% specified options and requested version is no
% longer available here.
%
% \begin{description}
% \item[\cs{MainClassName}] contains the first loaded class.
% \item[\cs{ClassList}] stores the class entries, eg.
%   \begin{quote}
%   \begin{tabular}{@{}l@{ }l@{}}
%     \cs{ClassList} $\rightarrow$&
%     |\ClassListEntry{myarticle}{a4paper}{}|\\
%     &|\ClassListEntry{article}{}{}|
%   \end{tabular}
%   \end{quote}
% \item[\cs{ClassListEntry}] has three arguments:
%   \begin{quote}
%   \begin{tabular}{@{}ll@{}}
%     |#1|: & class name\\
%     |#2|: & options given in \cs{documentclass}/\cs{LoadClass}\\
%     |#3|: & requested version, not the version of class
%   \end{tabular}
%   \end{quote}
% \item[\cs{PrintClassList}] prints the list on screen it can be
%    configured by
% \item[\cs{PrintClassListTitle}] for the title and
% \item[\cs{PrintClassListEntry}] for formatting the entries.
%    See the implementation for how to use these.
% \end{description}
%
% \StopEventually{
% }
%
% \section{Implementation}
%
%    \begin{macrocode}
%<*package>
%    \end{macrocode}
%    Package identification.
%    \begin{macrocode}
\NeedsTeXFormat{LaTeX2e}
\ProvidesPackage{classlist}%
  [2016/05/16 v1.5 Record classes used in a document (HO)]
%    \end{macrocode}
%
%    \begin{macrocode}
\let\ClassList\@empty
\let\MainClassName\relax
%    \end{macrocode}
%
%    Test, whether we are called before \cs{documentclass}.
%    \begin{macrocode}
\ifx\@classoptionslist\relax
  \let\CL@org@fileswith@pti@ns\@fileswith@pti@ns
  \def\@fileswith@pti@ns#1[#2]#3[#4]{%
%    \end{macrocode}
%    \begin{tabular}{@{}ll@{}}
%      |#1|: & \cs{@clsextension}\\
%      |#2|: & options of \cs{documentclass}/\cs{LoadClass}\\
%      |#3|: & class name\\
%      |#4|: & requested version
%    \end{tabular}
%    \begin{macrocode}
    \ifx#1\@clsextension
      \@ifl@aded#1{#3}{%
        \PackageInfo{classlist}{%
          Skipping class `#3', because\MessageBreak
          this class is already loaded%
        }%
      }{%
        \@ifundefined{MainClassName}{%
          \def\MainClassName{#3}%
        }{}%
        \@temptokena\expandafter{%
          \ClassList
          \ClassListEntry{#3}{#2}{#4}%
        }%
        \edef\ClassList{\the\@temptokena}%
      }%
    \fi
    \CL@org@fileswith@pti@ns{#1}[{#2}]{#3}[{#4}]%
  }%
  \let\@@fileswith@pti@ns\@fileswith@pti@ns
\else
%    \end{macrocode}
%    Called after \cs{documentclass}.
%    \begin{macrocode}
  \PackageInfo{classlist}{Use \string\@filelist\space method}%

  \let\ClassListEntry\relax
  \expandafter\def\expandafter\CL@test
      \expandafter#\expandafter1\@clsextension#2\@nil{%
    \ifx\\#2\\%
%    \end{macrocode}
%    Name does not contain \cs{@clsextension}
%    \begin{macrocode}
    \else
      \expandafter\CL@test@i\CL@entry\@nil
    \fi
  }%
  \expandafter\def\expandafter\CL@test@i
      \expandafter#\expandafter1\@clsextension#2\@nil{%
    \ifx\\#2\\%
      \@ifundefined{opt@\CL@entry}{%
      }{%
        \@ifundefined{MainClassName}{%
          \let\MainClassName\CL@entry
        }{%
        }%
        \edef\ClassList{%
          \ClassList
          \ClassListEntry{\CL@entry}{}{}%
        }%
      }%
    \else
%    \end{macrocode}
%    Names with more than one \cs{@clsextension} are not supported.
%    \begin{macrocode}
    \fi
  }%
  \@for\CL@entry:=\@filelist\do{%
    \expandafter\expandafter\expandafter\CL@test\expandafter
        \CL@entry\@clsextension\@nil
  }%
\fi
%    \end{macrocode}
%
%    \begin{macro}{\PrintClassListEntry}
%    \begin{macrocode}
\providecommand*{\PrintClassListEntry}[3]{%
  \toks@{* #1}%
  \typeout{\the\toks@}%
}
%    \end{macrocode}
%    \end{macro}
%    \begin{macro}{\PrintClassListTitle}
%    \begin{macrocode}
\providecommand*{\PrintClassListTitle}{%
  \typeout{Class list:}%
}
%    \end{macrocode}
%    \end{macro}
%    \begin{macro}{\PrintClassList}
%    \begin{macrocode}
\providecommand*{\PrintClassList}{%
  \begingroup
    \let\ClassListEntry\PrintClassListEntry
    \PrintClassListTitle
    \ClassList
  \endgroup
}
%    \end{macrocode}
%    \end{macro}
%    \begin{macro}{\CL@InfoEntry}
%    \begin{macrocode}
\def\CL@InfoEntry#1#2#3{%
  \advance\count@ by \@ne
  \def\x{#2}%
  \@onelevel@sanitize\x
  \edef\CL@Info{%
    \CL@Info
    \noexpand\MessageBreak
    (\the\count@) %
    #1 [\x]%
    \ifx\\#3\\%
    \else
      \space[#3]% hash-ok
    \fi
  }%
}
%    \end{macrocode}
%    \end{macro}
%    \begin{macrocode}
\AtBeginDocument{%
  \begingroup
    \count@=\z@
    \def\CL@Info{Class List:}%
    \let\ClassListEntry\CL@InfoEntry
    \ClassList
    \let\on@line\@empty
    \PackageInfo{classlist}{\CL@Info}%
  \endgroup
}
%    \end{macrocode}
%
%    \begin{macrocode}
%</package>
%    \end{macrocode}
%
% \section{Installation}
%
% \subsection{Download}
%
% \paragraph{Package.} This package is available on
% CTAN\footnote{\CTANpkg{classlist}}:
% \begin{description}
% \item[\CTAN{macros/latex/contrib/oberdiek/classlist.dtx}] The source file.
% \item[\CTAN{macros/latex/contrib/oberdiek/classlist.pdf}] Documentation.
% \end{description}
%
%
% \paragraph{Bundle.} All the packages of the bundle `oberdiek'
% are also available in a TDS compliant ZIP archive. There
% the packages are already unpacked and the documentation files
% are generated. The files and directories obey the TDS standard.
% \begin{description}
% \item[\CTANinstall{install/macros/latex/contrib/oberdiek.tds.zip}]
% \end{description}
% \emph{TDS} refers to the standard ``A Directory Structure
% for \TeX\ Files'' (\CTANpkg{tds}). Directories
% with \xfile{texmf} in their name are usually organized this way.
%
% \subsection{Bundle installation}
%
% \paragraph{Unpacking.} Unpack the \xfile{oberdiek.tds.zip} in the
% TDS tree (also known as \xfile{texmf} tree) of your choice.
% Example (linux):
% \begin{quote}
%   |unzip oberdiek.tds.zip -d ~/texmf|
% \end{quote}
%
% \subsection{Package installation}
%
% \paragraph{Unpacking.} The \xfile{.dtx} file is a self-extracting
% \docstrip\ archive. The files are extracted by running the
% \xfile{.dtx} through \plainTeX:
% \begin{quote}
%   \verb|tex classlist.dtx|
% \end{quote}
%
% \paragraph{TDS.} Now the different files must be moved into
% the different directories in your installation TDS tree
% (also known as \xfile{texmf} tree):
% \begin{quote}
% \def\t{^^A
% \begin{tabular}{@{}>{\ttfamily}l@{ $\rightarrow$ }>{\ttfamily}l@{}}
%   classlist.sty & tex/latex/oberdiek/classlist.sty\\
%   classlist.pdf & doc/latex/oberdiek/classlist.pdf\\
%   classlist.dtx & source/latex/oberdiek/classlist.dtx\\
% \end{tabular}^^A
% }^^A
% \sbox0{\t}^^A
% \ifdim\wd0>\linewidth
%   \begingroup
%     \advance\linewidth by\leftmargin
%     \advance\linewidth by\rightmargin
%   \edef\x{\endgroup
%     \def\noexpand\lw{\the\linewidth}^^A
%   }\x
%   \def\lwbox{^^A
%     \leavevmode
%     \hbox to \linewidth{^^A
%       \kern-\leftmargin\relax
%       \hss
%       \usebox0
%       \hss
%       \kern-\rightmargin\relax
%     }^^A
%   }^^A
%   \ifdim\wd0>\lw
%     \sbox0{\small\t}^^A
%     \ifdim\wd0>\linewidth
%       \ifdim\wd0>\lw
%         \sbox0{\footnotesize\t}^^A
%         \ifdim\wd0>\linewidth
%           \ifdim\wd0>\lw
%             \sbox0{\scriptsize\t}^^A
%             \ifdim\wd0>\linewidth
%               \ifdim\wd0>\lw
%                 \sbox0{\tiny\t}^^A
%                 \ifdim\wd0>\linewidth
%                   \lwbox
%                 \else
%                   \usebox0
%                 \fi
%               \else
%                 \lwbox
%               \fi
%             \else
%               \usebox0
%             \fi
%           \else
%             \lwbox
%           \fi
%         \else
%           \usebox0
%         \fi
%       \else
%         \lwbox
%       \fi
%     \else
%       \usebox0
%     \fi
%   \else
%     \lwbox
%   \fi
% \else
%   \usebox0
% \fi
% \end{quote}
% If you have a \xfile{docstrip.cfg} that configures and enables \docstrip's
% TDS installing feature, then some files can already be in the right
% place, see the documentation of \docstrip.
%
% \subsection{Refresh file name databases}
%
% If your \TeX~distribution
% (\TeX\,Live, \mikTeX, \dots) relies on file name databases, you must refresh
% these. For example, \TeX\,Live\ users run \verb|texhash| or
% \verb|mktexlsr|.
%
% \subsection{Some details for the interested}
%
% \paragraph{Unpacking with \LaTeX.}
% The \xfile{.dtx} chooses its action depending on the format:
% \begin{description}
% \item[\plainTeX:] Run \docstrip\ and extract the files.
% \item[\LaTeX:] Generate the documentation.
% \end{description}
% If you insist on using \LaTeX\ for \docstrip\ (really,
% \docstrip\ does not need \LaTeX), then inform the autodetect routine
% about your intention:
% \begin{quote}
%   \verb|latex \let\install=y\input{classlist.dtx}|
% \end{quote}
% Do not forget to quote the argument according to the demands
% of your shell.
%
% \paragraph{Generating the documentation.}
% You can use both the \xfile{.dtx} or the \xfile{.drv} to generate
% the documentation. The process can be configured by the
% configuration file \xfile{ltxdoc.cfg}. For instance, put this
% line into this file, if you want to have A4 as paper format:
% \begin{quote}
%   \verb|\PassOptionsToClass{a4paper}{article}|
% \end{quote}
% An example follows how to generate the
% documentation with pdf\LaTeX:
% \begin{quote}
%\begin{verbatim}
%pdflatex classlist.dtx
%makeindex -s gind.ist classlist.idx
%pdflatex classlist.dtx
%makeindex -s gind.ist classlist.idx
%pdflatex classlist.dtx
%\end{verbatim}
% \end{quote}
%
% \begin{History}
%   \begin{Version}{2005/06/19 v1.0}
%   \item
%     First published version: CTAN and newsgroup \xnewsgroup{comp.text.tex}:
%     \URL{``\link{Re: Finding the Document Class}''}^^A
%     {https://groups.google.com/group/comp.text.tex/msg/8ee9523c2dc13666}
%   \end{Version}
%   \begin{Version}{2005/06/19 v1.1}
%   \item
%     After \cs{documentclass} the package looks
%     at \cs{@filelist} instead of aborting with error.
%   \end{Version}
%   \begin{Version}{2006/02/20 v1.2}
%   \item
%     DTX framework.
%   \item
%     Fix for \cs{@@fileswith@pti@ns}.
%   \end{Version}
%   \begin{Version}{2008/08/11 v1.3}
%   \item
%     Code is not changed.
%   \item
%     URLs updated.
%   \end{Version}
%   \begin{Version}{2011/10/17 v1.4}
%   \item
%     Documentation fix: \cs{MainClass} $\rightarrow$ \cs{MainClassName}.
%   \end{Version}
%   \begin{Version}{2016/05/16 v1.5}
%   \item
%     Documentation updates.
%   \end{Version}
% \end{History}
%
% \PrintIndex
%
% \Finale
\endinput
|
% \end{quote}
% Do not forget to quote the argument according to the demands
% of your shell.
%
% \paragraph{Generating the documentation.}
% You can use both the \xfile{.dtx} or the \xfile{.drv} to generate
% the documentation. The process can be configured by the
% configuration file \xfile{ltxdoc.cfg}. For instance, put this
% line into this file, if you want to have A4 as paper format:
% \begin{quote}
%   \verb|\PassOptionsToClass{a4paper}{article}|
% \end{quote}
% An example follows how to generate the
% documentation with pdf\LaTeX:
% \begin{quote}
%\begin{verbatim}
%pdflatex classlist.dtx
%makeindex -s gind.ist classlist.idx
%pdflatex classlist.dtx
%makeindex -s gind.ist classlist.idx
%pdflatex classlist.dtx
%\end{verbatim}
% \end{quote}
%
% \begin{History}
%   \begin{Version}{2005/06/19 v1.0}
%   \item
%     First published version: CTAN and newsgroup \xnewsgroup{comp.text.tex}:
%     \URL{``\link{Re: Finding the Document Class}''}^^A
%     {https://groups.google.com/group/comp.text.tex/msg/8ee9523c2dc13666}
%   \end{Version}
%   \begin{Version}{2005/06/19 v1.1}
%   \item
%     After \cs{documentclass} the package looks
%     at \cs{@filelist} instead of aborting with error.
%   \end{Version}
%   \begin{Version}{2006/02/20 v1.2}
%   \item
%     DTX framework.
%   \item
%     Fix for \cs{@@fileswith@pti@ns}.
%   \end{Version}
%   \begin{Version}{2008/08/11 v1.3}
%   \item
%     Code is not changed.
%   \item
%     URLs updated.
%   \end{Version}
%   \begin{Version}{2011/10/17 v1.4}
%   \item
%     Documentation fix: \cs{MainClass} $\rightarrow$ \cs{MainClassName}.
%   \end{Version}
%   \begin{Version}{2016/05/16 v1.5}
%   \item
%     Documentation updates.
%   \end{Version}
% \end{History}
%
% \PrintIndex
%
% \Finale
\endinput

%        (quote the arguments according to the demands of your shell)
%
% Documentation:
%    (a) If classlist.drv is present:
%           latex classlist.drv
%    (b) Without classlist.drv:
%           latex classlist.dtx; ...
%    The class ltxdoc loads the configuration file ltxdoc.cfg
%    if available. Here you can specify further options, e.g.
%    use A4 as paper format:
%       \PassOptionsToClass{a4paper}{article}
%
%    Program calls to get the documentation (example):
%       pdflatex classlist.dtx
%       makeindex -s gind.ist classlist.idx
%       pdflatex classlist.dtx
%       makeindex -s gind.ist classlist.idx
%       pdflatex classlist.dtx
%
% Installation:
%    TDS:tex/latex/oberdiek/classlist.sty
%    TDS:doc/latex/oberdiek/classlist.pdf
%    TDS:source/latex/oberdiek/classlist.dtx
%
%<*ignore>
\begingroup
  \catcode123=1 %
  \catcode125=2 %
  \def\x{LaTeX2e}%
\expandafter\endgroup
\ifcase 0\ifx\install y1\fi\expandafter
         \ifx\csname processbatchFile\endcsname\relax\else1\fi
         \ifx\fmtname\x\else 1\fi\relax
\else\csname fi\endcsname
%</ignore>
%<*install>
\input docstrip.tex
\Msg{************************************************************************}
\Msg{* Installation}
\Msg{* Package: classlist 2016/05/16 v1.5 Record classes used in a document (HO)}
\Msg{************************************************************************}

\keepsilent
\askforoverwritefalse

\let\MetaPrefix\relax
\preamble

This is a generated file.

Project: classlist
Version: 2016/05/16 v1.5

Copyright (C)
   2005, 2006, 2008, 2011 Heiko Oberdiek
   2016-2019 Oberdiek Package Support Group

This work may be distributed and/or modified under the
conditions of the LaTeX Project Public License, either
version 1.3c of this license or (at your option) any later
version. This version of this license is in
   https://www.latex-project.org/lppl/lppl-1-3c.txt
and the latest version of this license is in
   https://www.latex-project.org/lppl.txt
and version 1.3 or later is part of all distributions of
LaTeX version 2005/12/01 or later.

This work has the LPPL maintenance status "maintained".

The Current Maintainers of this work are
Heiko Oberdiek and the Oberdiek Package Support Group
https://github.com/ho-tex/oberdiek/issues


This work consists of the main source file classlist.dtx
and the derived files
   classlist.sty, classlist.pdf, classlist.ins, classlist.drv.

\endpreamble
\let\MetaPrefix\DoubleperCent

\generate{%
  \file{classlist.ins}{\from{classlist.dtx}{install}}%
  \file{classlist.drv}{\from{classlist.dtx}{driver}}%
  \usedir{tex/latex/oberdiek}%
  \file{classlist.sty}{\from{classlist.dtx}{package}}%
}

\catcode32=13\relax% active space
\let =\space%
\Msg{************************************************************************}
\Msg{*}
\Msg{* To finish the installation you have to move the following}
\Msg{* file into a directory searched by TeX:}
\Msg{*}
\Msg{*     classlist.sty}
\Msg{*}
\Msg{* To produce the documentation run the file `classlist.drv'}
\Msg{* through LaTeX.}
\Msg{*}
\Msg{* Happy TeXing!}
\Msg{*}
\Msg{************************************************************************}

\endbatchfile
%</install>
%<*ignore>
\fi
%</ignore>
%<*driver>
\NeedsTeXFormat{LaTeX2e}
\ProvidesFile{classlist.drv}%
  [2016/05/16 v1.5 Record classes used in a document (HO)]%
\documentclass{ltxdoc}
\usepackage{holtxdoc}[2011/11/22]
\begin{document}
  \DocInput{classlist.dtx}%
\end{document}
%</driver>
% \fi
%
%
%
% \GetFileInfo{classlist.drv}
%
% \title{The \xpackage{classlist} package}
% \date{2016/05/16 v1.5}
% \author{Heiko Oberdiek\thanks
% {Please report any issues at \url{https://github.com/ho-tex/oberdiek/issues}}}
%
% \maketitle
%
% \begin{abstract}
% This package records the loaded classes and stores
% them in a list.
% \end{abstract}
%
% \tableofcontents
%
% \section{Documentation}
%
% \subsection{Background}
%
% This packages is an answer of a newsgroup question:
% \begin{quote}
% \begin{tabular}{@{}ll@{}}
%   Newsgroup: & comp.text.tex\\
%   Subject: & Finding the Document Class\\
%   From: & Herber Schulz\\
%   Date: & 18 Jun 2005 13:16:49 -0500\\
%   Message-ID: &
%    \textless
%    \texttt{herbs-D55DB9.13170418062005@news.isp.giganews.com}^^A
%    \textgreater
% \end{tabular}
% \end{quote}
%
% \subsection{Usage}
%
% Load this package before \cs{documentclass}:
% \begin{quote}
%   |\RequirePackage{classlist}|\\
%   |\documentclass[some,options]{whatever}|
% \end{quote}
% It then records the classes with options.
%
% If used after \cs{documentclass}, \cs{@filelist} is
% parsed for classes. The additional data
% specified options and requested version is no
% longer available here.
%
% \begin{description}
% \item[\cs{MainClassName}] contains the first loaded class.
% \item[\cs{ClassList}] stores the class entries, eg.
%   \begin{quote}
%   \begin{tabular}{@{}l@{ }l@{}}
%     \cs{ClassList} $\rightarrow$&
%     |\ClassListEntry{myarticle}{a4paper}{}|\\
%     &|\ClassListEntry{article}{}{}|
%   \end{tabular}
%   \end{quote}
% \item[\cs{ClassListEntry}] has three arguments:
%   \begin{quote}
%   \begin{tabular}{@{}ll@{}}
%     |#1|: & class name\\
%     |#2|: & options given in \cs{documentclass}/\cs{LoadClass}\\
%     |#3|: & requested version, not the version of class
%   \end{tabular}
%   \end{quote}
% \item[\cs{PrintClassList}] prints the list on screen it can be
%    configured by
% \item[\cs{PrintClassListTitle}] for the title and
% \item[\cs{PrintClassListEntry}] for formatting the entries.
%    See the implementation for how to use these.
% \end{description}
%
% \StopEventually{
% }
%
% \section{Implementation}
%
%    \begin{macrocode}
%<*package>
%    \end{macrocode}
%    Package identification.
%    \begin{macrocode}
\NeedsTeXFormat{LaTeX2e}
\ProvidesPackage{classlist}%
  [2016/05/16 v1.5 Record classes used in a document (HO)]
%    \end{macrocode}
%
%    \begin{macrocode}
\let\ClassList\@empty
\let\MainClassName\relax
%    \end{macrocode}
%
%    Test, whether we are called before \cs{documentclass}.
%    \begin{macrocode}
\ifx\@classoptionslist\relax
  \let\CL@org@fileswith@pti@ns\@fileswith@pti@ns
  \def\@fileswith@pti@ns#1[#2]#3[#4]{%
%    \end{macrocode}
%    \begin{tabular}{@{}ll@{}}
%      |#1|: & \cs{@clsextension}\\
%      |#2|: & options of \cs{documentclass}/\cs{LoadClass}\\
%      |#3|: & class name\\
%      |#4|: & requested version
%    \end{tabular}
%    \begin{macrocode}
    \ifx#1\@clsextension
      \@ifl@aded#1{#3}{%
        \PackageInfo{classlist}{%
          Skipping class `#3', because\MessageBreak
          this class is already loaded%
        }%
      }{%
        \@ifundefined{MainClassName}{%
          \def\MainClassName{#3}%
        }{}%
        \@temptokena\expandafter{%
          \ClassList
          \ClassListEntry{#3}{#2}{#4}%
        }%
        \edef\ClassList{\the\@temptokena}%
      }%
    \fi
    \CL@org@fileswith@pti@ns{#1}[{#2}]{#3}[{#4}]%
  }%
  \let\@@fileswith@pti@ns\@fileswith@pti@ns
\else
%    \end{macrocode}
%    Called after \cs{documentclass}.
%    \begin{macrocode}
  \PackageInfo{classlist}{Use \string\@filelist\space method}%

  \let\ClassListEntry\relax
  \expandafter\def\expandafter\CL@test
      \expandafter#\expandafter1\@clsextension#2\@nil{%
    \ifx\\#2\\%
%    \end{macrocode}
%    Name does not contain \cs{@clsextension}
%    \begin{macrocode}
    \else
      \expandafter\CL@test@i\CL@entry\@nil
    \fi
  }%
  \expandafter\def\expandafter\CL@test@i
      \expandafter#\expandafter1\@clsextension#2\@nil{%
    \ifx\\#2\\%
      \@ifundefined{opt@\CL@entry}{%
      }{%
        \@ifundefined{MainClassName}{%
          \let\MainClassName\CL@entry
        }{%
        }%
        \edef\ClassList{%
          \ClassList
          \ClassListEntry{\CL@entry}{}{}%
        }%
      }%
    \else
%    \end{macrocode}
%    Names with more than one \cs{@clsextension} are not supported.
%    \begin{macrocode}
    \fi
  }%
  \@for\CL@entry:=\@filelist\do{%
    \expandafter\expandafter\expandafter\CL@test\expandafter
        \CL@entry\@clsextension\@nil
  }%
\fi
%    \end{macrocode}
%
%    \begin{macro}{\PrintClassListEntry}
%    \begin{macrocode}
\providecommand*{\PrintClassListEntry}[3]{%
  \toks@{* #1}%
  \typeout{\the\toks@}%
}
%    \end{macrocode}
%    \end{macro}
%    \begin{macro}{\PrintClassListTitle}
%    \begin{macrocode}
\providecommand*{\PrintClassListTitle}{%
  \typeout{Class list:}%
}
%    \end{macrocode}
%    \end{macro}
%    \begin{macro}{\PrintClassList}
%    \begin{macrocode}
\providecommand*{\PrintClassList}{%
  \begingroup
    \let\ClassListEntry\PrintClassListEntry
    \PrintClassListTitle
    \ClassList
  \endgroup
}
%    \end{macrocode}
%    \end{macro}
%    \begin{macro}{\CL@InfoEntry}
%    \begin{macrocode}
\def\CL@InfoEntry#1#2#3{%
  \advance\count@ by \@ne
  \def\x{#2}%
  \@onelevel@sanitize\x
  \edef\CL@Info{%
    \CL@Info
    \noexpand\MessageBreak
    (\the\count@) %
    #1 [\x]%
    \ifx\\#3\\%
    \else
      \space[#3]% hash-ok
    \fi
  }%
}
%    \end{macrocode}
%    \end{macro}
%    \begin{macrocode}
\AtBeginDocument{%
  \begingroup
    \count@=\z@
    \def\CL@Info{Class List:}%
    \let\ClassListEntry\CL@InfoEntry
    \ClassList
    \let\on@line\@empty
    \PackageInfo{classlist}{\CL@Info}%
  \endgroup
}
%    \end{macrocode}
%
%    \begin{macrocode}
%</package>
%    \end{macrocode}
%
% \section{Installation}
%
% \subsection{Download}
%
% \paragraph{Package.} This package is available on
% CTAN\footnote{\CTANpkg{classlist}}:
% \begin{description}
% \item[\CTAN{macros/latex/contrib/oberdiek/classlist.dtx}] The source file.
% \item[\CTAN{macros/latex/contrib/oberdiek/classlist.pdf}] Documentation.
% \end{description}
%
%
% \paragraph{Bundle.} All the packages of the bundle `oberdiek'
% are also available in a TDS compliant ZIP archive. There
% the packages are already unpacked and the documentation files
% are generated. The files and directories obey the TDS standard.
% \begin{description}
% \item[\CTANinstall{install/macros/latex/contrib/oberdiek.tds.zip}]
% \end{description}
% \emph{TDS} refers to the standard ``A Directory Structure
% for \TeX\ Files'' (\CTANpkg{tds}). Directories
% with \xfile{texmf} in their name are usually organized this way.
%
% \subsection{Bundle installation}
%
% \paragraph{Unpacking.} Unpack the \xfile{oberdiek.tds.zip} in the
% TDS tree (also known as \xfile{texmf} tree) of your choice.
% Example (linux):
% \begin{quote}
%   |unzip oberdiek.tds.zip -d ~/texmf|
% \end{quote}
%
% \subsection{Package installation}
%
% \paragraph{Unpacking.} The \xfile{.dtx} file is a self-extracting
% \docstrip\ archive. The files are extracted by running the
% \xfile{.dtx} through \plainTeX:
% \begin{quote}
%   \verb|tex classlist.dtx|
% \end{quote}
%
% \paragraph{TDS.} Now the different files must be moved into
% the different directories in your installation TDS tree
% (also known as \xfile{texmf} tree):
% \begin{quote}
% \def\t{^^A
% \begin{tabular}{@{}>{\ttfamily}l@{ $\rightarrow$ }>{\ttfamily}l@{}}
%   classlist.sty & tex/latex/oberdiek/classlist.sty\\
%   classlist.pdf & doc/latex/oberdiek/classlist.pdf\\
%   classlist.dtx & source/latex/oberdiek/classlist.dtx\\
% \end{tabular}^^A
% }^^A
% \sbox0{\t}^^A
% \ifdim\wd0>\linewidth
%   \begingroup
%     \advance\linewidth by\leftmargin
%     \advance\linewidth by\rightmargin
%   \edef\x{\endgroup
%     \def\noexpand\lw{\the\linewidth}^^A
%   }\x
%   \def\lwbox{^^A
%     \leavevmode
%     \hbox to \linewidth{^^A
%       \kern-\leftmargin\relax
%       \hss
%       \usebox0
%       \hss
%       \kern-\rightmargin\relax
%     }^^A
%   }^^A
%   \ifdim\wd0>\lw
%     \sbox0{\small\t}^^A
%     \ifdim\wd0>\linewidth
%       \ifdim\wd0>\lw
%         \sbox0{\footnotesize\t}^^A
%         \ifdim\wd0>\linewidth
%           \ifdim\wd0>\lw
%             \sbox0{\scriptsize\t}^^A
%             \ifdim\wd0>\linewidth
%               \ifdim\wd0>\lw
%                 \sbox0{\tiny\t}^^A
%                 \ifdim\wd0>\linewidth
%                   \lwbox
%                 \else
%                   \usebox0
%                 \fi
%               \else
%                 \lwbox
%               \fi
%             \else
%               \usebox0
%             \fi
%           \else
%             \lwbox
%           \fi
%         \else
%           \usebox0
%         \fi
%       \else
%         \lwbox
%       \fi
%     \else
%       \usebox0
%     \fi
%   \else
%     \lwbox
%   \fi
% \else
%   \usebox0
% \fi
% \end{quote}
% If you have a \xfile{docstrip.cfg} that configures and enables \docstrip's
% TDS installing feature, then some files can already be in the right
% place, see the documentation of \docstrip.
%
% \subsection{Refresh file name databases}
%
% If your \TeX~distribution
% (\TeX\,Live, \mikTeX, \dots) relies on file name databases, you must refresh
% these. For example, \TeX\,Live\ users run \verb|texhash| or
% \verb|mktexlsr|.
%
% \subsection{Some details for the interested}
%
% \paragraph{Unpacking with \LaTeX.}
% The \xfile{.dtx} chooses its action depending on the format:
% \begin{description}
% \item[\plainTeX:] Run \docstrip\ and extract the files.
% \item[\LaTeX:] Generate the documentation.
% \end{description}
% If you insist on using \LaTeX\ for \docstrip\ (really,
% \docstrip\ does not need \LaTeX), then inform the autodetect routine
% about your intention:
% \begin{quote}
%   \verb|latex \let\install=y% \iffalse meta-comment
%
% File: classlist.dtx
% Version: 2016/05/16 v1.5
% Info: Record classes used in a document
%
% Copyright (C)
%    2005, 2006, 2008, 2011 Heiko Oberdiek
%    2016-2019 Oberdiek Package Support Group
%    https://github.com/ho-tex/oberdiek/issues
%
% This work may be distributed and/or modified under the
% conditions of the LaTeX Project Public License, either
% version 1.3c of this license or (at your option) any later
% version. This version of this license is in
%    https://www.latex-project.org/lppl/lppl-1-3c.txt
% and the latest version of this license is in
%    https://www.latex-project.org/lppl.txt
% and version 1.3 or later is part of all distributions of
% LaTeX version 2005/12/01 or later.
%
% This work has the LPPL maintenance status "maintained".
%
% The Current Maintainers of this work are
% Heiko Oberdiek and the Oberdiek Package Support Group
% https://github.com/ho-tex/oberdiek/issues
%
% This work consists of the main source file classlist.dtx
% and the derived files
%    classlist.sty, classlist.pdf, classlist.ins, classlist.drv.
%
% Distribution:
%    CTAN:macros/latex/contrib/oberdiek/classlist.dtx
%    CTAN:macros/latex/contrib/oberdiek/classlist.pdf
%
% Unpacking:
%    (a) If classlist.ins is present:
%           tex classlist.ins
%    (b) Without classlist.ins:
%           tex classlist.dtx
%    (c) If you insist on using LaTeX
%           latex \let\install=y% \iffalse meta-comment
%
% File: classlist.dtx
% Version: 2016/05/16 v1.5
% Info: Record classes used in a document
%
% Copyright (C)
%    2005, 2006, 2008, 2011 Heiko Oberdiek
%    2016-2019 Oberdiek Package Support Group
%    https://github.com/ho-tex/oberdiek/issues
%
% This work may be distributed and/or modified under the
% conditions of the LaTeX Project Public License, either
% version 1.3c of this license or (at your option) any later
% version. This version of this license is in
%    https://www.latex-project.org/lppl/lppl-1-3c.txt
% and the latest version of this license is in
%    https://www.latex-project.org/lppl.txt
% and version 1.3 or later is part of all distributions of
% LaTeX version 2005/12/01 or later.
%
% This work has the LPPL maintenance status "maintained".
%
% The Current Maintainers of this work are
% Heiko Oberdiek and the Oberdiek Package Support Group
% https://github.com/ho-tex/oberdiek/issues
%
% This work consists of the main source file classlist.dtx
% and the derived files
%    classlist.sty, classlist.pdf, classlist.ins, classlist.drv.
%
% Distribution:
%    CTAN:macros/latex/contrib/oberdiek/classlist.dtx
%    CTAN:macros/latex/contrib/oberdiek/classlist.pdf
%
% Unpacking:
%    (a) If classlist.ins is present:
%           tex classlist.ins
%    (b) Without classlist.ins:
%           tex classlist.dtx
%    (c) If you insist on using LaTeX
%           latex \let\install=y\input{classlist.dtx}
%        (quote the arguments according to the demands of your shell)
%
% Documentation:
%    (a) If classlist.drv is present:
%           latex classlist.drv
%    (b) Without classlist.drv:
%           latex classlist.dtx; ...
%    The class ltxdoc loads the configuration file ltxdoc.cfg
%    if available. Here you can specify further options, e.g.
%    use A4 as paper format:
%       \PassOptionsToClass{a4paper}{article}
%
%    Program calls to get the documentation (example):
%       pdflatex classlist.dtx
%       makeindex -s gind.ist classlist.idx
%       pdflatex classlist.dtx
%       makeindex -s gind.ist classlist.idx
%       pdflatex classlist.dtx
%
% Installation:
%    TDS:tex/latex/oberdiek/classlist.sty
%    TDS:doc/latex/oberdiek/classlist.pdf
%    TDS:source/latex/oberdiek/classlist.dtx
%
%<*ignore>
\begingroup
  \catcode123=1 %
  \catcode125=2 %
  \def\x{LaTeX2e}%
\expandafter\endgroup
\ifcase 0\ifx\install y1\fi\expandafter
         \ifx\csname processbatchFile\endcsname\relax\else1\fi
         \ifx\fmtname\x\else 1\fi\relax
\else\csname fi\endcsname
%</ignore>
%<*install>
\input docstrip.tex
\Msg{************************************************************************}
\Msg{* Installation}
\Msg{* Package: classlist 2016/05/16 v1.5 Record classes used in a document (HO)}
\Msg{************************************************************************}

\keepsilent
\askforoverwritefalse

\let\MetaPrefix\relax
\preamble

This is a generated file.

Project: classlist
Version: 2016/05/16 v1.5

Copyright (C)
   2005, 2006, 2008, 2011 Heiko Oberdiek
   2016-2019 Oberdiek Package Support Group

This work may be distributed and/or modified under the
conditions of the LaTeX Project Public License, either
version 1.3c of this license or (at your option) any later
version. This version of this license is in
   https://www.latex-project.org/lppl/lppl-1-3c.txt
and the latest version of this license is in
   https://www.latex-project.org/lppl.txt
and version 1.3 or later is part of all distributions of
LaTeX version 2005/12/01 or later.

This work has the LPPL maintenance status "maintained".

The Current Maintainers of this work are
Heiko Oberdiek and the Oberdiek Package Support Group
https://github.com/ho-tex/oberdiek/issues


This work consists of the main source file classlist.dtx
and the derived files
   classlist.sty, classlist.pdf, classlist.ins, classlist.drv.

\endpreamble
\let\MetaPrefix\DoubleperCent

\generate{%
  \file{classlist.ins}{\from{classlist.dtx}{install}}%
  \file{classlist.drv}{\from{classlist.dtx}{driver}}%
  \usedir{tex/latex/oberdiek}%
  \file{classlist.sty}{\from{classlist.dtx}{package}}%
}

\catcode32=13\relax% active space
\let =\space%
\Msg{************************************************************************}
\Msg{*}
\Msg{* To finish the installation you have to move the following}
\Msg{* file into a directory searched by TeX:}
\Msg{*}
\Msg{*     classlist.sty}
\Msg{*}
\Msg{* To produce the documentation run the file `classlist.drv'}
\Msg{* through LaTeX.}
\Msg{*}
\Msg{* Happy TeXing!}
\Msg{*}
\Msg{************************************************************************}

\endbatchfile
%</install>
%<*ignore>
\fi
%</ignore>
%<*driver>
\NeedsTeXFormat{LaTeX2e}
\ProvidesFile{classlist.drv}%
  [2016/05/16 v1.5 Record classes used in a document (HO)]%
\documentclass{ltxdoc}
\usepackage{holtxdoc}[2011/11/22]
\begin{document}
  \DocInput{classlist.dtx}%
\end{document}
%</driver>
% \fi
%
%
%
% \GetFileInfo{classlist.drv}
%
% \title{The \xpackage{classlist} package}
% \date{2016/05/16 v1.5}
% \author{Heiko Oberdiek\thanks
% {Please report any issues at \url{https://github.com/ho-tex/oberdiek/issues}}}
%
% \maketitle
%
% \begin{abstract}
% This package records the loaded classes and stores
% them in a list.
% \end{abstract}
%
% \tableofcontents
%
% \section{Documentation}
%
% \subsection{Background}
%
% This packages is an answer of a newsgroup question:
% \begin{quote}
% \begin{tabular}{@{}ll@{}}
%   Newsgroup: & comp.text.tex\\
%   Subject: & Finding the Document Class\\
%   From: & Herber Schulz\\
%   Date: & 18 Jun 2005 13:16:49 -0500\\
%   Message-ID: &
%    \textless
%    \texttt{herbs-D55DB9.13170418062005@news.isp.giganews.com}^^A
%    \textgreater
% \end{tabular}
% \end{quote}
%
% \subsection{Usage}
%
% Load this package before \cs{documentclass}:
% \begin{quote}
%   |\RequirePackage{classlist}|\\
%   |\documentclass[some,options]{whatever}|
% \end{quote}
% It then records the classes with options.
%
% If used after \cs{documentclass}, \cs{@filelist} is
% parsed for classes. The additional data
% specified options and requested version is no
% longer available here.
%
% \begin{description}
% \item[\cs{MainClassName}] contains the first loaded class.
% \item[\cs{ClassList}] stores the class entries, eg.
%   \begin{quote}
%   \begin{tabular}{@{}l@{ }l@{}}
%     \cs{ClassList} $\rightarrow$&
%     |\ClassListEntry{myarticle}{a4paper}{}|\\
%     &|\ClassListEntry{article}{}{}|
%   \end{tabular}
%   \end{quote}
% \item[\cs{ClassListEntry}] has three arguments:
%   \begin{quote}
%   \begin{tabular}{@{}ll@{}}
%     |#1|: & class name\\
%     |#2|: & options given in \cs{documentclass}/\cs{LoadClass}\\
%     |#3|: & requested version, not the version of class
%   \end{tabular}
%   \end{quote}
% \item[\cs{PrintClassList}] prints the list on screen it can be
%    configured by
% \item[\cs{PrintClassListTitle}] for the title and
% \item[\cs{PrintClassListEntry}] for formatting the entries.
%    See the implementation for how to use these.
% \end{description}
%
% \StopEventually{
% }
%
% \section{Implementation}
%
%    \begin{macrocode}
%<*package>
%    \end{macrocode}
%    Package identification.
%    \begin{macrocode}
\NeedsTeXFormat{LaTeX2e}
\ProvidesPackage{classlist}%
  [2016/05/16 v1.5 Record classes used in a document (HO)]
%    \end{macrocode}
%
%    \begin{macrocode}
\let\ClassList\@empty
\let\MainClassName\relax
%    \end{macrocode}
%
%    Test, whether we are called before \cs{documentclass}.
%    \begin{macrocode}
\ifx\@classoptionslist\relax
  \let\CL@org@fileswith@pti@ns\@fileswith@pti@ns
  \def\@fileswith@pti@ns#1[#2]#3[#4]{%
%    \end{macrocode}
%    \begin{tabular}{@{}ll@{}}
%      |#1|: & \cs{@clsextension}\\
%      |#2|: & options of \cs{documentclass}/\cs{LoadClass}\\
%      |#3|: & class name\\
%      |#4|: & requested version
%    \end{tabular}
%    \begin{macrocode}
    \ifx#1\@clsextension
      \@ifl@aded#1{#3}{%
        \PackageInfo{classlist}{%
          Skipping class `#3', because\MessageBreak
          this class is already loaded%
        }%
      }{%
        \@ifundefined{MainClassName}{%
          \def\MainClassName{#3}%
        }{}%
        \@temptokena\expandafter{%
          \ClassList
          \ClassListEntry{#3}{#2}{#4}%
        }%
        \edef\ClassList{\the\@temptokena}%
      }%
    \fi
    \CL@org@fileswith@pti@ns{#1}[{#2}]{#3}[{#4}]%
  }%
  \let\@@fileswith@pti@ns\@fileswith@pti@ns
\else
%    \end{macrocode}
%    Called after \cs{documentclass}.
%    \begin{macrocode}
  \PackageInfo{classlist}{Use \string\@filelist\space method}%

  \let\ClassListEntry\relax
  \expandafter\def\expandafter\CL@test
      \expandafter#\expandafter1\@clsextension#2\@nil{%
    \ifx\\#2\\%
%    \end{macrocode}
%    Name does not contain \cs{@clsextension}
%    \begin{macrocode}
    \else
      \expandafter\CL@test@i\CL@entry\@nil
    \fi
  }%
  \expandafter\def\expandafter\CL@test@i
      \expandafter#\expandafter1\@clsextension#2\@nil{%
    \ifx\\#2\\%
      \@ifundefined{opt@\CL@entry}{%
      }{%
        \@ifundefined{MainClassName}{%
          \let\MainClassName\CL@entry
        }{%
        }%
        \edef\ClassList{%
          \ClassList
          \ClassListEntry{\CL@entry}{}{}%
        }%
      }%
    \else
%    \end{macrocode}
%    Names with more than one \cs{@clsextension} are not supported.
%    \begin{macrocode}
    \fi
  }%
  \@for\CL@entry:=\@filelist\do{%
    \expandafter\expandafter\expandafter\CL@test\expandafter
        \CL@entry\@clsextension\@nil
  }%
\fi
%    \end{macrocode}
%
%    \begin{macro}{\PrintClassListEntry}
%    \begin{macrocode}
\providecommand*{\PrintClassListEntry}[3]{%
  \toks@{* #1}%
  \typeout{\the\toks@}%
}
%    \end{macrocode}
%    \end{macro}
%    \begin{macro}{\PrintClassListTitle}
%    \begin{macrocode}
\providecommand*{\PrintClassListTitle}{%
  \typeout{Class list:}%
}
%    \end{macrocode}
%    \end{macro}
%    \begin{macro}{\PrintClassList}
%    \begin{macrocode}
\providecommand*{\PrintClassList}{%
  \begingroup
    \let\ClassListEntry\PrintClassListEntry
    \PrintClassListTitle
    \ClassList
  \endgroup
}
%    \end{macrocode}
%    \end{macro}
%    \begin{macro}{\CL@InfoEntry}
%    \begin{macrocode}
\def\CL@InfoEntry#1#2#3{%
  \advance\count@ by \@ne
  \def\x{#2}%
  \@onelevel@sanitize\x
  \edef\CL@Info{%
    \CL@Info
    \noexpand\MessageBreak
    (\the\count@) %
    #1 [\x]%
    \ifx\\#3\\%
    \else
      \space[#3]% hash-ok
    \fi
  }%
}
%    \end{macrocode}
%    \end{macro}
%    \begin{macrocode}
\AtBeginDocument{%
  \begingroup
    \count@=\z@
    \def\CL@Info{Class List:}%
    \let\ClassListEntry\CL@InfoEntry
    \ClassList
    \let\on@line\@empty
    \PackageInfo{classlist}{\CL@Info}%
  \endgroup
}
%    \end{macrocode}
%
%    \begin{macrocode}
%</package>
%    \end{macrocode}
%
% \section{Installation}
%
% \subsection{Download}
%
% \paragraph{Package.} This package is available on
% CTAN\footnote{\CTANpkg{classlist}}:
% \begin{description}
% \item[\CTAN{macros/latex/contrib/oberdiek/classlist.dtx}] The source file.
% \item[\CTAN{macros/latex/contrib/oberdiek/classlist.pdf}] Documentation.
% \end{description}
%
%
% \paragraph{Bundle.} All the packages of the bundle `oberdiek'
% are also available in a TDS compliant ZIP archive. There
% the packages are already unpacked and the documentation files
% are generated. The files and directories obey the TDS standard.
% \begin{description}
% \item[\CTANinstall{install/macros/latex/contrib/oberdiek.tds.zip}]
% \end{description}
% \emph{TDS} refers to the standard ``A Directory Structure
% for \TeX\ Files'' (\CTANpkg{tds}). Directories
% with \xfile{texmf} in their name are usually organized this way.
%
% \subsection{Bundle installation}
%
% \paragraph{Unpacking.} Unpack the \xfile{oberdiek.tds.zip} in the
% TDS tree (also known as \xfile{texmf} tree) of your choice.
% Example (linux):
% \begin{quote}
%   |unzip oberdiek.tds.zip -d ~/texmf|
% \end{quote}
%
% \subsection{Package installation}
%
% \paragraph{Unpacking.} The \xfile{.dtx} file is a self-extracting
% \docstrip\ archive. The files are extracted by running the
% \xfile{.dtx} through \plainTeX:
% \begin{quote}
%   \verb|tex classlist.dtx|
% \end{quote}
%
% \paragraph{TDS.} Now the different files must be moved into
% the different directories in your installation TDS tree
% (also known as \xfile{texmf} tree):
% \begin{quote}
% \def\t{^^A
% \begin{tabular}{@{}>{\ttfamily}l@{ $\rightarrow$ }>{\ttfamily}l@{}}
%   classlist.sty & tex/latex/oberdiek/classlist.sty\\
%   classlist.pdf & doc/latex/oberdiek/classlist.pdf\\
%   classlist.dtx & source/latex/oberdiek/classlist.dtx\\
% \end{tabular}^^A
% }^^A
% \sbox0{\t}^^A
% \ifdim\wd0>\linewidth
%   \begingroup
%     \advance\linewidth by\leftmargin
%     \advance\linewidth by\rightmargin
%   \edef\x{\endgroup
%     \def\noexpand\lw{\the\linewidth}^^A
%   }\x
%   \def\lwbox{^^A
%     \leavevmode
%     \hbox to \linewidth{^^A
%       \kern-\leftmargin\relax
%       \hss
%       \usebox0
%       \hss
%       \kern-\rightmargin\relax
%     }^^A
%   }^^A
%   \ifdim\wd0>\lw
%     \sbox0{\small\t}^^A
%     \ifdim\wd0>\linewidth
%       \ifdim\wd0>\lw
%         \sbox0{\footnotesize\t}^^A
%         \ifdim\wd0>\linewidth
%           \ifdim\wd0>\lw
%             \sbox0{\scriptsize\t}^^A
%             \ifdim\wd0>\linewidth
%               \ifdim\wd0>\lw
%                 \sbox0{\tiny\t}^^A
%                 \ifdim\wd0>\linewidth
%                   \lwbox
%                 \else
%                   \usebox0
%                 \fi
%               \else
%                 \lwbox
%               \fi
%             \else
%               \usebox0
%             \fi
%           \else
%             \lwbox
%           \fi
%         \else
%           \usebox0
%         \fi
%       \else
%         \lwbox
%       \fi
%     \else
%       \usebox0
%     \fi
%   \else
%     \lwbox
%   \fi
% \else
%   \usebox0
% \fi
% \end{quote}
% If you have a \xfile{docstrip.cfg} that configures and enables \docstrip's
% TDS installing feature, then some files can already be in the right
% place, see the documentation of \docstrip.
%
% \subsection{Refresh file name databases}
%
% If your \TeX~distribution
% (\TeX\,Live, \mikTeX, \dots) relies on file name databases, you must refresh
% these. For example, \TeX\,Live\ users run \verb|texhash| or
% \verb|mktexlsr|.
%
% \subsection{Some details for the interested}
%
% \paragraph{Unpacking with \LaTeX.}
% The \xfile{.dtx} chooses its action depending on the format:
% \begin{description}
% \item[\plainTeX:] Run \docstrip\ and extract the files.
% \item[\LaTeX:] Generate the documentation.
% \end{description}
% If you insist on using \LaTeX\ for \docstrip\ (really,
% \docstrip\ does not need \LaTeX), then inform the autodetect routine
% about your intention:
% \begin{quote}
%   \verb|latex \let\install=y\input{classlist.dtx}|
% \end{quote}
% Do not forget to quote the argument according to the demands
% of your shell.
%
% \paragraph{Generating the documentation.}
% You can use both the \xfile{.dtx} or the \xfile{.drv} to generate
% the documentation. The process can be configured by the
% configuration file \xfile{ltxdoc.cfg}. For instance, put this
% line into this file, if you want to have A4 as paper format:
% \begin{quote}
%   \verb|\PassOptionsToClass{a4paper}{article}|
% \end{quote}
% An example follows how to generate the
% documentation with pdf\LaTeX:
% \begin{quote}
%\begin{verbatim}
%pdflatex classlist.dtx
%makeindex -s gind.ist classlist.idx
%pdflatex classlist.dtx
%makeindex -s gind.ist classlist.idx
%pdflatex classlist.dtx
%\end{verbatim}
% \end{quote}
%
% \begin{History}
%   \begin{Version}{2005/06/19 v1.0}
%   \item
%     First published version: CTAN and newsgroup \xnewsgroup{comp.text.tex}:
%     \URL{``\link{Re: Finding the Document Class}''}^^A
%     {https://groups.google.com/group/comp.text.tex/msg/8ee9523c2dc13666}
%   \end{Version}
%   \begin{Version}{2005/06/19 v1.1}
%   \item
%     After \cs{documentclass} the package looks
%     at \cs{@filelist} instead of aborting with error.
%   \end{Version}
%   \begin{Version}{2006/02/20 v1.2}
%   \item
%     DTX framework.
%   \item
%     Fix for \cs{@@fileswith@pti@ns}.
%   \end{Version}
%   \begin{Version}{2008/08/11 v1.3}
%   \item
%     Code is not changed.
%   \item
%     URLs updated.
%   \end{Version}
%   \begin{Version}{2011/10/17 v1.4}
%   \item
%     Documentation fix: \cs{MainClass} $\rightarrow$ \cs{MainClassName}.
%   \end{Version}
%   \begin{Version}{2016/05/16 v1.5}
%   \item
%     Documentation updates.
%   \end{Version}
% \end{History}
%
% \PrintIndex
%
% \Finale
\endinput

%        (quote the arguments according to the demands of your shell)
%
% Documentation:
%    (a) If classlist.drv is present:
%           latex classlist.drv
%    (b) Without classlist.drv:
%           latex classlist.dtx; ...
%    The class ltxdoc loads the configuration file ltxdoc.cfg
%    if available. Here you can specify further options, e.g.
%    use A4 as paper format:
%       \PassOptionsToClass{a4paper}{article}
%
%    Program calls to get the documentation (example):
%       pdflatex classlist.dtx
%       makeindex -s gind.ist classlist.idx
%       pdflatex classlist.dtx
%       makeindex -s gind.ist classlist.idx
%       pdflatex classlist.dtx
%
% Installation:
%    TDS:tex/latex/oberdiek/classlist.sty
%    TDS:doc/latex/oberdiek/classlist.pdf
%    TDS:source/latex/oberdiek/classlist.dtx
%
%<*ignore>
\begingroup
  \catcode123=1 %
  \catcode125=2 %
  \def\x{LaTeX2e}%
\expandafter\endgroup
\ifcase 0\ifx\install y1\fi\expandafter
         \ifx\csname processbatchFile\endcsname\relax\else1\fi
         \ifx\fmtname\x\else 1\fi\relax
\else\csname fi\endcsname
%</ignore>
%<*install>
\input docstrip.tex
\Msg{************************************************************************}
\Msg{* Installation}
\Msg{* Package: classlist 2016/05/16 v1.5 Record classes used in a document (HO)}
\Msg{************************************************************************}

\keepsilent
\askforoverwritefalse

\let\MetaPrefix\relax
\preamble

This is a generated file.

Project: classlist
Version: 2016/05/16 v1.5

Copyright (C)
   2005, 2006, 2008, 2011 Heiko Oberdiek
   2016-2019 Oberdiek Package Support Group

This work may be distributed and/or modified under the
conditions of the LaTeX Project Public License, either
version 1.3c of this license or (at your option) any later
version. This version of this license is in
   https://www.latex-project.org/lppl/lppl-1-3c.txt
and the latest version of this license is in
   https://www.latex-project.org/lppl.txt
and version 1.3 or later is part of all distributions of
LaTeX version 2005/12/01 or later.

This work has the LPPL maintenance status "maintained".

The Current Maintainers of this work are
Heiko Oberdiek and the Oberdiek Package Support Group
https://github.com/ho-tex/oberdiek/issues


This work consists of the main source file classlist.dtx
and the derived files
   classlist.sty, classlist.pdf, classlist.ins, classlist.drv.

\endpreamble
\let\MetaPrefix\DoubleperCent

\generate{%
  \file{classlist.ins}{\from{classlist.dtx}{install}}%
  \file{classlist.drv}{\from{classlist.dtx}{driver}}%
  \usedir{tex/latex/oberdiek}%
  \file{classlist.sty}{\from{classlist.dtx}{package}}%
}

\catcode32=13\relax% active space
\let =\space%
\Msg{************************************************************************}
\Msg{*}
\Msg{* To finish the installation you have to move the following}
\Msg{* file into a directory searched by TeX:}
\Msg{*}
\Msg{*     classlist.sty}
\Msg{*}
\Msg{* To produce the documentation run the file `classlist.drv'}
\Msg{* through LaTeX.}
\Msg{*}
\Msg{* Happy TeXing!}
\Msg{*}
\Msg{************************************************************************}

\endbatchfile
%</install>
%<*ignore>
\fi
%</ignore>
%<*driver>
\NeedsTeXFormat{LaTeX2e}
\ProvidesFile{classlist.drv}%
  [2016/05/16 v1.5 Record classes used in a document (HO)]%
\documentclass{ltxdoc}
\usepackage{holtxdoc}[2011/11/22]
\begin{document}
  \DocInput{classlist.dtx}%
\end{document}
%</driver>
% \fi
%
%
%
% \GetFileInfo{classlist.drv}
%
% \title{The \xpackage{classlist} package}
% \date{2016/05/16 v1.5}
% \author{Heiko Oberdiek\thanks
% {Please report any issues at \url{https://github.com/ho-tex/oberdiek/issues}}}
%
% \maketitle
%
% \begin{abstract}
% This package records the loaded classes and stores
% them in a list.
% \end{abstract}
%
% \tableofcontents
%
% \section{Documentation}
%
% \subsection{Background}
%
% This packages is an answer of a newsgroup question:
% \begin{quote}
% \begin{tabular}{@{}ll@{}}
%   Newsgroup: & comp.text.tex\\
%   Subject: & Finding the Document Class\\
%   From: & Herber Schulz\\
%   Date: & 18 Jun 2005 13:16:49 -0500\\
%   Message-ID: &
%    \textless
%    \texttt{herbs-D55DB9.13170418062005@news.isp.giganews.com}^^A
%    \textgreater
% \end{tabular}
% \end{quote}
%
% \subsection{Usage}
%
% Load this package before \cs{documentclass}:
% \begin{quote}
%   |\RequirePackage{classlist}|\\
%   |\documentclass[some,options]{whatever}|
% \end{quote}
% It then records the classes with options.
%
% If used after \cs{documentclass}, \cs{@filelist} is
% parsed for classes. The additional data
% specified options and requested version is no
% longer available here.
%
% \begin{description}
% \item[\cs{MainClassName}] contains the first loaded class.
% \item[\cs{ClassList}] stores the class entries, eg.
%   \begin{quote}
%   \begin{tabular}{@{}l@{ }l@{}}
%     \cs{ClassList} $\rightarrow$&
%     |\ClassListEntry{myarticle}{a4paper}{}|\\
%     &|\ClassListEntry{article}{}{}|
%   \end{tabular}
%   \end{quote}
% \item[\cs{ClassListEntry}] has three arguments:
%   \begin{quote}
%   \begin{tabular}{@{}ll@{}}
%     |#1|: & class name\\
%     |#2|: & options given in \cs{documentclass}/\cs{LoadClass}\\
%     |#3|: & requested version, not the version of class
%   \end{tabular}
%   \end{quote}
% \item[\cs{PrintClassList}] prints the list on screen it can be
%    configured by
% \item[\cs{PrintClassListTitle}] for the title and
% \item[\cs{PrintClassListEntry}] for formatting the entries.
%    See the implementation for how to use these.
% \end{description}
%
% \StopEventually{
% }
%
% \section{Implementation}
%
%    \begin{macrocode}
%<*package>
%    \end{macrocode}
%    Package identification.
%    \begin{macrocode}
\NeedsTeXFormat{LaTeX2e}
\ProvidesPackage{classlist}%
  [2016/05/16 v1.5 Record classes used in a document (HO)]
%    \end{macrocode}
%
%    \begin{macrocode}
\let\ClassList\@empty
\let\MainClassName\relax
%    \end{macrocode}
%
%    Test, whether we are called before \cs{documentclass}.
%    \begin{macrocode}
\ifx\@classoptionslist\relax
  \let\CL@org@fileswith@pti@ns\@fileswith@pti@ns
  \def\@fileswith@pti@ns#1[#2]#3[#4]{%
%    \end{macrocode}
%    \begin{tabular}{@{}ll@{}}
%      |#1|: & \cs{@clsextension}\\
%      |#2|: & options of \cs{documentclass}/\cs{LoadClass}\\
%      |#3|: & class name\\
%      |#4|: & requested version
%    \end{tabular}
%    \begin{macrocode}
    \ifx#1\@clsextension
      \@ifl@aded#1{#3}{%
        \PackageInfo{classlist}{%
          Skipping class `#3', because\MessageBreak
          this class is already loaded%
        }%
      }{%
        \@ifundefined{MainClassName}{%
          \def\MainClassName{#3}%
        }{}%
        \@temptokena\expandafter{%
          \ClassList
          \ClassListEntry{#3}{#2}{#4}%
        }%
        \edef\ClassList{\the\@temptokena}%
      }%
    \fi
    \CL@org@fileswith@pti@ns{#1}[{#2}]{#3}[{#4}]%
  }%
  \let\@@fileswith@pti@ns\@fileswith@pti@ns
\else
%    \end{macrocode}
%    Called after \cs{documentclass}.
%    \begin{macrocode}
  \PackageInfo{classlist}{Use \string\@filelist\space method}%

  \let\ClassListEntry\relax
  \expandafter\def\expandafter\CL@test
      \expandafter#\expandafter1\@clsextension#2\@nil{%
    \ifx\\#2\\%
%    \end{macrocode}
%    Name does not contain \cs{@clsextension}
%    \begin{macrocode}
    \else
      \expandafter\CL@test@i\CL@entry\@nil
    \fi
  }%
  \expandafter\def\expandafter\CL@test@i
      \expandafter#\expandafter1\@clsextension#2\@nil{%
    \ifx\\#2\\%
      \@ifundefined{opt@\CL@entry}{%
      }{%
        \@ifundefined{MainClassName}{%
          \let\MainClassName\CL@entry
        }{%
        }%
        \edef\ClassList{%
          \ClassList
          \ClassListEntry{\CL@entry}{}{}%
        }%
      }%
    \else
%    \end{macrocode}
%    Names with more than one \cs{@clsextension} are not supported.
%    \begin{macrocode}
    \fi
  }%
  \@for\CL@entry:=\@filelist\do{%
    \expandafter\expandafter\expandafter\CL@test\expandafter
        \CL@entry\@clsextension\@nil
  }%
\fi
%    \end{macrocode}
%
%    \begin{macro}{\PrintClassListEntry}
%    \begin{macrocode}
\providecommand*{\PrintClassListEntry}[3]{%
  \toks@{* #1}%
  \typeout{\the\toks@}%
}
%    \end{macrocode}
%    \end{macro}
%    \begin{macro}{\PrintClassListTitle}
%    \begin{macrocode}
\providecommand*{\PrintClassListTitle}{%
  \typeout{Class list:}%
}
%    \end{macrocode}
%    \end{macro}
%    \begin{macro}{\PrintClassList}
%    \begin{macrocode}
\providecommand*{\PrintClassList}{%
  \begingroup
    \let\ClassListEntry\PrintClassListEntry
    \PrintClassListTitle
    \ClassList
  \endgroup
}
%    \end{macrocode}
%    \end{macro}
%    \begin{macro}{\CL@InfoEntry}
%    \begin{macrocode}
\def\CL@InfoEntry#1#2#3{%
  \advance\count@ by \@ne
  \def\x{#2}%
  \@onelevel@sanitize\x
  \edef\CL@Info{%
    \CL@Info
    \noexpand\MessageBreak
    (\the\count@) %
    #1 [\x]%
    \ifx\\#3\\%
    \else
      \space[#3]% hash-ok
    \fi
  }%
}
%    \end{macrocode}
%    \end{macro}
%    \begin{macrocode}
\AtBeginDocument{%
  \begingroup
    \count@=\z@
    \def\CL@Info{Class List:}%
    \let\ClassListEntry\CL@InfoEntry
    \ClassList
    \let\on@line\@empty
    \PackageInfo{classlist}{\CL@Info}%
  \endgroup
}
%    \end{macrocode}
%
%    \begin{macrocode}
%</package>
%    \end{macrocode}
%
% \section{Installation}
%
% \subsection{Download}
%
% \paragraph{Package.} This package is available on
% CTAN\footnote{\CTANpkg{classlist}}:
% \begin{description}
% \item[\CTAN{macros/latex/contrib/oberdiek/classlist.dtx}] The source file.
% \item[\CTAN{macros/latex/contrib/oberdiek/classlist.pdf}] Documentation.
% \end{description}
%
%
% \paragraph{Bundle.} All the packages of the bundle `oberdiek'
% are also available in a TDS compliant ZIP archive. There
% the packages are already unpacked and the documentation files
% are generated. The files and directories obey the TDS standard.
% \begin{description}
% \item[\CTANinstall{install/macros/latex/contrib/oberdiek.tds.zip}]
% \end{description}
% \emph{TDS} refers to the standard ``A Directory Structure
% for \TeX\ Files'' (\CTANpkg{tds}). Directories
% with \xfile{texmf} in their name are usually organized this way.
%
% \subsection{Bundle installation}
%
% \paragraph{Unpacking.} Unpack the \xfile{oberdiek.tds.zip} in the
% TDS tree (also known as \xfile{texmf} tree) of your choice.
% Example (linux):
% \begin{quote}
%   |unzip oberdiek.tds.zip -d ~/texmf|
% \end{quote}
%
% \subsection{Package installation}
%
% \paragraph{Unpacking.} The \xfile{.dtx} file is a self-extracting
% \docstrip\ archive. The files are extracted by running the
% \xfile{.dtx} through \plainTeX:
% \begin{quote}
%   \verb|tex classlist.dtx|
% \end{quote}
%
% \paragraph{TDS.} Now the different files must be moved into
% the different directories in your installation TDS tree
% (also known as \xfile{texmf} tree):
% \begin{quote}
% \def\t{^^A
% \begin{tabular}{@{}>{\ttfamily}l@{ $\rightarrow$ }>{\ttfamily}l@{}}
%   classlist.sty & tex/latex/oberdiek/classlist.sty\\
%   classlist.pdf & doc/latex/oberdiek/classlist.pdf\\
%   classlist.dtx & source/latex/oberdiek/classlist.dtx\\
% \end{tabular}^^A
% }^^A
% \sbox0{\t}^^A
% \ifdim\wd0>\linewidth
%   \begingroup
%     \advance\linewidth by\leftmargin
%     \advance\linewidth by\rightmargin
%   \edef\x{\endgroup
%     \def\noexpand\lw{\the\linewidth}^^A
%   }\x
%   \def\lwbox{^^A
%     \leavevmode
%     \hbox to \linewidth{^^A
%       \kern-\leftmargin\relax
%       \hss
%       \usebox0
%       \hss
%       \kern-\rightmargin\relax
%     }^^A
%   }^^A
%   \ifdim\wd0>\lw
%     \sbox0{\small\t}^^A
%     \ifdim\wd0>\linewidth
%       \ifdim\wd0>\lw
%         \sbox0{\footnotesize\t}^^A
%         \ifdim\wd0>\linewidth
%           \ifdim\wd0>\lw
%             \sbox0{\scriptsize\t}^^A
%             \ifdim\wd0>\linewidth
%               \ifdim\wd0>\lw
%                 \sbox0{\tiny\t}^^A
%                 \ifdim\wd0>\linewidth
%                   \lwbox
%                 \else
%                   \usebox0
%                 \fi
%               \else
%                 \lwbox
%               \fi
%             \else
%               \usebox0
%             \fi
%           \else
%             \lwbox
%           \fi
%         \else
%           \usebox0
%         \fi
%       \else
%         \lwbox
%       \fi
%     \else
%       \usebox0
%     \fi
%   \else
%     \lwbox
%   \fi
% \else
%   \usebox0
% \fi
% \end{quote}
% If you have a \xfile{docstrip.cfg} that configures and enables \docstrip's
% TDS installing feature, then some files can already be in the right
% place, see the documentation of \docstrip.
%
% \subsection{Refresh file name databases}
%
% If your \TeX~distribution
% (\TeX\,Live, \mikTeX, \dots) relies on file name databases, you must refresh
% these. For example, \TeX\,Live\ users run \verb|texhash| or
% \verb|mktexlsr|.
%
% \subsection{Some details for the interested}
%
% \paragraph{Unpacking with \LaTeX.}
% The \xfile{.dtx} chooses its action depending on the format:
% \begin{description}
% \item[\plainTeX:] Run \docstrip\ and extract the files.
% \item[\LaTeX:] Generate the documentation.
% \end{description}
% If you insist on using \LaTeX\ for \docstrip\ (really,
% \docstrip\ does not need \LaTeX), then inform the autodetect routine
% about your intention:
% \begin{quote}
%   \verb|latex \let\install=y% \iffalse meta-comment
%
% File: classlist.dtx
% Version: 2016/05/16 v1.5
% Info: Record classes used in a document
%
% Copyright (C)
%    2005, 2006, 2008, 2011 Heiko Oberdiek
%    2016-2019 Oberdiek Package Support Group
%    https://github.com/ho-tex/oberdiek/issues
%
% This work may be distributed and/or modified under the
% conditions of the LaTeX Project Public License, either
% version 1.3c of this license or (at your option) any later
% version. This version of this license is in
%    https://www.latex-project.org/lppl/lppl-1-3c.txt
% and the latest version of this license is in
%    https://www.latex-project.org/lppl.txt
% and version 1.3 or later is part of all distributions of
% LaTeX version 2005/12/01 or later.
%
% This work has the LPPL maintenance status "maintained".
%
% The Current Maintainers of this work are
% Heiko Oberdiek and the Oberdiek Package Support Group
% https://github.com/ho-tex/oberdiek/issues
%
% This work consists of the main source file classlist.dtx
% and the derived files
%    classlist.sty, classlist.pdf, classlist.ins, classlist.drv.
%
% Distribution:
%    CTAN:macros/latex/contrib/oberdiek/classlist.dtx
%    CTAN:macros/latex/contrib/oberdiek/classlist.pdf
%
% Unpacking:
%    (a) If classlist.ins is present:
%           tex classlist.ins
%    (b) Without classlist.ins:
%           tex classlist.dtx
%    (c) If you insist on using LaTeX
%           latex \let\install=y\input{classlist.dtx}
%        (quote the arguments according to the demands of your shell)
%
% Documentation:
%    (a) If classlist.drv is present:
%           latex classlist.drv
%    (b) Without classlist.drv:
%           latex classlist.dtx; ...
%    The class ltxdoc loads the configuration file ltxdoc.cfg
%    if available. Here you can specify further options, e.g.
%    use A4 as paper format:
%       \PassOptionsToClass{a4paper}{article}
%
%    Program calls to get the documentation (example):
%       pdflatex classlist.dtx
%       makeindex -s gind.ist classlist.idx
%       pdflatex classlist.dtx
%       makeindex -s gind.ist classlist.idx
%       pdflatex classlist.dtx
%
% Installation:
%    TDS:tex/latex/oberdiek/classlist.sty
%    TDS:doc/latex/oberdiek/classlist.pdf
%    TDS:source/latex/oberdiek/classlist.dtx
%
%<*ignore>
\begingroup
  \catcode123=1 %
  \catcode125=2 %
  \def\x{LaTeX2e}%
\expandafter\endgroup
\ifcase 0\ifx\install y1\fi\expandafter
         \ifx\csname processbatchFile\endcsname\relax\else1\fi
         \ifx\fmtname\x\else 1\fi\relax
\else\csname fi\endcsname
%</ignore>
%<*install>
\input docstrip.tex
\Msg{************************************************************************}
\Msg{* Installation}
\Msg{* Package: classlist 2016/05/16 v1.5 Record classes used in a document (HO)}
\Msg{************************************************************************}

\keepsilent
\askforoverwritefalse

\let\MetaPrefix\relax
\preamble

This is a generated file.

Project: classlist
Version: 2016/05/16 v1.5

Copyright (C)
   2005, 2006, 2008, 2011 Heiko Oberdiek
   2016-2019 Oberdiek Package Support Group

This work may be distributed and/or modified under the
conditions of the LaTeX Project Public License, either
version 1.3c of this license or (at your option) any later
version. This version of this license is in
   https://www.latex-project.org/lppl/lppl-1-3c.txt
and the latest version of this license is in
   https://www.latex-project.org/lppl.txt
and version 1.3 or later is part of all distributions of
LaTeX version 2005/12/01 or later.

This work has the LPPL maintenance status "maintained".

The Current Maintainers of this work are
Heiko Oberdiek and the Oberdiek Package Support Group
https://github.com/ho-tex/oberdiek/issues


This work consists of the main source file classlist.dtx
and the derived files
   classlist.sty, classlist.pdf, classlist.ins, classlist.drv.

\endpreamble
\let\MetaPrefix\DoubleperCent

\generate{%
  \file{classlist.ins}{\from{classlist.dtx}{install}}%
  \file{classlist.drv}{\from{classlist.dtx}{driver}}%
  \usedir{tex/latex/oberdiek}%
  \file{classlist.sty}{\from{classlist.dtx}{package}}%
}

\catcode32=13\relax% active space
\let =\space%
\Msg{************************************************************************}
\Msg{*}
\Msg{* To finish the installation you have to move the following}
\Msg{* file into a directory searched by TeX:}
\Msg{*}
\Msg{*     classlist.sty}
\Msg{*}
\Msg{* To produce the documentation run the file `classlist.drv'}
\Msg{* through LaTeX.}
\Msg{*}
\Msg{* Happy TeXing!}
\Msg{*}
\Msg{************************************************************************}

\endbatchfile
%</install>
%<*ignore>
\fi
%</ignore>
%<*driver>
\NeedsTeXFormat{LaTeX2e}
\ProvidesFile{classlist.drv}%
  [2016/05/16 v1.5 Record classes used in a document (HO)]%
\documentclass{ltxdoc}
\usepackage{holtxdoc}[2011/11/22]
\begin{document}
  \DocInput{classlist.dtx}%
\end{document}
%</driver>
% \fi
%
%
%
% \GetFileInfo{classlist.drv}
%
% \title{The \xpackage{classlist} package}
% \date{2016/05/16 v1.5}
% \author{Heiko Oberdiek\thanks
% {Please report any issues at \url{https://github.com/ho-tex/oberdiek/issues}}}
%
% \maketitle
%
% \begin{abstract}
% This package records the loaded classes and stores
% them in a list.
% \end{abstract}
%
% \tableofcontents
%
% \section{Documentation}
%
% \subsection{Background}
%
% This packages is an answer of a newsgroup question:
% \begin{quote}
% \begin{tabular}{@{}ll@{}}
%   Newsgroup: & comp.text.tex\\
%   Subject: & Finding the Document Class\\
%   From: & Herber Schulz\\
%   Date: & 18 Jun 2005 13:16:49 -0500\\
%   Message-ID: &
%    \textless
%    \texttt{herbs-D55DB9.13170418062005@news.isp.giganews.com}^^A
%    \textgreater
% \end{tabular}
% \end{quote}
%
% \subsection{Usage}
%
% Load this package before \cs{documentclass}:
% \begin{quote}
%   |\RequirePackage{classlist}|\\
%   |\documentclass[some,options]{whatever}|
% \end{quote}
% It then records the classes with options.
%
% If used after \cs{documentclass}, \cs{@filelist} is
% parsed for classes. The additional data
% specified options and requested version is no
% longer available here.
%
% \begin{description}
% \item[\cs{MainClassName}] contains the first loaded class.
% \item[\cs{ClassList}] stores the class entries, eg.
%   \begin{quote}
%   \begin{tabular}{@{}l@{ }l@{}}
%     \cs{ClassList} $\rightarrow$&
%     |\ClassListEntry{myarticle}{a4paper}{}|\\
%     &|\ClassListEntry{article}{}{}|
%   \end{tabular}
%   \end{quote}
% \item[\cs{ClassListEntry}] has three arguments:
%   \begin{quote}
%   \begin{tabular}{@{}ll@{}}
%     |#1|: & class name\\
%     |#2|: & options given in \cs{documentclass}/\cs{LoadClass}\\
%     |#3|: & requested version, not the version of class
%   \end{tabular}
%   \end{quote}
% \item[\cs{PrintClassList}] prints the list on screen it can be
%    configured by
% \item[\cs{PrintClassListTitle}] for the title and
% \item[\cs{PrintClassListEntry}] for formatting the entries.
%    See the implementation for how to use these.
% \end{description}
%
% \StopEventually{
% }
%
% \section{Implementation}
%
%    \begin{macrocode}
%<*package>
%    \end{macrocode}
%    Package identification.
%    \begin{macrocode}
\NeedsTeXFormat{LaTeX2e}
\ProvidesPackage{classlist}%
  [2016/05/16 v1.5 Record classes used in a document (HO)]
%    \end{macrocode}
%
%    \begin{macrocode}
\let\ClassList\@empty
\let\MainClassName\relax
%    \end{macrocode}
%
%    Test, whether we are called before \cs{documentclass}.
%    \begin{macrocode}
\ifx\@classoptionslist\relax
  \let\CL@org@fileswith@pti@ns\@fileswith@pti@ns
  \def\@fileswith@pti@ns#1[#2]#3[#4]{%
%    \end{macrocode}
%    \begin{tabular}{@{}ll@{}}
%      |#1|: & \cs{@clsextension}\\
%      |#2|: & options of \cs{documentclass}/\cs{LoadClass}\\
%      |#3|: & class name\\
%      |#4|: & requested version
%    \end{tabular}
%    \begin{macrocode}
    \ifx#1\@clsextension
      \@ifl@aded#1{#3}{%
        \PackageInfo{classlist}{%
          Skipping class `#3', because\MessageBreak
          this class is already loaded%
        }%
      }{%
        \@ifundefined{MainClassName}{%
          \def\MainClassName{#3}%
        }{}%
        \@temptokena\expandafter{%
          \ClassList
          \ClassListEntry{#3}{#2}{#4}%
        }%
        \edef\ClassList{\the\@temptokena}%
      }%
    \fi
    \CL@org@fileswith@pti@ns{#1}[{#2}]{#3}[{#4}]%
  }%
  \let\@@fileswith@pti@ns\@fileswith@pti@ns
\else
%    \end{macrocode}
%    Called after \cs{documentclass}.
%    \begin{macrocode}
  \PackageInfo{classlist}{Use \string\@filelist\space method}%

  \let\ClassListEntry\relax
  \expandafter\def\expandafter\CL@test
      \expandafter#\expandafter1\@clsextension#2\@nil{%
    \ifx\\#2\\%
%    \end{macrocode}
%    Name does not contain \cs{@clsextension}
%    \begin{macrocode}
    \else
      \expandafter\CL@test@i\CL@entry\@nil
    \fi
  }%
  \expandafter\def\expandafter\CL@test@i
      \expandafter#\expandafter1\@clsextension#2\@nil{%
    \ifx\\#2\\%
      \@ifundefined{opt@\CL@entry}{%
      }{%
        \@ifundefined{MainClassName}{%
          \let\MainClassName\CL@entry
        }{%
        }%
        \edef\ClassList{%
          \ClassList
          \ClassListEntry{\CL@entry}{}{}%
        }%
      }%
    \else
%    \end{macrocode}
%    Names with more than one \cs{@clsextension} are not supported.
%    \begin{macrocode}
    \fi
  }%
  \@for\CL@entry:=\@filelist\do{%
    \expandafter\expandafter\expandafter\CL@test\expandafter
        \CL@entry\@clsextension\@nil
  }%
\fi
%    \end{macrocode}
%
%    \begin{macro}{\PrintClassListEntry}
%    \begin{macrocode}
\providecommand*{\PrintClassListEntry}[3]{%
  \toks@{* #1}%
  \typeout{\the\toks@}%
}
%    \end{macrocode}
%    \end{macro}
%    \begin{macro}{\PrintClassListTitle}
%    \begin{macrocode}
\providecommand*{\PrintClassListTitle}{%
  \typeout{Class list:}%
}
%    \end{macrocode}
%    \end{macro}
%    \begin{macro}{\PrintClassList}
%    \begin{macrocode}
\providecommand*{\PrintClassList}{%
  \begingroup
    \let\ClassListEntry\PrintClassListEntry
    \PrintClassListTitle
    \ClassList
  \endgroup
}
%    \end{macrocode}
%    \end{macro}
%    \begin{macro}{\CL@InfoEntry}
%    \begin{macrocode}
\def\CL@InfoEntry#1#2#3{%
  \advance\count@ by \@ne
  \def\x{#2}%
  \@onelevel@sanitize\x
  \edef\CL@Info{%
    \CL@Info
    \noexpand\MessageBreak
    (\the\count@) %
    #1 [\x]%
    \ifx\\#3\\%
    \else
      \space[#3]% hash-ok
    \fi
  }%
}
%    \end{macrocode}
%    \end{macro}
%    \begin{macrocode}
\AtBeginDocument{%
  \begingroup
    \count@=\z@
    \def\CL@Info{Class List:}%
    \let\ClassListEntry\CL@InfoEntry
    \ClassList
    \let\on@line\@empty
    \PackageInfo{classlist}{\CL@Info}%
  \endgroup
}
%    \end{macrocode}
%
%    \begin{macrocode}
%</package>
%    \end{macrocode}
%
% \section{Installation}
%
% \subsection{Download}
%
% \paragraph{Package.} This package is available on
% CTAN\footnote{\CTANpkg{classlist}}:
% \begin{description}
% \item[\CTAN{macros/latex/contrib/oberdiek/classlist.dtx}] The source file.
% \item[\CTAN{macros/latex/contrib/oberdiek/classlist.pdf}] Documentation.
% \end{description}
%
%
% \paragraph{Bundle.} All the packages of the bundle `oberdiek'
% are also available in a TDS compliant ZIP archive. There
% the packages are already unpacked and the documentation files
% are generated. The files and directories obey the TDS standard.
% \begin{description}
% \item[\CTANinstall{install/macros/latex/contrib/oberdiek.tds.zip}]
% \end{description}
% \emph{TDS} refers to the standard ``A Directory Structure
% for \TeX\ Files'' (\CTANpkg{tds}). Directories
% with \xfile{texmf} in their name are usually organized this way.
%
% \subsection{Bundle installation}
%
% \paragraph{Unpacking.} Unpack the \xfile{oberdiek.tds.zip} in the
% TDS tree (also known as \xfile{texmf} tree) of your choice.
% Example (linux):
% \begin{quote}
%   |unzip oberdiek.tds.zip -d ~/texmf|
% \end{quote}
%
% \subsection{Package installation}
%
% \paragraph{Unpacking.} The \xfile{.dtx} file is a self-extracting
% \docstrip\ archive. The files are extracted by running the
% \xfile{.dtx} through \plainTeX:
% \begin{quote}
%   \verb|tex classlist.dtx|
% \end{quote}
%
% \paragraph{TDS.} Now the different files must be moved into
% the different directories in your installation TDS tree
% (also known as \xfile{texmf} tree):
% \begin{quote}
% \def\t{^^A
% \begin{tabular}{@{}>{\ttfamily}l@{ $\rightarrow$ }>{\ttfamily}l@{}}
%   classlist.sty & tex/latex/oberdiek/classlist.sty\\
%   classlist.pdf & doc/latex/oberdiek/classlist.pdf\\
%   classlist.dtx & source/latex/oberdiek/classlist.dtx\\
% \end{tabular}^^A
% }^^A
% \sbox0{\t}^^A
% \ifdim\wd0>\linewidth
%   \begingroup
%     \advance\linewidth by\leftmargin
%     \advance\linewidth by\rightmargin
%   \edef\x{\endgroup
%     \def\noexpand\lw{\the\linewidth}^^A
%   }\x
%   \def\lwbox{^^A
%     \leavevmode
%     \hbox to \linewidth{^^A
%       \kern-\leftmargin\relax
%       \hss
%       \usebox0
%       \hss
%       \kern-\rightmargin\relax
%     }^^A
%   }^^A
%   \ifdim\wd0>\lw
%     \sbox0{\small\t}^^A
%     \ifdim\wd0>\linewidth
%       \ifdim\wd0>\lw
%         \sbox0{\footnotesize\t}^^A
%         \ifdim\wd0>\linewidth
%           \ifdim\wd0>\lw
%             \sbox0{\scriptsize\t}^^A
%             \ifdim\wd0>\linewidth
%               \ifdim\wd0>\lw
%                 \sbox0{\tiny\t}^^A
%                 \ifdim\wd0>\linewidth
%                   \lwbox
%                 \else
%                   \usebox0
%                 \fi
%               \else
%                 \lwbox
%               \fi
%             \else
%               \usebox0
%             \fi
%           \else
%             \lwbox
%           \fi
%         \else
%           \usebox0
%         \fi
%       \else
%         \lwbox
%       \fi
%     \else
%       \usebox0
%     \fi
%   \else
%     \lwbox
%   \fi
% \else
%   \usebox0
% \fi
% \end{quote}
% If you have a \xfile{docstrip.cfg} that configures and enables \docstrip's
% TDS installing feature, then some files can already be in the right
% place, see the documentation of \docstrip.
%
% \subsection{Refresh file name databases}
%
% If your \TeX~distribution
% (\TeX\,Live, \mikTeX, \dots) relies on file name databases, you must refresh
% these. For example, \TeX\,Live\ users run \verb|texhash| or
% \verb|mktexlsr|.
%
% \subsection{Some details for the interested}
%
% \paragraph{Unpacking with \LaTeX.}
% The \xfile{.dtx} chooses its action depending on the format:
% \begin{description}
% \item[\plainTeX:] Run \docstrip\ and extract the files.
% \item[\LaTeX:] Generate the documentation.
% \end{description}
% If you insist on using \LaTeX\ for \docstrip\ (really,
% \docstrip\ does not need \LaTeX), then inform the autodetect routine
% about your intention:
% \begin{quote}
%   \verb|latex \let\install=y\input{classlist.dtx}|
% \end{quote}
% Do not forget to quote the argument according to the demands
% of your shell.
%
% \paragraph{Generating the documentation.}
% You can use both the \xfile{.dtx} or the \xfile{.drv} to generate
% the documentation. The process can be configured by the
% configuration file \xfile{ltxdoc.cfg}. For instance, put this
% line into this file, if you want to have A4 as paper format:
% \begin{quote}
%   \verb|\PassOptionsToClass{a4paper}{article}|
% \end{quote}
% An example follows how to generate the
% documentation with pdf\LaTeX:
% \begin{quote}
%\begin{verbatim}
%pdflatex classlist.dtx
%makeindex -s gind.ist classlist.idx
%pdflatex classlist.dtx
%makeindex -s gind.ist classlist.idx
%pdflatex classlist.dtx
%\end{verbatim}
% \end{quote}
%
% \begin{History}
%   \begin{Version}{2005/06/19 v1.0}
%   \item
%     First published version: CTAN and newsgroup \xnewsgroup{comp.text.tex}:
%     \URL{``\link{Re: Finding the Document Class}''}^^A
%     {https://groups.google.com/group/comp.text.tex/msg/8ee9523c2dc13666}
%   \end{Version}
%   \begin{Version}{2005/06/19 v1.1}
%   \item
%     After \cs{documentclass} the package looks
%     at \cs{@filelist} instead of aborting with error.
%   \end{Version}
%   \begin{Version}{2006/02/20 v1.2}
%   \item
%     DTX framework.
%   \item
%     Fix for \cs{@@fileswith@pti@ns}.
%   \end{Version}
%   \begin{Version}{2008/08/11 v1.3}
%   \item
%     Code is not changed.
%   \item
%     URLs updated.
%   \end{Version}
%   \begin{Version}{2011/10/17 v1.4}
%   \item
%     Documentation fix: \cs{MainClass} $\rightarrow$ \cs{MainClassName}.
%   \end{Version}
%   \begin{Version}{2016/05/16 v1.5}
%   \item
%     Documentation updates.
%   \end{Version}
% \end{History}
%
% \PrintIndex
%
% \Finale
\endinput
|
% \end{quote}
% Do not forget to quote the argument according to the demands
% of your shell.
%
% \paragraph{Generating the documentation.}
% You can use both the \xfile{.dtx} or the \xfile{.drv} to generate
% the documentation. The process can be configured by the
% configuration file \xfile{ltxdoc.cfg}. For instance, put this
% line into this file, if you want to have A4 as paper format:
% \begin{quote}
%   \verb|\PassOptionsToClass{a4paper}{article}|
% \end{quote}
% An example follows how to generate the
% documentation with pdf\LaTeX:
% \begin{quote}
%\begin{verbatim}
%pdflatex classlist.dtx
%makeindex -s gind.ist classlist.idx
%pdflatex classlist.dtx
%makeindex -s gind.ist classlist.idx
%pdflatex classlist.dtx
%\end{verbatim}
% \end{quote}
%
% \begin{History}
%   \begin{Version}{2005/06/19 v1.0}
%   \item
%     First published version: CTAN and newsgroup \xnewsgroup{comp.text.tex}:
%     \URL{``\link{Re: Finding the Document Class}''}^^A
%     {https://groups.google.com/group/comp.text.tex/msg/8ee9523c2dc13666}
%   \end{Version}
%   \begin{Version}{2005/06/19 v1.1}
%   \item
%     After \cs{documentclass} the package looks
%     at \cs{@filelist} instead of aborting with error.
%   \end{Version}
%   \begin{Version}{2006/02/20 v1.2}
%   \item
%     DTX framework.
%   \item
%     Fix for \cs{@@fileswith@pti@ns}.
%   \end{Version}
%   \begin{Version}{2008/08/11 v1.3}
%   \item
%     Code is not changed.
%   \item
%     URLs updated.
%   \end{Version}
%   \begin{Version}{2011/10/17 v1.4}
%   \item
%     Documentation fix: \cs{MainClass} $\rightarrow$ \cs{MainClassName}.
%   \end{Version}
%   \begin{Version}{2016/05/16 v1.5}
%   \item
%     Documentation updates.
%   \end{Version}
% \end{History}
%
% \PrintIndex
%
% \Finale
\endinput
|
% \end{quote}
% Do not forget to quote the argument according to the demands
% of your shell.
%
% \paragraph{Generating the documentation.}
% You can use both the \xfile{.dtx} or the \xfile{.drv} to generate
% the documentation. The process can be configured by the
% configuration file \xfile{ltxdoc.cfg}. For instance, put this
% line into this file, if you want to have A4 as paper format:
% \begin{quote}
%   \verb|\PassOptionsToClass{a4paper}{article}|
% \end{quote}
% An example follows how to generate the
% documentation with pdf\LaTeX:
% \begin{quote}
%\begin{verbatim}
%pdflatex classlist.dtx
%makeindex -s gind.ist classlist.idx
%pdflatex classlist.dtx
%makeindex -s gind.ist classlist.idx
%pdflatex classlist.dtx
%\end{verbatim}
% \end{quote}
%
% \begin{History}
%   \begin{Version}{2005/06/19 v1.0}
%   \item
%     First published version: CTAN and newsgroup \xnewsgroup{comp.text.tex}:
%     \URL{``\link{Re: Finding the Document Class}''}^^A
%     {https://groups.google.com/group/comp.text.tex/msg/8ee9523c2dc13666}
%   \end{Version}
%   \begin{Version}{2005/06/19 v1.1}
%   \item
%     After \cs{documentclass} the package looks
%     at \cs{@filelist} instead of aborting with error.
%   \end{Version}
%   \begin{Version}{2006/02/20 v1.2}
%   \item
%     DTX framework.
%   \item
%     Fix for \cs{@@fileswith@pti@ns}.
%   \end{Version}
%   \begin{Version}{2008/08/11 v1.3}
%   \item
%     Code is not changed.
%   \item
%     URLs updated.
%   \end{Version}
%   \begin{Version}{2011/10/17 v1.4}
%   \item
%     Documentation fix: \cs{MainClass} $\rightarrow$ \cs{MainClassName}.
%   \end{Version}
%   \begin{Version}{2016/05/16 v1.5}
%   \item
%     Documentation updates.
%   \end{Version}
% \end{History}
%
% \PrintIndex
%
% \Finale
\endinput

%        (quote the arguments according to the demands of your shell)
%
% Documentation:
%    (a) If classlist.drv is present:
%           latex classlist.drv
%    (b) Without classlist.drv:
%           latex classlist.dtx; ...
%    The class ltxdoc loads the configuration file ltxdoc.cfg
%    if available. Here you can specify further options, e.g.
%    use A4 as paper format:
%       \PassOptionsToClass{a4paper}{article}
%
%    Program calls to get the documentation (example):
%       pdflatex classlist.dtx
%       makeindex -s gind.ist classlist.idx
%       pdflatex classlist.dtx
%       makeindex -s gind.ist classlist.idx
%       pdflatex classlist.dtx
%
% Installation:
%    TDS:tex/latex/oberdiek/classlist.sty
%    TDS:doc/latex/oberdiek/classlist.pdf
%    TDS:source/latex/oberdiek/classlist.dtx
%
%<*ignore>
\begingroup
  \catcode123=1 %
  \catcode125=2 %
  \def\x{LaTeX2e}%
\expandafter\endgroup
\ifcase 0\ifx\install y1\fi\expandafter
         \ifx\csname processbatchFile\endcsname\relax\else1\fi
         \ifx\fmtname\x\else 1\fi\relax
\else\csname fi\endcsname
%</ignore>
%<*install>
\input docstrip.tex
\Msg{************************************************************************}
\Msg{* Installation}
\Msg{* Package: classlist 2016/05/16 v1.5 Record classes used in a document (HO)}
\Msg{************************************************************************}

\keepsilent
\askforoverwritefalse

\let\MetaPrefix\relax
\preamble

This is a generated file.

Project: classlist
Version: 2016/05/16 v1.5

Copyright (C)
   2005, 2006, 2008, 2011 Heiko Oberdiek
   2016-2019 Oberdiek Package Support Group

This work may be distributed and/or modified under the
conditions of the LaTeX Project Public License, either
version 1.3c of this license or (at your option) any later
version. This version of this license is in
   https://www.latex-project.org/lppl/lppl-1-3c.txt
and the latest version of this license is in
   https://www.latex-project.org/lppl.txt
and version 1.3 or later is part of all distributions of
LaTeX version 2005/12/01 or later.

This work has the LPPL maintenance status "maintained".

The Current Maintainers of this work are
Heiko Oberdiek and the Oberdiek Package Support Group
https://github.com/ho-tex/oberdiek/issues


This work consists of the main source file classlist.dtx
and the derived files
   classlist.sty, classlist.pdf, classlist.ins, classlist.drv.

\endpreamble
\let\MetaPrefix\DoubleperCent

\generate{%
  \file{classlist.ins}{\from{classlist.dtx}{install}}%
  \file{classlist.drv}{\from{classlist.dtx}{driver}}%
  \usedir{tex/latex/oberdiek}%
  \file{classlist.sty}{\from{classlist.dtx}{package}}%
}

\catcode32=13\relax% active space
\let =\space%
\Msg{************************************************************************}
\Msg{*}
\Msg{* To finish the installation you have to move the following}
\Msg{* file into a directory searched by TeX:}
\Msg{*}
\Msg{*     classlist.sty}
\Msg{*}
\Msg{* To produce the documentation run the file `classlist.drv'}
\Msg{* through LaTeX.}
\Msg{*}
\Msg{* Happy TeXing!}
\Msg{*}
\Msg{************************************************************************}

\endbatchfile
%</install>
%<*ignore>
\fi
%</ignore>
%<*driver>
\NeedsTeXFormat{LaTeX2e}
\ProvidesFile{classlist.drv}%
  [2016/05/16 v1.5 Record classes used in a document (HO)]%
\documentclass{ltxdoc}
\usepackage{holtxdoc}[2011/11/22]
\begin{document}
  \DocInput{classlist.dtx}%
\end{document}
%</driver>
% \fi
%
%
%
% \GetFileInfo{classlist.drv}
%
% \title{The \xpackage{classlist} package}
% \date{2016/05/16 v1.5}
% \author{Heiko Oberdiek\thanks
% {Please report any issues at \url{https://github.com/ho-tex/oberdiek/issues}}}
%
% \maketitle
%
% \begin{abstract}
% This package records the loaded classes and stores
% them in a list.
% \end{abstract}
%
% \tableofcontents
%
% \section{Documentation}
%
% \subsection{Background}
%
% This packages is an answer of a newsgroup question:
% \begin{quote}
% \begin{tabular}{@{}ll@{}}
%   Newsgroup: & comp.text.tex\\
%   Subject: & Finding the Document Class\\
%   From: & Herber Schulz\\
%   Date: & 18 Jun 2005 13:16:49 -0500\\
%   Message-ID: &
%    \textless
%    \texttt{herbs-D55DB9.13170418062005@news.isp.giganews.com}^^A
%    \textgreater
% \end{tabular}
% \end{quote}
%
% \subsection{Usage}
%
% Load this package before \cs{documentclass}:
% \begin{quote}
%   |\RequirePackage{classlist}|\\
%   |\documentclass[some,options]{whatever}|
% \end{quote}
% It then records the classes with options.
%
% If used after \cs{documentclass}, \cs{@filelist} is
% parsed for classes. The additional data
% specified options and requested version is no
% longer available here.
%
% \begin{description}
% \item[\cs{MainClassName}] contains the first loaded class.
% \item[\cs{ClassList}] stores the class entries, eg.
%   \begin{quote}
%   \begin{tabular}{@{}l@{ }l@{}}
%     \cs{ClassList} $\rightarrow$&
%     |\ClassListEntry{myarticle}{a4paper}{}|\\
%     &|\ClassListEntry{article}{}{}|
%   \end{tabular}
%   \end{quote}
% \item[\cs{ClassListEntry}] has three arguments:
%   \begin{quote}
%   \begin{tabular}{@{}ll@{}}
%     |#1|: & class name\\
%     |#2|: & options given in \cs{documentclass}/\cs{LoadClass}\\
%     |#3|: & requested version, not the version of class
%   \end{tabular}
%   \end{quote}
% \item[\cs{PrintClassList}] prints the list on screen it can be
%    configured by
% \item[\cs{PrintClassListTitle}] for the title and
% \item[\cs{PrintClassListEntry}] for formatting the entries.
%    See the implementation for how to use these.
% \end{description}
%
% \StopEventually{
% }
%
% \section{Implementation}
%
%    \begin{macrocode}
%<*package>
%    \end{macrocode}
%    Package identification.
%    \begin{macrocode}
\NeedsTeXFormat{LaTeX2e}
\ProvidesPackage{classlist}%
  [2016/05/16 v1.5 Record classes used in a document (HO)]
%    \end{macrocode}
%
%    \begin{macrocode}
\let\ClassList\@empty
\let\MainClassName\relax
%    \end{macrocode}
%
%    Test, whether we are called before \cs{documentclass}.
%    \begin{macrocode}
\ifx\@classoptionslist\relax
  \let\CL@org@fileswith@pti@ns\@fileswith@pti@ns
  \def\@fileswith@pti@ns#1[#2]#3[#4]{%
%    \end{macrocode}
%    \begin{tabular}{@{}ll@{}}
%      |#1|: & \cs{@clsextension}\\
%      |#2|: & options of \cs{documentclass}/\cs{LoadClass}\\
%      |#3|: & class name\\
%      |#4|: & requested version
%    \end{tabular}
%    \begin{macrocode}
    \ifx#1\@clsextension
      \@ifl@aded#1{#3}{%
        \PackageInfo{classlist}{%
          Skipping class `#3', because\MessageBreak
          this class is already loaded%
        }%
      }{%
        \@ifundefined{MainClassName}{%
          \def\MainClassName{#3}%
        }{}%
        \@temptokena\expandafter{%
          \ClassList
          \ClassListEntry{#3}{#2}{#4}%
        }%
        \edef\ClassList{\the\@temptokena}%
      }%
    \fi
    \CL@org@fileswith@pti@ns{#1}[{#2}]{#3}[{#4}]%
  }%
  \let\@@fileswith@pti@ns\@fileswith@pti@ns
\else
%    \end{macrocode}
%    Called after \cs{documentclass}.
%    \begin{macrocode}
  \PackageInfo{classlist}{Use \string\@filelist\space method}%

  \let\ClassListEntry\relax
  \expandafter\def\expandafter\CL@test
      \expandafter#\expandafter1\@clsextension#2\@nil{%
    \ifx\\#2\\%
%    \end{macrocode}
%    Name does not contain \cs{@clsextension}
%    \begin{macrocode}
    \else
      \expandafter\CL@test@i\CL@entry\@nil
    \fi
  }%
  \expandafter\def\expandafter\CL@test@i
      \expandafter#\expandafter1\@clsextension#2\@nil{%
    \ifx\\#2\\%
      \@ifundefined{opt@\CL@entry}{%
      }{%
        \@ifundefined{MainClassName}{%
          \let\MainClassName\CL@entry
        }{%
        }%
        \edef\ClassList{%
          \ClassList
          \ClassListEntry{\CL@entry}{}{}%
        }%
      }%
    \else
%    \end{macrocode}
%    Names with more than one \cs{@clsextension} are not supported.
%    \begin{macrocode}
    \fi
  }%
  \@for\CL@entry:=\@filelist\do{%
    \expandafter\expandafter\expandafter\CL@test\expandafter
        \CL@entry\@clsextension\@nil
  }%
\fi
%    \end{macrocode}
%
%    \begin{macro}{\PrintClassListEntry}
%    \begin{macrocode}
\providecommand*{\PrintClassListEntry}[3]{%
  \toks@{* #1}%
  \typeout{\the\toks@}%
}
%    \end{macrocode}
%    \end{macro}
%    \begin{macro}{\PrintClassListTitle}
%    \begin{macrocode}
\providecommand*{\PrintClassListTitle}{%
  \typeout{Class list:}%
}
%    \end{macrocode}
%    \end{macro}
%    \begin{macro}{\PrintClassList}
%    \begin{macrocode}
\providecommand*{\PrintClassList}{%
  \begingroup
    \let\ClassListEntry\PrintClassListEntry
    \PrintClassListTitle
    \ClassList
  \endgroup
}
%    \end{macrocode}
%    \end{macro}
%    \begin{macro}{\CL@InfoEntry}
%    \begin{macrocode}
\def\CL@InfoEntry#1#2#3{%
  \advance\count@ by \@ne
  \def\x{#2}%
  \@onelevel@sanitize\x
  \edef\CL@Info{%
    \CL@Info
    \noexpand\MessageBreak
    (\the\count@) %
    #1 [\x]%
    \ifx\\#3\\%
    \else
      \space[#3]% hash-ok
    \fi
  }%
}
%    \end{macrocode}
%    \end{macro}
%    \begin{macrocode}
\AtBeginDocument{%
  \begingroup
    \count@=\z@
    \def\CL@Info{Class List:}%
    \let\ClassListEntry\CL@InfoEntry
    \ClassList
    \let\on@line\@empty
    \PackageInfo{classlist}{\CL@Info}%
  \endgroup
}
%    \end{macrocode}
%
%    \begin{macrocode}
%</package>
%    \end{macrocode}
%
% \section{Installation}
%
% \subsection{Download}
%
% \paragraph{Package.} This package is available on
% CTAN\footnote{\CTANpkg{classlist}}:
% \begin{description}
% \item[\CTAN{macros/latex/contrib/oberdiek/classlist.dtx}] The source file.
% \item[\CTAN{macros/latex/contrib/oberdiek/classlist.pdf}] Documentation.
% \end{description}
%
%
% \paragraph{Bundle.} All the packages of the bundle `oberdiek'
% are also available in a TDS compliant ZIP archive. There
% the packages are already unpacked and the documentation files
% are generated. The files and directories obey the TDS standard.
% \begin{description}
% \item[\CTANinstall{install/macros/latex/contrib/oberdiek.tds.zip}]
% \end{description}
% \emph{TDS} refers to the standard ``A Directory Structure
% for \TeX\ Files'' (\CTANpkg{tds}). Directories
% with \xfile{texmf} in their name are usually organized this way.
%
% \subsection{Bundle installation}
%
% \paragraph{Unpacking.} Unpack the \xfile{oberdiek.tds.zip} in the
% TDS tree (also known as \xfile{texmf} tree) of your choice.
% Example (linux):
% \begin{quote}
%   |unzip oberdiek.tds.zip -d ~/texmf|
% \end{quote}
%
% \subsection{Package installation}
%
% \paragraph{Unpacking.} The \xfile{.dtx} file is a self-extracting
% \docstrip\ archive. The files are extracted by running the
% \xfile{.dtx} through \plainTeX:
% \begin{quote}
%   \verb|tex classlist.dtx|
% \end{quote}
%
% \paragraph{TDS.} Now the different files must be moved into
% the different directories in your installation TDS tree
% (also known as \xfile{texmf} tree):
% \begin{quote}
% \def\t{^^A
% \begin{tabular}{@{}>{\ttfamily}l@{ $\rightarrow$ }>{\ttfamily}l@{}}
%   classlist.sty & tex/latex/oberdiek/classlist.sty\\
%   classlist.pdf & doc/latex/oberdiek/classlist.pdf\\
%   classlist.dtx & source/latex/oberdiek/classlist.dtx\\
% \end{tabular}^^A
% }^^A
% \sbox0{\t}^^A
% \ifdim\wd0>\linewidth
%   \begingroup
%     \advance\linewidth by\leftmargin
%     \advance\linewidth by\rightmargin
%   \edef\x{\endgroup
%     \def\noexpand\lw{\the\linewidth}^^A
%   }\x
%   \def\lwbox{^^A
%     \leavevmode
%     \hbox to \linewidth{^^A
%       \kern-\leftmargin\relax
%       \hss
%       \usebox0
%       \hss
%       \kern-\rightmargin\relax
%     }^^A
%   }^^A
%   \ifdim\wd0>\lw
%     \sbox0{\small\t}^^A
%     \ifdim\wd0>\linewidth
%       \ifdim\wd0>\lw
%         \sbox0{\footnotesize\t}^^A
%         \ifdim\wd0>\linewidth
%           \ifdim\wd0>\lw
%             \sbox0{\scriptsize\t}^^A
%             \ifdim\wd0>\linewidth
%               \ifdim\wd0>\lw
%                 \sbox0{\tiny\t}^^A
%                 \ifdim\wd0>\linewidth
%                   \lwbox
%                 \else
%                   \usebox0
%                 \fi
%               \else
%                 \lwbox
%               \fi
%             \else
%               \usebox0
%             \fi
%           \else
%             \lwbox
%           \fi
%         \else
%           \usebox0
%         \fi
%       \else
%         \lwbox
%       \fi
%     \else
%       \usebox0
%     \fi
%   \else
%     \lwbox
%   \fi
% \else
%   \usebox0
% \fi
% \end{quote}
% If you have a \xfile{docstrip.cfg} that configures and enables \docstrip's
% TDS installing feature, then some files can already be in the right
% place, see the documentation of \docstrip.
%
% \subsection{Refresh file name databases}
%
% If your \TeX~distribution
% (\TeX\,Live, \mikTeX, \dots) relies on file name databases, you must refresh
% these. For example, \TeX\,Live\ users run \verb|texhash| or
% \verb|mktexlsr|.
%
% \subsection{Some details for the interested}
%
% \paragraph{Unpacking with \LaTeX.}
% The \xfile{.dtx} chooses its action depending on the format:
% \begin{description}
% \item[\plainTeX:] Run \docstrip\ and extract the files.
% \item[\LaTeX:] Generate the documentation.
% \end{description}
% If you insist on using \LaTeX\ for \docstrip\ (really,
% \docstrip\ does not need \LaTeX), then inform the autodetect routine
% about your intention:
% \begin{quote}
%   \verb|latex \let\install=y% \iffalse meta-comment
%
% File: classlist.dtx
% Version: 2016/05/16 v1.5
% Info: Record classes used in a document
%
% Copyright (C)
%    2005, 2006, 2008, 2011 Heiko Oberdiek
%    2016-2019 Oberdiek Package Support Group
%    https://github.com/ho-tex/oberdiek/issues
%
% This work may be distributed and/or modified under the
% conditions of the LaTeX Project Public License, either
% version 1.3c of this license or (at your option) any later
% version. This version of this license is in
%    https://www.latex-project.org/lppl/lppl-1-3c.txt
% and the latest version of this license is in
%    https://www.latex-project.org/lppl.txt
% and version 1.3 or later is part of all distributions of
% LaTeX version 2005/12/01 or later.
%
% This work has the LPPL maintenance status "maintained".
%
% The Current Maintainers of this work are
% Heiko Oberdiek and the Oberdiek Package Support Group
% https://github.com/ho-tex/oberdiek/issues
%
% This work consists of the main source file classlist.dtx
% and the derived files
%    classlist.sty, classlist.pdf, classlist.ins, classlist.drv.
%
% Distribution:
%    CTAN:macros/latex/contrib/oberdiek/classlist.dtx
%    CTAN:macros/latex/contrib/oberdiek/classlist.pdf
%
% Unpacking:
%    (a) If classlist.ins is present:
%           tex classlist.ins
%    (b) Without classlist.ins:
%           tex classlist.dtx
%    (c) If you insist on using LaTeX
%           latex \let\install=y% \iffalse meta-comment
%
% File: classlist.dtx
% Version: 2016/05/16 v1.5
% Info: Record classes used in a document
%
% Copyright (C)
%    2005, 2006, 2008, 2011 Heiko Oberdiek
%    2016-2019 Oberdiek Package Support Group
%    https://github.com/ho-tex/oberdiek/issues
%
% This work may be distributed and/or modified under the
% conditions of the LaTeX Project Public License, either
% version 1.3c of this license or (at your option) any later
% version. This version of this license is in
%    https://www.latex-project.org/lppl/lppl-1-3c.txt
% and the latest version of this license is in
%    https://www.latex-project.org/lppl.txt
% and version 1.3 or later is part of all distributions of
% LaTeX version 2005/12/01 or later.
%
% This work has the LPPL maintenance status "maintained".
%
% The Current Maintainers of this work are
% Heiko Oberdiek and the Oberdiek Package Support Group
% https://github.com/ho-tex/oberdiek/issues
%
% This work consists of the main source file classlist.dtx
% and the derived files
%    classlist.sty, classlist.pdf, classlist.ins, classlist.drv.
%
% Distribution:
%    CTAN:macros/latex/contrib/oberdiek/classlist.dtx
%    CTAN:macros/latex/contrib/oberdiek/classlist.pdf
%
% Unpacking:
%    (a) If classlist.ins is present:
%           tex classlist.ins
%    (b) Without classlist.ins:
%           tex classlist.dtx
%    (c) If you insist on using LaTeX
%           latex \let\install=y% \iffalse meta-comment
%
% File: classlist.dtx
% Version: 2016/05/16 v1.5
% Info: Record classes used in a document
%
% Copyright (C)
%    2005, 2006, 2008, 2011 Heiko Oberdiek
%    2016-2019 Oberdiek Package Support Group
%    https://github.com/ho-tex/oberdiek/issues
%
% This work may be distributed and/or modified under the
% conditions of the LaTeX Project Public License, either
% version 1.3c of this license or (at your option) any later
% version. This version of this license is in
%    https://www.latex-project.org/lppl/lppl-1-3c.txt
% and the latest version of this license is in
%    https://www.latex-project.org/lppl.txt
% and version 1.3 or later is part of all distributions of
% LaTeX version 2005/12/01 or later.
%
% This work has the LPPL maintenance status "maintained".
%
% The Current Maintainers of this work are
% Heiko Oberdiek and the Oberdiek Package Support Group
% https://github.com/ho-tex/oberdiek/issues
%
% This work consists of the main source file classlist.dtx
% and the derived files
%    classlist.sty, classlist.pdf, classlist.ins, classlist.drv.
%
% Distribution:
%    CTAN:macros/latex/contrib/oberdiek/classlist.dtx
%    CTAN:macros/latex/contrib/oberdiek/classlist.pdf
%
% Unpacking:
%    (a) If classlist.ins is present:
%           tex classlist.ins
%    (b) Without classlist.ins:
%           tex classlist.dtx
%    (c) If you insist on using LaTeX
%           latex \let\install=y\input{classlist.dtx}
%        (quote the arguments according to the demands of your shell)
%
% Documentation:
%    (a) If classlist.drv is present:
%           latex classlist.drv
%    (b) Without classlist.drv:
%           latex classlist.dtx; ...
%    The class ltxdoc loads the configuration file ltxdoc.cfg
%    if available. Here you can specify further options, e.g.
%    use A4 as paper format:
%       \PassOptionsToClass{a4paper}{article}
%
%    Program calls to get the documentation (example):
%       pdflatex classlist.dtx
%       makeindex -s gind.ist classlist.idx
%       pdflatex classlist.dtx
%       makeindex -s gind.ist classlist.idx
%       pdflatex classlist.dtx
%
% Installation:
%    TDS:tex/latex/oberdiek/classlist.sty
%    TDS:doc/latex/oberdiek/classlist.pdf
%    TDS:source/latex/oberdiek/classlist.dtx
%
%<*ignore>
\begingroup
  \catcode123=1 %
  \catcode125=2 %
  \def\x{LaTeX2e}%
\expandafter\endgroup
\ifcase 0\ifx\install y1\fi\expandafter
         \ifx\csname processbatchFile\endcsname\relax\else1\fi
         \ifx\fmtname\x\else 1\fi\relax
\else\csname fi\endcsname
%</ignore>
%<*install>
\input docstrip.tex
\Msg{************************************************************************}
\Msg{* Installation}
\Msg{* Package: classlist 2016/05/16 v1.5 Record classes used in a document (HO)}
\Msg{************************************************************************}

\keepsilent
\askforoverwritefalse

\let\MetaPrefix\relax
\preamble

This is a generated file.

Project: classlist
Version: 2016/05/16 v1.5

Copyright (C)
   2005, 2006, 2008, 2011 Heiko Oberdiek
   2016-2019 Oberdiek Package Support Group

This work may be distributed and/or modified under the
conditions of the LaTeX Project Public License, either
version 1.3c of this license or (at your option) any later
version. This version of this license is in
   https://www.latex-project.org/lppl/lppl-1-3c.txt
and the latest version of this license is in
   https://www.latex-project.org/lppl.txt
and version 1.3 or later is part of all distributions of
LaTeX version 2005/12/01 or later.

This work has the LPPL maintenance status "maintained".

The Current Maintainers of this work are
Heiko Oberdiek and the Oberdiek Package Support Group
https://github.com/ho-tex/oberdiek/issues


This work consists of the main source file classlist.dtx
and the derived files
   classlist.sty, classlist.pdf, classlist.ins, classlist.drv.

\endpreamble
\let\MetaPrefix\DoubleperCent

\generate{%
  \file{classlist.ins}{\from{classlist.dtx}{install}}%
  \file{classlist.drv}{\from{classlist.dtx}{driver}}%
  \usedir{tex/latex/oberdiek}%
  \file{classlist.sty}{\from{classlist.dtx}{package}}%
}

\catcode32=13\relax% active space
\let =\space%
\Msg{************************************************************************}
\Msg{*}
\Msg{* To finish the installation you have to move the following}
\Msg{* file into a directory searched by TeX:}
\Msg{*}
\Msg{*     classlist.sty}
\Msg{*}
\Msg{* To produce the documentation run the file `classlist.drv'}
\Msg{* through LaTeX.}
\Msg{*}
\Msg{* Happy TeXing!}
\Msg{*}
\Msg{************************************************************************}

\endbatchfile
%</install>
%<*ignore>
\fi
%</ignore>
%<*driver>
\NeedsTeXFormat{LaTeX2e}
\ProvidesFile{classlist.drv}%
  [2016/05/16 v1.5 Record classes used in a document (HO)]%
\documentclass{ltxdoc}
\usepackage{holtxdoc}[2011/11/22]
\begin{document}
  \DocInput{classlist.dtx}%
\end{document}
%</driver>
% \fi
%
%
%
% \GetFileInfo{classlist.drv}
%
% \title{The \xpackage{classlist} package}
% \date{2016/05/16 v1.5}
% \author{Heiko Oberdiek\thanks
% {Please report any issues at \url{https://github.com/ho-tex/oberdiek/issues}}}
%
% \maketitle
%
% \begin{abstract}
% This package records the loaded classes and stores
% them in a list.
% \end{abstract}
%
% \tableofcontents
%
% \section{Documentation}
%
% \subsection{Background}
%
% This packages is an answer of a newsgroup question:
% \begin{quote}
% \begin{tabular}{@{}ll@{}}
%   Newsgroup: & comp.text.tex\\
%   Subject: & Finding the Document Class\\
%   From: & Herber Schulz\\
%   Date: & 18 Jun 2005 13:16:49 -0500\\
%   Message-ID: &
%    \textless
%    \texttt{herbs-D55DB9.13170418062005@news.isp.giganews.com}^^A
%    \textgreater
% \end{tabular}
% \end{quote}
%
% \subsection{Usage}
%
% Load this package before \cs{documentclass}:
% \begin{quote}
%   |\RequirePackage{classlist}|\\
%   |\documentclass[some,options]{whatever}|
% \end{quote}
% It then records the classes with options.
%
% If used after \cs{documentclass}, \cs{@filelist} is
% parsed for classes. The additional data
% specified options and requested version is no
% longer available here.
%
% \begin{description}
% \item[\cs{MainClassName}] contains the first loaded class.
% \item[\cs{ClassList}] stores the class entries, eg.
%   \begin{quote}
%   \begin{tabular}{@{}l@{ }l@{}}
%     \cs{ClassList} $\rightarrow$&
%     |\ClassListEntry{myarticle}{a4paper}{}|\\
%     &|\ClassListEntry{article}{}{}|
%   \end{tabular}
%   \end{quote}
% \item[\cs{ClassListEntry}] has three arguments:
%   \begin{quote}
%   \begin{tabular}{@{}ll@{}}
%     |#1|: & class name\\
%     |#2|: & options given in \cs{documentclass}/\cs{LoadClass}\\
%     |#3|: & requested version, not the version of class
%   \end{tabular}
%   \end{quote}
% \item[\cs{PrintClassList}] prints the list on screen it can be
%    configured by
% \item[\cs{PrintClassListTitle}] for the title and
% \item[\cs{PrintClassListEntry}] for formatting the entries.
%    See the implementation for how to use these.
% \end{description}
%
% \StopEventually{
% }
%
% \section{Implementation}
%
%    \begin{macrocode}
%<*package>
%    \end{macrocode}
%    Package identification.
%    \begin{macrocode}
\NeedsTeXFormat{LaTeX2e}
\ProvidesPackage{classlist}%
  [2016/05/16 v1.5 Record classes used in a document (HO)]
%    \end{macrocode}
%
%    \begin{macrocode}
\let\ClassList\@empty
\let\MainClassName\relax
%    \end{macrocode}
%
%    Test, whether we are called before \cs{documentclass}.
%    \begin{macrocode}
\ifx\@classoptionslist\relax
  \let\CL@org@fileswith@pti@ns\@fileswith@pti@ns
  \def\@fileswith@pti@ns#1[#2]#3[#4]{%
%    \end{macrocode}
%    \begin{tabular}{@{}ll@{}}
%      |#1|: & \cs{@clsextension}\\
%      |#2|: & options of \cs{documentclass}/\cs{LoadClass}\\
%      |#3|: & class name\\
%      |#4|: & requested version
%    \end{tabular}
%    \begin{macrocode}
    \ifx#1\@clsextension
      \@ifl@aded#1{#3}{%
        \PackageInfo{classlist}{%
          Skipping class `#3', because\MessageBreak
          this class is already loaded%
        }%
      }{%
        \@ifundefined{MainClassName}{%
          \def\MainClassName{#3}%
        }{}%
        \@temptokena\expandafter{%
          \ClassList
          \ClassListEntry{#3}{#2}{#4}%
        }%
        \edef\ClassList{\the\@temptokena}%
      }%
    \fi
    \CL@org@fileswith@pti@ns{#1}[{#2}]{#3}[{#4}]%
  }%
  \let\@@fileswith@pti@ns\@fileswith@pti@ns
\else
%    \end{macrocode}
%    Called after \cs{documentclass}.
%    \begin{macrocode}
  \PackageInfo{classlist}{Use \string\@filelist\space method}%

  \let\ClassListEntry\relax
  \expandafter\def\expandafter\CL@test
      \expandafter#\expandafter1\@clsextension#2\@nil{%
    \ifx\\#2\\%
%    \end{macrocode}
%    Name does not contain \cs{@clsextension}
%    \begin{macrocode}
    \else
      \expandafter\CL@test@i\CL@entry\@nil
    \fi
  }%
  \expandafter\def\expandafter\CL@test@i
      \expandafter#\expandafter1\@clsextension#2\@nil{%
    \ifx\\#2\\%
      \@ifundefined{opt@\CL@entry}{%
      }{%
        \@ifundefined{MainClassName}{%
          \let\MainClassName\CL@entry
        }{%
        }%
        \edef\ClassList{%
          \ClassList
          \ClassListEntry{\CL@entry}{}{}%
        }%
      }%
    \else
%    \end{macrocode}
%    Names with more than one \cs{@clsextension} are not supported.
%    \begin{macrocode}
    \fi
  }%
  \@for\CL@entry:=\@filelist\do{%
    \expandafter\expandafter\expandafter\CL@test\expandafter
        \CL@entry\@clsextension\@nil
  }%
\fi
%    \end{macrocode}
%
%    \begin{macro}{\PrintClassListEntry}
%    \begin{macrocode}
\providecommand*{\PrintClassListEntry}[3]{%
  \toks@{* #1}%
  \typeout{\the\toks@}%
}
%    \end{macrocode}
%    \end{macro}
%    \begin{macro}{\PrintClassListTitle}
%    \begin{macrocode}
\providecommand*{\PrintClassListTitle}{%
  \typeout{Class list:}%
}
%    \end{macrocode}
%    \end{macro}
%    \begin{macro}{\PrintClassList}
%    \begin{macrocode}
\providecommand*{\PrintClassList}{%
  \begingroup
    \let\ClassListEntry\PrintClassListEntry
    \PrintClassListTitle
    \ClassList
  \endgroup
}
%    \end{macrocode}
%    \end{macro}
%    \begin{macro}{\CL@InfoEntry}
%    \begin{macrocode}
\def\CL@InfoEntry#1#2#3{%
  \advance\count@ by \@ne
  \def\x{#2}%
  \@onelevel@sanitize\x
  \edef\CL@Info{%
    \CL@Info
    \noexpand\MessageBreak
    (\the\count@) %
    #1 [\x]%
    \ifx\\#3\\%
    \else
      \space[#3]% hash-ok
    \fi
  }%
}
%    \end{macrocode}
%    \end{macro}
%    \begin{macrocode}
\AtBeginDocument{%
  \begingroup
    \count@=\z@
    \def\CL@Info{Class List:}%
    \let\ClassListEntry\CL@InfoEntry
    \ClassList
    \let\on@line\@empty
    \PackageInfo{classlist}{\CL@Info}%
  \endgroup
}
%    \end{macrocode}
%
%    \begin{macrocode}
%</package>
%    \end{macrocode}
%
% \section{Installation}
%
% \subsection{Download}
%
% \paragraph{Package.} This package is available on
% CTAN\footnote{\CTANpkg{classlist}}:
% \begin{description}
% \item[\CTAN{macros/latex/contrib/oberdiek/classlist.dtx}] The source file.
% \item[\CTAN{macros/latex/contrib/oberdiek/classlist.pdf}] Documentation.
% \end{description}
%
%
% \paragraph{Bundle.} All the packages of the bundle `oberdiek'
% are also available in a TDS compliant ZIP archive. There
% the packages are already unpacked and the documentation files
% are generated. The files and directories obey the TDS standard.
% \begin{description}
% \item[\CTANinstall{install/macros/latex/contrib/oberdiek.tds.zip}]
% \end{description}
% \emph{TDS} refers to the standard ``A Directory Structure
% for \TeX\ Files'' (\CTANpkg{tds}). Directories
% with \xfile{texmf} in their name are usually organized this way.
%
% \subsection{Bundle installation}
%
% \paragraph{Unpacking.} Unpack the \xfile{oberdiek.tds.zip} in the
% TDS tree (also known as \xfile{texmf} tree) of your choice.
% Example (linux):
% \begin{quote}
%   |unzip oberdiek.tds.zip -d ~/texmf|
% \end{quote}
%
% \subsection{Package installation}
%
% \paragraph{Unpacking.} The \xfile{.dtx} file is a self-extracting
% \docstrip\ archive. The files are extracted by running the
% \xfile{.dtx} through \plainTeX:
% \begin{quote}
%   \verb|tex classlist.dtx|
% \end{quote}
%
% \paragraph{TDS.} Now the different files must be moved into
% the different directories in your installation TDS tree
% (also known as \xfile{texmf} tree):
% \begin{quote}
% \def\t{^^A
% \begin{tabular}{@{}>{\ttfamily}l@{ $\rightarrow$ }>{\ttfamily}l@{}}
%   classlist.sty & tex/latex/oberdiek/classlist.sty\\
%   classlist.pdf & doc/latex/oberdiek/classlist.pdf\\
%   classlist.dtx & source/latex/oberdiek/classlist.dtx\\
% \end{tabular}^^A
% }^^A
% \sbox0{\t}^^A
% \ifdim\wd0>\linewidth
%   \begingroup
%     \advance\linewidth by\leftmargin
%     \advance\linewidth by\rightmargin
%   \edef\x{\endgroup
%     \def\noexpand\lw{\the\linewidth}^^A
%   }\x
%   \def\lwbox{^^A
%     \leavevmode
%     \hbox to \linewidth{^^A
%       \kern-\leftmargin\relax
%       \hss
%       \usebox0
%       \hss
%       \kern-\rightmargin\relax
%     }^^A
%   }^^A
%   \ifdim\wd0>\lw
%     \sbox0{\small\t}^^A
%     \ifdim\wd0>\linewidth
%       \ifdim\wd0>\lw
%         \sbox0{\footnotesize\t}^^A
%         \ifdim\wd0>\linewidth
%           \ifdim\wd0>\lw
%             \sbox0{\scriptsize\t}^^A
%             \ifdim\wd0>\linewidth
%               \ifdim\wd0>\lw
%                 \sbox0{\tiny\t}^^A
%                 \ifdim\wd0>\linewidth
%                   \lwbox
%                 \else
%                   \usebox0
%                 \fi
%               \else
%                 \lwbox
%               \fi
%             \else
%               \usebox0
%             \fi
%           \else
%             \lwbox
%           \fi
%         \else
%           \usebox0
%         \fi
%       \else
%         \lwbox
%       \fi
%     \else
%       \usebox0
%     \fi
%   \else
%     \lwbox
%   \fi
% \else
%   \usebox0
% \fi
% \end{quote}
% If you have a \xfile{docstrip.cfg} that configures and enables \docstrip's
% TDS installing feature, then some files can already be in the right
% place, see the documentation of \docstrip.
%
% \subsection{Refresh file name databases}
%
% If your \TeX~distribution
% (\TeX\,Live, \mikTeX, \dots) relies on file name databases, you must refresh
% these. For example, \TeX\,Live\ users run \verb|texhash| or
% \verb|mktexlsr|.
%
% \subsection{Some details for the interested}
%
% \paragraph{Unpacking with \LaTeX.}
% The \xfile{.dtx} chooses its action depending on the format:
% \begin{description}
% \item[\plainTeX:] Run \docstrip\ and extract the files.
% \item[\LaTeX:] Generate the documentation.
% \end{description}
% If you insist on using \LaTeX\ for \docstrip\ (really,
% \docstrip\ does not need \LaTeX), then inform the autodetect routine
% about your intention:
% \begin{quote}
%   \verb|latex \let\install=y\input{classlist.dtx}|
% \end{quote}
% Do not forget to quote the argument according to the demands
% of your shell.
%
% \paragraph{Generating the documentation.}
% You can use both the \xfile{.dtx} or the \xfile{.drv} to generate
% the documentation. The process can be configured by the
% configuration file \xfile{ltxdoc.cfg}. For instance, put this
% line into this file, if you want to have A4 as paper format:
% \begin{quote}
%   \verb|\PassOptionsToClass{a4paper}{article}|
% \end{quote}
% An example follows how to generate the
% documentation with pdf\LaTeX:
% \begin{quote}
%\begin{verbatim}
%pdflatex classlist.dtx
%makeindex -s gind.ist classlist.idx
%pdflatex classlist.dtx
%makeindex -s gind.ist classlist.idx
%pdflatex classlist.dtx
%\end{verbatim}
% \end{quote}
%
% \begin{History}
%   \begin{Version}{2005/06/19 v1.0}
%   \item
%     First published version: CTAN and newsgroup \xnewsgroup{comp.text.tex}:
%     \URL{``\link{Re: Finding the Document Class}''}^^A
%     {https://groups.google.com/group/comp.text.tex/msg/8ee9523c2dc13666}
%   \end{Version}
%   \begin{Version}{2005/06/19 v1.1}
%   \item
%     After \cs{documentclass} the package looks
%     at \cs{@filelist} instead of aborting with error.
%   \end{Version}
%   \begin{Version}{2006/02/20 v1.2}
%   \item
%     DTX framework.
%   \item
%     Fix for \cs{@@fileswith@pti@ns}.
%   \end{Version}
%   \begin{Version}{2008/08/11 v1.3}
%   \item
%     Code is not changed.
%   \item
%     URLs updated.
%   \end{Version}
%   \begin{Version}{2011/10/17 v1.4}
%   \item
%     Documentation fix: \cs{MainClass} $\rightarrow$ \cs{MainClassName}.
%   \end{Version}
%   \begin{Version}{2016/05/16 v1.5}
%   \item
%     Documentation updates.
%   \end{Version}
% \end{History}
%
% \PrintIndex
%
% \Finale
\endinput

%        (quote the arguments according to the demands of your shell)
%
% Documentation:
%    (a) If classlist.drv is present:
%           latex classlist.drv
%    (b) Without classlist.drv:
%           latex classlist.dtx; ...
%    The class ltxdoc loads the configuration file ltxdoc.cfg
%    if available. Here you can specify further options, e.g.
%    use A4 as paper format:
%       \PassOptionsToClass{a4paper}{article}
%
%    Program calls to get the documentation (example):
%       pdflatex classlist.dtx
%       makeindex -s gind.ist classlist.idx
%       pdflatex classlist.dtx
%       makeindex -s gind.ist classlist.idx
%       pdflatex classlist.dtx
%
% Installation:
%    TDS:tex/latex/oberdiek/classlist.sty
%    TDS:doc/latex/oberdiek/classlist.pdf
%    TDS:source/latex/oberdiek/classlist.dtx
%
%<*ignore>
\begingroup
  \catcode123=1 %
  \catcode125=2 %
  \def\x{LaTeX2e}%
\expandafter\endgroup
\ifcase 0\ifx\install y1\fi\expandafter
         \ifx\csname processbatchFile\endcsname\relax\else1\fi
         \ifx\fmtname\x\else 1\fi\relax
\else\csname fi\endcsname
%</ignore>
%<*install>
\input docstrip.tex
\Msg{************************************************************************}
\Msg{* Installation}
\Msg{* Package: classlist 2016/05/16 v1.5 Record classes used in a document (HO)}
\Msg{************************************************************************}

\keepsilent
\askforoverwritefalse

\let\MetaPrefix\relax
\preamble

This is a generated file.

Project: classlist
Version: 2016/05/16 v1.5

Copyright (C)
   2005, 2006, 2008, 2011 Heiko Oberdiek
   2016-2019 Oberdiek Package Support Group

This work may be distributed and/or modified under the
conditions of the LaTeX Project Public License, either
version 1.3c of this license or (at your option) any later
version. This version of this license is in
   https://www.latex-project.org/lppl/lppl-1-3c.txt
and the latest version of this license is in
   https://www.latex-project.org/lppl.txt
and version 1.3 or later is part of all distributions of
LaTeX version 2005/12/01 or later.

This work has the LPPL maintenance status "maintained".

The Current Maintainers of this work are
Heiko Oberdiek and the Oberdiek Package Support Group
https://github.com/ho-tex/oberdiek/issues


This work consists of the main source file classlist.dtx
and the derived files
   classlist.sty, classlist.pdf, classlist.ins, classlist.drv.

\endpreamble
\let\MetaPrefix\DoubleperCent

\generate{%
  \file{classlist.ins}{\from{classlist.dtx}{install}}%
  \file{classlist.drv}{\from{classlist.dtx}{driver}}%
  \usedir{tex/latex/oberdiek}%
  \file{classlist.sty}{\from{classlist.dtx}{package}}%
}

\catcode32=13\relax% active space
\let =\space%
\Msg{************************************************************************}
\Msg{*}
\Msg{* To finish the installation you have to move the following}
\Msg{* file into a directory searched by TeX:}
\Msg{*}
\Msg{*     classlist.sty}
\Msg{*}
\Msg{* To produce the documentation run the file `classlist.drv'}
\Msg{* through LaTeX.}
\Msg{*}
\Msg{* Happy TeXing!}
\Msg{*}
\Msg{************************************************************************}

\endbatchfile
%</install>
%<*ignore>
\fi
%</ignore>
%<*driver>
\NeedsTeXFormat{LaTeX2e}
\ProvidesFile{classlist.drv}%
  [2016/05/16 v1.5 Record classes used in a document (HO)]%
\documentclass{ltxdoc}
\usepackage{holtxdoc}[2011/11/22]
\begin{document}
  \DocInput{classlist.dtx}%
\end{document}
%</driver>
% \fi
%
%
%
% \GetFileInfo{classlist.drv}
%
% \title{The \xpackage{classlist} package}
% \date{2016/05/16 v1.5}
% \author{Heiko Oberdiek\thanks
% {Please report any issues at \url{https://github.com/ho-tex/oberdiek/issues}}}
%
% \maketitle
%
% \begin{abstract}
% This package records the loaded classes and stores
% them in a list.
% \end{abstract}
%
% \tableofcontents
%
% \section{Documentation}
%
% \subsection{Background}
%
% This packages is an answer of a newsgroup question:
% \begin{quote}
% \begin{tabular}{@{}ll@{}}
%   Newsgroup: & comp.text.tex\\
%   Subject: & Finding the Document Class\\
%   From: & Herber Schulz\\
%   Date: & 18 Jun 2005 13:16:49 -0500\\
%   Message-ID: &
%    \textless
%    \texttt{herbs-D55DB9.13170418062005@news.isp.giganews.com}^^A
%    \textgreater
% \end{tabular}
% \end{quote}
%
% \subsection{Usage}
%
% Load this package before \cs{documentclass}:
% \begin{quote}
%   |\RequirePackage{classlist}|\\
%   |\documentclass[some,options]{whatever}|
% \end{quote}
% It then records the classes with options.
%
% If used after \cs{documentclass}, \cs{@filelist} is
% parsed for classes. The additional data
% specified options and requested version is no
% longer available here.
%
% \begin{description}
% \item[\cs{MainClassName}] contains the first loaded class.
% \item[\cs{ClassList}] stores the class entries, eg.
%   \begin{quote}
%   \begin{tabular}{@{}l@{ }l@{}}
%     \cs{ClassList} $\rightarrow$&
%     |\ClassListEntry{myarticle}{a4paper}{}|\\
%     &|\ClassListEntry{article}{}{}|
%   \end{tabular}
%   \end{quote}
% \item[\cs{ClassListEntry}] has three arguments:
%   \begin{quote}
%   \begin{tabular}{@{}ll@{}}
%     |#1|: & class name\\
%     |#2|: & options given in \cs{documentclass}/\cs{LoadClass}\\
%     |#3|: & requested version, not the version of class
%   \end{tabular}
%   \end{quote}
% \item[\cs{PrintClassList}] prints the list on screen it can be
%    configured by
% \item[\cs{PrintClassListTitle}] for the title and
% \item[\cs{PrintClassListEntry}] for formatting the entries.
%    See the implementation for how to use these.
% \end{description}
%
% \StopEventually{
% }
%
% \section{Implementation}
%
%    \begin{macrocode}
%<*package>
%    \end{macrocode}
%    Package identification.
%    \begin{macrocode}
\NeedsTeXFormat{LaTeX2e}
\ProvidesPackage{classlist}%
  [2016/05/16 v1.5 Record classes used in a document (HO)]
%    \end{macrocode}
%
%    \begin{macrocode}
\let\ClassList\@empty
\let\MainClassName\relax
%    \end{macrocode}
%
%    Test, whether we are called before \cs{documentclass}.
%    \begin{macrocode}
\ifx\@classoptionslist\relax
  \let\CL@org@fileswith@pti@ns\@fileswith@pti@ns
  \def\@fileswith@pti@ns#1[#2]#3[#4]{%
%    \end{macrocode}
%    \begin{tabular}{@{}ll@{}}
%      |#1|: & \cs{@clsextension}\\
%      |#2|: & options of \cs{documentclass}/\cs{LoadClass}\\
%      |#3|: & class name\\
%      |#4|: & requested version
%    \end{tabular}
%    \begin{macrocode}
    \ifx#1\@clsextension
      \@ifl@aded#1{#3}{%
        \PackageInfo{classlist}{%
          Skipping class `#3', because\MessageBreak
          this class is already loaded%
        }%
      }{%
        \@ifundefined{MainClassName}{%
          \def\MainClassName{#3}%
        }{}%
        \@temptokena\expandafter{%
          \ClassList
          \ClassListEntry{#3}{#2}{#4}%
        }%
        \edef\ClassList{\the\@temptokena}%
      }%
    \fi
    \CL@org@fileswith@pti@ns{#1}[{#2}]{#3}[{#4}]%
  }%
  \let\@@fileswith@pti@ns\@fileswith@pti@ns
\else
%    \end{macrocode}
%    Called after \cs{documentclass}.
%    \begin{macrocode}
  \PackageInfo{classlist}{Use \string\@filelist\space method}%

  \let\ClassListEntry\relax
  \expandafter\def\expandafter\CL@test
      \expandafter#\expandafter1\@clsextension#2\@nil{%
    \ifx\\#2\\%
%    \end{macrocode}
%    Name does not contain \cs{@clsextension}
%    \begin{macrocode}
    \else
      \expandafter\CL@test@i\CL@entry\@nil
    \fi
  }%
  \expandafter\def\expandafter\CL@test@i
      \expandafter#\expandafter1\@clsextension#2\@nil{%
    \ifx\\#2\\%
      \@ifundefined{opt@\CL@entry}{%
      }{%
        \@ifundefined{MainClassName}{%
          \let\MainClassName\CL@entry
        }{%
        }%
        \edef\ClassList{%
          \ClassList
          \ClassListEntry{\CL@entry}{}{}%
        }%
      }%
    \else
%    \end{macrocode}
%    Names with more than one \cs{@clsextension} are not supported.
%    \begin{macrocode}
    \fi
  }%
  \@for\CL@entry:=\@filelist\do{%
    \expandafter\expandafter\expandafter\CL@test\expandafter
        \CL@entry\@clsextension\@nil
  }%
\fi
%    \end{macrocode}
%
%    \begin{macro}{\PrintClassListEntry}
%    \begin{macrocode}
\providecommand*{\PrintClassListEntry}[3]{%
  \toks@{* #1}%
  \typeout{\the\toks@}%
}
%    \end{macrocode}
%    \end{macro}
%    \begin{macro}{\PrintClassListTitle}
%    \begin{macrocode}
\providecommand*{\PrintClassListTitle}{%
  \typeout{Class list:}%
}
%    \end{macrocode}
%    \end{macro}
%    \begin{macro}{\PrintClassList}
%    \begin{macrocode}
\providecommand*{\PrintClassList}{%
  \begingroup
    \let\ClassListEntry\PrintClassListEntry
    \PrintClassListTitle
    \ClassList
  \endgroup
}
%    \end{macrocode}
%    \end{macro}
%    \begin{macro}{\CL@InfoEntry}
%    \begin{macrocode}
\def\CL@InfoEntry#1#2#3{%
  \advance\count@ by \@ne
  \def\x{#2}%
  \@onelevel@sanitize\x
  \edef\CL@Info{%
    \CL@Info
    \noexpand\MessageBreak
    (\the\count@) %
    #1 [\x]%
    \ifx\\#3\\%
    \else
      \space[#3]% hash-ok
    \fi
  }%
}
%    \end{macrocode}
%    \end{macro}
%    \begin{macrocode}
\AtBeginDocument{%
  \begingroup
    \count@=\z@
    \def\CL@Info{Class List:}%
    \let\ClassListEntry\CL@InfoEntry
    \ClassList
    \let\on@line\@empty
    \PackageInfo{classlist}{\CL@Info}%
  \endgroup
}
%    \end{macrocode}
%
%    \begin{macrocode}
%</package>
%    \end{macrocode}
%
% \section{Installation}
%
% \subsection{Download}
%
% \paragraph{Package.} This package is available on
% CTAN\footnote{\CTANpkg{classlist}}:
% \begin{description}
% \item[\CTAN{macros/latex/contrib/oberdiek/classlist.dtx}] The source file.
% \item[\CTAN{macros/latex/contrib/oberdiek/classlist.pdf}] Documentation.
% \end{description}
%
%
% \paragraph{Bundle.} All the packages of the bundle `oberdiek'
% are also available in a TDS compliant ZIP archive. There
% the packages are already unpacked and the documentation files
% are generated. The files and directories obey the TDS standard.
% \begin{description}
% \item[\CTANinstall{install/macros/latex/contrib/oberdiek.tds.zip}]
% \end{description}
% \emph{TDS} refers to the standard ``A Directory Structure
% for \TeX\ Files'' (\CTANpkg{tds}). Directories
% with \xfile{texmf} in their name are usually organized this way.
%
% \subsection{Bundle installation}
%
% \paragraph{Unpacking.} Unpack the \xfile{oberdiek.tds.zip} in the
% TDS tree (also known as \xfile{texmf} tree) of your choice.
% Example (linux):
% \begin{quote}
%   |unzip oberdiek.tds.zip -d ~/texmf|
% \end{quote}
%
% \subsection{Package installation}
%
% \paragraph{Unpacking.} The \xfile{.dtx} file is a self-extracting
% \docstrip\ archive. The files are extracted by running the
% \xfile{.dtx} through \plainTeX:
% \begin{quote}
%   \verb|tex classlist.dtx|
% \end{quote}
%
% \paragraph{TDS.} Now the different files must be moved into
% the different directories in your installation TDS tree
% (also known as \xfile{texmf} tree):
% \begin{quote}
% \def\t{^^A
% \begin{tabular}{@{}>{\ttfamily}l@{ $\rightarrow$ }>{\ttfamily}l@{}}
%   classlist.sty & tex/latex/oberdiek/classlist.sty\\
%   classlist.pdf & doc/latex/oberdiek/classlist.pdf\\
%   classlist.dtx & source/latex/oberdiek/classlist.dtx\\
% \end{tabular}^^A
% }^^A
% \sbox0{\t}^^A
% \ifdim\wd0>\linewidth
%   \begingroup
%     \advance\linewidth by\leftmargin
%     \advance\linewidth by\rightmargin
%   \edef\x{\endgroup
%     \def\noexpand\lw{\the\linewidth}^^A
%   }\x
%   \def\lwbox{^^A
%     \leavevmode
%     \hbox to \linewidth{^^A
%       \kern-\leftmargin\relax
%       \hss
%       \usebox0
%       \hss
%       \kern-\rightmargin\relax
%     }^^A
%   }^^A
%   \ifdim\wd0>\lw
%     \sbox0{\small\t}^^A
%     \ifdim\wd0>\linewidth
%       \ifdim\wd0>\lw
%         \sbox0{\footnotesize\t}^^A
%         \ifdim\wd0>\linewidth
%           \ifdim\wd0>\lw
%             \sbox0{\scriptsize\t}^^A
%             \ifdim\wd0>\linewidth
%               \ifdim\wd0>\lw
%                 \sbox0{\tiny\t}^^A
%                 \ifdim\wd0>\linewidth
%                   \lwbox
%                 \else
%                   \usebox0
%                 \fi
%               \else
%                 \lwbox
%               \fi
%             \else
%               \usebox0
%             \fi
%           \else
%             \lwbox
%           \fi
%         \else
%           \usebox0
%         \fi
%       \else
%         \lwbox
%       \fi
%     \else
%       \usebox0
%     \fi
%   \else
%     \lwbox
%   \fi
% \else
%   \usebox0
% \fi
% \end{quote}
% If you have a \xfile{docstrip.cfg} that configures and enables \docstrip's
% TDS installing feature, then some files can already be in the right
% place, see the documentation of \docstrip.
%
% \subsection{Refresh file name databases}
%
% If your \TeX~distribution
% (\TeX\,Live, \mikTeX, \dots) relies on file name databases, you must refresh
% these. For example, \TeX\,Live\ users run \verb|texhash| or
% \verb|mktexlsr|.
%
% \subsection{Some details for the interested}
%
% \paragraph{Unpacking with \LaTeX.}
% The \xfile{.dtx} chooses its action depending on the format:
% \begin{description}
% \item[\plainTeX:] Run \docstrip\ and extract the files.
% \item[\LaTeX:] Generate the documentation.
% \end{description}
% If you insist on using \LaTeX\ for \docstrip\ (really,
% \docstrip\ does not need \LaTeX), then inform the autodetect routine
% about your intention:
% \begin{quote}
%   \verb|latex \let\install=y% \iffalse meta-comment
%
% File: classlist.dtx
% Version: 2016/05/16 v1.5
% Info: Record classes used in a document
%
% Copyright (C)
%    2005, 2006, 2008, 2011 Heiko Oberdiek
%    2016-2019 Oberdiek Package Support Group
%    https://github.com/ho-tex/oberdiek/issues
%
% This work may be distributed and/or modified under the
% conditions of the LaTeX Project Public License, either
% version 1.3c of this license or (at your option) any later
% version. This version of this license is in
%    https://www.latex-project.org/lppl/lppl-1-3c.txt
% and the latest version of this license is in
%    https://www.latex-project.org/lppl.txt
% and version 1.3 or later is part of all distributions of
% LaTeX version 2005/12/01 or later.
%
% This work has the LPPL maintenance status "maintained".
%
% The Current Maintainers of this work are
% Heiko Oberdiek and the Oberdiek Package Support Group
% https://github.com/ho-tex/oberdiek/issues
%
% This work consists of the main source file classlist.dtx
% and the derived files
%    classlist.sty, classlist.pdf, classlist.ins, classlist.drv.
%
% Distribution:
%    CTAN:macros/latex/contrib/oberdiek/classlist.dtx
%    CTAN:macros/latex/contrib/oberdiek/classlist.pdf
%
% Unpacking:
%    (a) If classlist.ins is present:
%           tex classlist.ins
%    (b) Without classlist.ins:
%           tex classlist.dtx
%    (c) If you insist on using LaTeX
%           latex \let\install=y\input{classlist.dtx}
%        (quote the arguments according to the demands of your shell)
%
% Documentation:
%    (a) If classlist.drv is present:
%           latex classlist.drv
%    (b) Without classlist.drv:
%           latex classlist.dtx; ...
%    The class ltxdoc loads the configuration file ltxdoc.cfg
%    if available. Here you can specify further options, e.g.
%    use A4 as paper format:
%       \PassOptionsToClass{a4paper}{article}
%
%    Program calls to get the documentation (example):
%       pdflatex classlist.dtx
%       makeindex -s gind.ist classlist.idx
%       pdflatex classlist.dtx
%       makeindex -s gind.ist classlist.idx
%       pdflatex classlist.dtx
%
% Installation:
%    TDS:tex/latex/oberdiek/classlist.sty
%    TDS:doc/latex/oberdiek/classlist.pdf
%    TDS:source/latex/oberdiek/classlist.dtx
%
%<*ignore>
\begingroup
  \catcode123=1 %
  \catcode125=2 %
  \def\x{LaTeX2e}%
\expandafter\endgroup
\ifcase 0\ifx\install y1\fi\expandafter
         \ifx\csname processbatchFile\endcsname\relax\else1\fi
         \ifx\fmtname\x\else 1\fi\relax
\else\csname fi\endcsname
%</ignore>
%<*install>
\input docstrip.tex
\Msg{************************************************************************}
\Msg{* Installation}
\Msg{* Package: classlist 2016/05/16 v1.5 Record classes used in a document (HO)}
\Msg{************************************************************************}

\keepsilent
\askforoverwritefalse

\let\MetaPrefix\relax
\preamble

This is a generated file.

Project: classlist
Version: 2016/05/16 v1.5

Copyright (C)
   2005, 2006, 2008, 2011 Heiko Oberdiek
   2016-2019 Oberdiek Package Support Group

This work may be distributed and/or modified under the
conditions of the LaTeX Project Public License, either
version 1.3c of this license or (at your option) any later
version. This version of this license is in
   https://www.latex-project.org/lppl/lppl-1-3c.txt
and the latest version of this license is in
   https://www.latex-project.org/lppl.txt
and version 1.3 or later is part of all distributions of
LaTeX version 2005/12/01 or later.

This work has the LPPL maintenance status "maintained".

The Current Maintainers of this work are
Heiko Oberdiek and the Oberdiek Package Support Group
https://github.com/ho-tex/oberdiek/issues


This work consists of the main source file classlist.dtx
and the derived files
   classlist.sty, classlist.pdf, classlist.ins, classlist.drv.

\endpreamble
\let\MetaPrefix\DoubleperCent

\generate{%
  \file{classlist.ins}{\from{classlist.dtx}{install}}%
  \file{classlist.drv}{\from{classlist.dtx}{driver}}%
  \usedir{tex/latex/oberdiek}%
  \file{classlist.sty}{\from{classlist.dtx}{package}}%
}

\catcode32=13\relax% active space
\let =\space%
\Msg{************************************************************************}
\Msg{*}
\Msg{* To finish the installation you have to move the following}
\Msg{* file into a directory searched by TeX:}
\Msg{*}
\Msg{*     classlist.sty}
\Msg{*}
\Msg{* To produce the documentation run the file `classlist.drv'}
\Msg{* through LaTeX.}
\Msg{*}
\Msg{* Happy TeXing!}
\Msg{*}
\Msg{************************************************************************}

\endbatchfile
%</install>
%<*ignore>
\fi
%</ignore>
%<*driver>
\NeedsTeXFormat{LaTeX2e}
\ProvidesFile{classlist.drv}%
  [2016/05/16 v1.5 Record classes used in a document (HO)]%
\documentclass{ltxdoc}
\usepackage{holtxdoc}[2011/11/22]
\begin{document}
  \DocInput{classlist.dtx}%
\end{document}
%</driver>
% \fi
%
%
%
% \GetFileInfo{classlist.drv}
%
% \title{The \xpackage{classlist} package}
% \date{2016/05/16 v1.5}
% \author{Heiko Oberdiek\thanks
% {Please report any issues at \url{https://github.com/ho-tex/oberdiek/issues}}}
%
% \maketitle
%
% \begin{abstract}
% This package records the loaded classes and stores
% them in a list.
% \end{abstract}
%
% \tableofcontents
%
% \section{Documentation}
%
% \subsection{Background}
%
% This packages is an answer of a newsgroup question:
% \begin{quote}
% \begin{tabular}{@{}ll@{}}
%   Newsgroup: & comp.text.tex\\
%   Subject: & Finding the Document Class\\
%   From: & Herber Schulz\\
%   Date: & 18 Jun 2005 13:16:49 -0500\\
%   Message-ID: &
%    \textless
%    \texttt{herbs-D55DB9.13170418062005@news.isp.giganews.com}^^A
%    \textgreater
% \end{tabular}
% \end{quote}
%
% \subsection{Usage}
%
% Load this package before \cs{documentclass}:
% \begin{quote}
%   |\RequirePackage{classlist}|\\
%   |\documentclass[some,options]{whatever}|
% \end{quote}
% It then records the classes with options.
%
% If used after \cs{documentclass}, \cs{@filelist} is
% parsed for classes. The additional data
% specified options and requested version is no
% longer available here.
%
% \begin{description}
% \item[\cs{MainClassName}] contains the first loaded class.
% \item[\cs{ClassList}] stores the class entries, eg.
%   \begin{quote}
%   \begin{tabular}{@{}l@{ }l@{}}
%     \cs{ClassList} $\rightarrow$&
%     |\ClassListEntry{myarticle}{a4paper}{}|\\
%     &|\ClassListEntry{article}{}{}|
%   \end{tabular}
%   \end{quote}
% \item[\cs{ClassListEntry}] has three arguments:
%   \begin{quote}
%   \begin{tabular}{@{}ll@{}}
%     |#1|: & class name\\
%     |#2|: & options given in \cs{documentclass}/\cs{LoadClass}\\
%     |#3|: & requested version, not the version of class
%   \end{tabular}
%   \end{quote}
% \item[\cs{PrintClassList}] prints the list on screen it can be
%    configured by
% \item[\cs{PrintClassListTitle}] for the title and
% \item[\cs{PrintClassListEntry}] for formatting the entries.
%    See the implementation for how to use these.
% \end{description}
%
% \StopEventually{
% }
%
% \section{Implementation}
%
%    \begin{macrocode}
%<*package>
%    \end{macrocode}
%    Package identification.
%    \begin{macrocode}
\NeedsTeXFormat{LaTeX2e}
\ProvidesPackage{classlist}%
  [2016/05/16 v1.5 Record classes used in a document (HO)]
%    \end{macrocode}
%
%    \begin{macrocode}
\let\ClassList\@empty
\let\MainClassName\relax
%    \end{macrocode}
%
%    Test, whether we are called before \cs{documentclass}.
%    \begin{macrocode}
\ifx\@classoptionslist\relax
  \let\CL@org@fileswith@pti@ns\@fileswith@pti@ns
  \def\@fileswith@pti@ns#1[#2]#3[#4]{%
%    \end{macrocode}
%    \begin{tabular}{@{}ll@{}}
%      |#1|: & \cs{@clsextension}\\
%      |#2|: & options of \cs{documentclass}/\cs{LoadClass}\\
%      |#3|: & class name\\
%      |#4|: & requested version
%    \end{tabular}
%    \begin{macrocode}
    \ifx#1\@clsextension
      \@ifl@aded#1{#3}{%
        \PackageInfo{classlist}{%
          Skipping class `#3', because\MessageBreak
          this class is already loaded%
        }%
      }{%
        \@ifundefined{MainClassName}{%
          \def\MainClassName{#3}%
        }{}%
        \@temptokena\expandafter{%
          \ClassList
          \ClassListEntry{#3}{#2}{#4}%
        }%
        \edef\ClassList{\the\@temptokena}%
      }%
    \fi
    \CL@org@fileswith@pti@ns{#1}[{#2}]{#3}[{#4}]%
  }%
  \let\@@fileswith@pti@ns\@fileswith@pti@ns
\else
%    \end{macrocode}
%    Called after \cs{documentclass}.
%    \begin{macrocode}
  \PackageInfo{classlist}{Use \string\@filelist\space method}%

  \let\ClassListEntry\relax
  \expandafter\def\expandafter\CL@test
      \expandafter#\expandafter1\@clsextension#2\@nil{%
    \ifx\\#2\\%
%    \end{macrocode}
%    Name does not contain \cs{@clsextension}
%    \begin{macrocode}
    \else
      \expandafter\CL@test@i\CL@entry\@nil
    \fi
  }%
  \expandafter\def\expandafter\CL@test@i
      \expandafter#\expandafter1\@clsextension#2\@nil{%
    \ifx\\#2\\%
      \@ifundefined{opt@\CL@entry}{%
      }{%
        \@ifundefined{MainClassName}{%
          \let\MainClassName\CL@entry
        }{%
        }%
        \edef\ClassList{%
          \ClassList
          \ClassListEntry{\CL@entry}{}{}%
        }%
      }%
    \else
%    \end{macrocode}
%    Names with more than one \cs{@clsextension} are not supported.
%    \begin{macrocode}
    \fi
  }%
  \@for\CL@entry:=\@filelist\do{%
    \expandafter\expandafter\expandafter\CL@test\expandafter
        \CL@entry\@clsextension\@nil
  }%
\fi
%    \end{macrocode}
%
%    \begin{macro}{\PrintClassListEntry}
%    \begin{macrocode}
\providecommand*{\PrintClassListEntry}[3]{%
  \toks@{* #1}%
  \typeout{\the\toks@}%
}
%    \end{macrocode}
%    \end{macro}
%    \begin{macro}{\PrintClassListTitle}
%    \begin{macrocode}
\providecommand*{\PrintClassListTitle}{%
  \typeout{Class list:}%
}
%    \end{macrocode}
%    \end{macro}
%    \begin{macro}{\PrintClassList}
%    \begin{macrocode}
\providecommand*{\PrintClassList}{%
  \begingroup
    \let\ClassListEntry\PrintClassListEntry
    \PrintClassListTitle
    \ClassList
  \endgroup
}
%    \end{macrocode}
%    \end{macro}
%    \begin{macro}{\CL@InfoEntry}
%    \begin{macrocode}
\def\CL@InfoEntry#1#2#3{%
  \advance\count@ by \@ne
  \def\x{#2}%
  \@onelevel@sanitize\x
  \edef\CL@Info{%
    \CL@Info
    \noexpand\MessageBreak
    (\the\count@) %
    #1 [\x]%
    \ifx\\#3\\%
    \else
      \space[#3]% hash-ok
    \fi
  }%
}
%    \end{macrocode}
%    \end{macro}
%    \begin{macrocode}
\AtBeginDocument{%
  \begingroup
    \count@=\z@
    \def\CL@Info{Class List:}%
    \let\ClassListEntry\CL@InfoEntry
    \ClassList
    \let\on@line\@empty
    \PackageInfo{classlist}{\CL@Info}%
  \endgroup
}
%    \end{macrocode}
%
%    \begin{macrocode}
%</package>
%    \end{macrocode}
%
% \section{Installation}
%
% \subsection{Download}
%
% \paragraph{Package.} This package is available on
% CTAN\footnote{\CTANpkg{classlist}}:
% \begin{description}
% \item[\CTAN{macros/latex/contrib/oberdiek/classlist.dtx}] The source file.
% \item[\CTAN{macros/latex/contrib/oberdiek/classlist.pdf}] Documentation.
% \end{description}
%
%
% \paragraph{Bundle.} All the packages of the bundle `oberdiek'
% are also available in a TDS compliant ZIP archive. There
% the packages are already unpacked and the documentation files
% are generated. The files and directories obey the TDS standard.
% \begin{description}
% \item[\CTANinstall{install/macros/latex/contrib/oberdiek.tds.zip}]
% \end{description}
% \emph{TDS} refers to the standard ``A Directory Structure
% for \TeX\ Files'' (\CTANpkg{tds}). Directories
% with \xfile{texmf} in their name are usually organized this way.
%
% \subsection{Bundle installation}
%
% \paragraph{Unpacking.} Unpack the \xfile{oberdiek.tds.zip} in the
% TDS tree (also known as \xfile{texmf} tree) of your choice.
% Example (linux):
% \begin{quote}
%   |unzip oberdiek.tds.zip -d ~/texmf|
% \end{quote}
%
% \subsection{Package installation}
%
% \paragraph{Unpacking.} The \xfile{.dtx} file is a self-extracting
% \docstrip\ archive. The files are extracted by running the
% \xfile{.dtx} through \plainTeX:
% \begin{quote}
%   \verb|tex classlist.dtx|
% \end{quote}
%
% \paragraph{TDS.} Now the different files must be moved into
% the different directories in your installation TDS tree
% (also known as \xfile{texmf} tree):
% \begin{quote}
% \def\t{^^A
% \begin{tabular}{@{}>{\ttfamily}l@{ $\rightarrow$ }>{\ttfamily}l@{}}
%   classlist.sty & tex/latex/oberdiek/classlist.sty\\
%   classlist.pdf & doc/latex/oberdiek/classlist.pdf\\
%   classlist.dtx & source/latex/oberdiek/classlist.dtx\\
% \end{tabular}^^A
% }^^A
% \sbox0{\t}^^A
% \ifdim\wd0>\linewidth
%   \begingroup
%     \advance\linewidth by\leftmargin
%     \advance\linewidth by\rightmargin
%   \edef\x{\endgroup
%     \def\noexpand\lw{\the\linewidth}^^A
%   }\x
%   \def\lwbox{^^A
%     \leavevmode
%     \hbox to \linewidth{^^A
%       \kern-\leftmargin\relax
%       \hss
%       \usebox0
%       \hss
%       \kern-\rightmargin\relax
%     }^^A
%   }^^A
%   \ifdim\wd0>\lw
%     \sbox0{\small\t}^^A
%     \ifdim\wd0>\linewidth
%       \ifdim\wd0>\lw
%         \sbox0{\footnotesize\t}^^A
%         \ifdim\wd0>\linewidth
%           \ifdim\wd0>\lw
%             \sbox0{\scriptsize\t}^^A
%             \ifdim\wd0>\linewidth
%               \ifdim\wd0>\lw
%                 \sbox0{\tiny\t}^^A
%                 \ifdim\wd0>\linewidth
%                   \lwbox
%                 \else
%                   \usebox0
%                 \fi
%               \else
%                 \lwbox
%               \fi
%             \else
%               \usebox0
%             \fi
%           \else
%             \lwbox
%           \fi
%         \else
%           \usebox0
%         \fi
%       \else
%         \lwbox
%       \fi
%     \else
%       \usebox0
%     \fi
%   \else
%     \lwbox
%   \fi
% \else
%   \usebox0
% \fi
% \end{quote}
% If you have a \xfile{docstrip.cfg} that configures and enables \docstrip's
% TDS installing feature, then some files can already be in the right
% place, see the documentation of \docstrip.
%
% \subsection{Refresh file name databases}
%
% If your \TeX~distribution
% (\TeX\,Live, \mikTeX, \dots) relies on file name databases, you must refresh
% these. For example, \TeX\,Live\ users run \verb|texhash| or
% \verb|mktexlsr|.
%
% \subsection{Some details for the interested}
%
% \paragraph{Unpacking with \LaTeX.}
% The \xfile{.dtx} chooses its action depending on the format:
% \begin{description}
% \item[\plainTeX:] Run \docstrip\ and extract the files.
% \item[\LaTeX:] Generate the documentation.
% \end{description}
% If you insist on using \LaTeX\ for \docstrip\ (really,
% \docstrip\ does not need \LaTeX), then inform the autodetect routine
% about your intention:
% \begin{quote}
%   \verb|latex \let\install=y\input{classlist.dtx}|
% \end{quote}
% Do not forget to quote the argument according to the demands
% of your shell.
%
% \paragraph{Generating the documentation.}
% You can use both the \xfile{.dtx} or the \xfile{.drv} to generate
% the documentation. The process can be configured by the
% configuration file \xfile{ltxdoc.cfg}. For instance, put this
% line into this file, if you want to have A4 as paper format:
% \begin{quote}
%   \verb|\PassOptionsToClass{a4paper}{article}|
% \end{quote}
% An example follows how to generate the
% documentation with pdf\LaTeX:
% \begin{quote}
%\begin{verbatim}
%pdflatex classlist.dtx
%makeindex -s gind.ist classlist.idx
%pdflatex classlist.dtx
%makeindex -s gind.ist classlist.idx
%pdflatex classlist.dtx
%\end{verbatim}
% \end{quote}
%
% \begin{History}
%   \begin{Version}{2005/06/19 v1.0}
%   \item
%     First published version: CTAN and newsgroup \xnewsgroup{comp.text.tex}:
%     \URL{``\link{Re: Finding the Document Class}''}^^A
%     {https://groups.google.com/group/comp.text.tex/msg/8ee9523c2dc13666}
%   \end{Version}
%   \begin{Version}{2005/06/19 v1.1}
%   \item
%     After \cs{documentclass} the package looks
%     at \cs{@filelist} instead of aborting with error.
%   \end{Version}
%   \begin{Version}{2006/02/20 v1.2}
%   \item
%     DTX framework.
%   \item
%     Fix for \cs{@@fileswith@pti@ns}.
%   \end{Version}
%   \begin{Version}{2008/08/11 v1.3}
%   \item
%     Code is not changed.
%   \item
%     URLs updated.
%   \end{Version}
%   \begin{Version}{2011/10/17 v1.4}
%   \item
%     Documentation fix: \cs{MainClass} $\rightarrow$ \cs{MainClassName}.
%   \end{Version}
%   \begin{Version}{2016/05/16 v1.5}
%   \item
%     Documentation updates.
%   \end{Version}
% \end{History}
%
% \PrintIndex
%
% \Finale
\endinput
|
% \end{quote}
% Do not forget to quote the argument according to the demands
% of your shell.
%
% \paragraph{Generating the documentation.}
% You can use both the \xfile{.dtx} or the \xfile{.drv} to generate
% the documentation. The process can be configured by the
% configuration file \xfile{ltxdoc.cfg}. For instance, put this
% line into this file, if you want to have A4 as paper format:
% \begin{quote}
%   \verb|\PassOptionsToClass{a4paper}{article}|
% \end{quote}
% An example follows how to generate the
% documentation with pdf\LaTeX:
% \begin{quote}
%\begin{verbatim}
%pdflatex classlist.dtx
%makeindex -s gind.ist classlist.idx
%pdflatex classlist.dtx
%makeindex -s gind.ist classlist.idx
%pdflatex classlist.dtx
%\end{verbatim}
% \end{quote}
%
% \begin{History}
%   \begin{Version}{2005/06/19 v1.0}
%   \item
%     First published version: CTAN and newsgroup \xnewsgroup{comp.text.tex}:
%     \URL{``\link{Re: Finding the Document Class}''}^^A
%     {https://groups.google.com/group/comp.text.tex/msg/8ee9523c2dc13666}
%   \end{Version}
%   \begin{Version}{2005/06/19 v1.1}
%   \item
%     After \cs{documentclass} the package looks
%     at \cs{@filelist} instead of aborting with error.
%   \end{Version}
%   \begin{Version}{2006/02/20 v1.2}
%   \item
%     DTX framework.
%   \item
%     Fix for \cs{@@fileswith@pti@ns}.
%   \end{Version}
%   \begin{Version}{2008/08/11 v1.3}
%   \item
%     Code is not changed.
%   \item
%     URLs updated.
%   \end{Version}
%   \begin{Version}{2011/10/17 v1.4}
%   \item
%     Documentation fix: \cs{MainClass} $\rightarrow$ \cs{MainClassName}.
%   \end{Version}
%   \begin{Version}{2016/05/16 v1.5}
%   \item
%     Documentation updates.
%   \end{Version}
% \end{History}
%
% \PrintIndex
%
% \Finale
\endinput

%        (quote the arguments according to the demands of your shell)
%
% Documentation:
%    (a) If classlist.drv is present:
%           latex classlist.drv
%    (b) Without classlist.drv:
%           latex classlist.dtx; ...
%    The class ltxdoc loads the configuration file ltxdoc.cfg
%    if available. Here you can specify further options, e.g.
%    use A4 as paper format:
%       \PassOptionsToClass{a4paper}{article}
%
%    Program calls to get the documentation (example):
%       pdflatex classlist.dtx
%       makeindex -s gind.ist classlist.idx
%       pdflatex classlist.dtx
%       makeindex -s gind.ist classlist.idx
%       pdflatex classlist.dtx
%
% Installation:
%    TDS:tex/latex/oberdiek/classlist.sty
%    TDS:doc/latex/oberdiek/classlist.pdf
%    TDS:source/latex/oberdiek/classlist.dtx
%
%<*ignore>
\begingroup
  \catcode123=1 %
  \catcode125=2 %
  \def\x{LaTeX2e}%
\expandafter\endgroup
\ifcase 0\ifx\install y1\fi\expandafter
         \ifx\csname processbatchFile\endcsname\relax\else1\fi
         \ifx\fmtname\x\else 1\fi\relax
\else\csname fi\endcsname
%</ignore>
%<*install>
\input docstrip.tex
\Msg{************************************************************************}
\Msg{* Installation}
\Msg{* Package: classlist 2016/05/16 v1.5 Record classes used in a document (HO)}
\Msg{************************************************************************}

\keepsilent
\askforoverwritefalse

\let\MetaPrefix\relax
\preamble

This is a generated file.

Project: classlist
Version: 2016/05/16 v1.5

Copyright (C)
   2005, 2006, 2008, 2011 Heiko Oberdiek
   2016-2019 Oberdiek Package Support Group

This work may be distributed and/or modified under the
conditions of the LaTeX Project Public License, either
version 1.3c of this license or (at your option) any later
version. This version of this license is in
   https://www.latex-project.org/lppl/lppl-1-3c.txt
and the latest version of this license is in
   https://www.latex-project.org/lppl.txt
and version 1.3 or later is part of all distributions of
LaTeX version 2005/12/01 or later.

This work has the LPPL maintenance status "maintained".

The Current Maintainers of this work are
Heiko Oberdiek and the Oberdiek Package Support Group
https://github.com/ho-tex/oberdiek/issues


This work consists of the main source file classlist.dtx
and the derived files
   classlist.sty, classlist.pdf, classlist.ins, classlist.drv.

\endpreamble
\let\MetaPrefix\DoubleperCent

\generate{%
  \file{classlist.ins}{\from{classlist.dtx}{install}}%
  \file{classlist.drv}{\from{classlist.dtx}{driver}}%
  \usedir{tex/latex/oberdiek}%
  \file{classlist.sty}{\from{classlist.dtx}{package}}%
}

\catcode32=13\relax% active space
\let =\space%
\Msg{************************************************************************}
\Msg{*}
\Msg{* To finish the installation you have to move the following}
\Msg{* file into a directory searched by TeX:}
\Msg{*}
\Msg{*     classlist.sty}
\Msg{*}
\Msg{* To produce the documentation run the file `classlist.drv'}
\Msg{* through LaTeX.}
\Msg{*}
\Msg{* Happy TeXing!}
\Msg{*}
\Msg{************************************************************************}

\endbatchfile
%</install>
%<*ignore>
\fi
%</ignore>
%<*driver>
\NeedsTeXFormat{LaTeX2e}
\ProvidesFile{classlist.drv}%
  [2016/05/16 v1.5 Record classes used in a document (HO)]%
\documentclass{ltxdoc}
\usepackage{holtxdoc}[2011/11/22]
\begin{document}
  \DocInput{classlist.dtx}%
\end{document}
%</driver>
% \fi
%
%
%
% \GetFileInfo{classlist.drv}
%
% \title{The \xpackage{classlist} package}
% \date{2016/05/16 v1.5}
% \author{Heiko Oberdiek\thanks
% {Please report any issues at \url{https://github.com/ho-tex/oberdiek/issues}}}
%
% \maketitle
%
% \begin{abstract}
% This package records the loaded classes and stores
% them in a list.
% \end{abstract}
%
% \tableofcontents
%
% \section{Documentation}
%
% \subsection{Background}
%
% This packages is an answer of a newsgroup question:
% \begin{quote}
% \begin{tabular}{@{}ll@{}}
%   Newsgroup: & comp.text.tex\\
%   Subject: & Finding the Document Class\\
%   From: & Herber Schulz\\
%   Date: & 18 Jun 2005 13:16:49 -0500\\
%   Message-ID: &
%    \textless
%    \texttt{herbs-D55DB9.13170418062005@news.isp.giganews.com}^^A
%    \textgreater
% \end{tabular}
% \end{quote}
%
% \subsection{Usage}
%
% Load this package before \cs{documentclass}:
% \begin{quote}
%   |\RequirePackage{classlist}|\\
%   |\documentclass[some,options]{whatever}|
% \end{quote}
% It then records the classes with options.
%
% If used after \cs{documentclass}, \cs{@filelist} is
% parsed for classes. The additional data
% specified options and requested version is no
% longer available here.
%
% \begin{description}
% \item[\cs{MainClassName}] contains the first loaded class.
% \item[\cs{ClassList}] stores the class entries, eg.
%   \begin{quote}
%   \begin{tabular}{@{}l@{ }l@{}}
%     \cs{ClassList} $\rightarrow$&
%     |\ClassListEntry{myarticle}{a4paper}{}|\\
%     &|\ClassListEntry{article}{}{}|
%   \end{tabular}
%   \end{quote}
% \item[\cs{ClassListEntry}] has three arguments:
%   \begin{quote}
%   \begin{tabular}{@{}ll@{}}
%     |#1|: & class name\\
%     |#2|: & options given in \cs{documentclass}/\cs{LoadClass}\\
%     |#3|: & requested version, not the version of class
%   \end{tabular}
%   \end{quote}
% \item[\cs{PrintClassList}] prints the list on screen it can be
%    configured by
% \item[\cs{PrintClassListTitle}] for the title and
% \item[\cs{PrintClassListEntry}] for formatting the entries.
%    See the implementation for how to use these.
% \end{description}
%
% \StopEventually{
% }
%
% \section{Implementation}
%
%    \begin{macrocode}
%<*package>
%    \end{macrocode}
%    Package identification.
%    \begin{macrocode}
\NeedsTeXFormat{LaTeX2e}
\ProvidesPackage{classlist}%
  [2016/05/16 v1.5 Record classes used in a document (HO)]
%    \end{macrocode}
%
%    \begin{macrocode}
\let\ClassList\@empty
\let\MainClassName\relax
%    \end{macrocode}
%
%    Test, whether we are called before \cs{documentclass}.
%    \begin{macrocode}
\ifx\@classoptionslist\relax
  \let\CL@org@fileswith@pti@ns\@fileswith@pti@ns
  \def\@fileswith@pti@ns#1[#2]#3[#4]{%
%    \end{macrocode}
%    \begin{tabular}{@{}ll@{}}
%      |#1|: & \cs{@clsextension}\\
%      |#2|: & options of \cs{documentclass}/\cs{LoadClass}\\
%      |#3|: & class name\\
%      |#4|: & requested version
%    \end{tabular}
%    \begin{macrocode}
    \ifx#1\@clsextension
      \@ifl@aded#1{#3}{%
        \PackageInfo{classlist}{%
          Skipping class `#3', because\MessageBreak
          this class is already loaded%
        }%
      }{%
        \@ifundefined{MainClassName}{%
          \def\MainClassName{#3}%
        }{}%
        \@temptokena\expandafter{%
          \ClassList
          \ClassListEntry{#3}{#2}{#4}%
        }%
        \edef\ClassList{\the\@temptokena}%
      }%
    \fi
    \CL@org@fileswith@pti@ns{#1}[{#2}]{#3}[{#4}]%
  }%
  \let\@@fileswith@pti@ns\@fileswith@pti@ns
\else
%    \end{macrocode}
%    Called after \cs{documentclass}.
%    \begin{macrocode}
  \PackageInfo{classlist}{Use \string\@filelist\space method}%

  \let\ClassListEntry\relax
  \expandafter\def\expandafter\CL@test
      \expandafter#\expandafter1\@clsextension#2\@nil{%
    \ifx\\#2\\%
%    \end{macrocode}
%    Name does not contain \cs{@clsextension}
%    \begin{macrocode}
    \else
      \expandafter\CL@test@i\CL@entry\@nil
    \fi
  }%
  \expandafter\def\expandafter\CL@test@i
      \expandafter#\expandafter1\@clsextension#2\@nil{%
    \ifx\\#2\\%
      \@ifundefined{opt@\CL@entry}{%
      }{%
        \@ifundefined{MainClassName}{%
          \let\MainClassName\CL@entry
        }{%
        }%
        \edef\ClassList{%
          \ClassList
          \ClassListEntry{\CL@entry}{}{}%
        }%
      }%
    \else
%    \end{macrocode}
%    Names with more than one \cs{@clsextension} are not supported.
%    \begin{macrocode}
    \fi
  }%
  \@for\CL@entry:=\@filelist\do{%
    \expandafter\expandafter\expandafter\CL@test\expandafter
        \CL@entry\@clsextension\@nil
  }%
\fi
%    \end{macrocode}
%
%    \begin{macro}{\PrintClassListEntry}
%    \begin{macrocode}
\providecommand*{\PrintClassListEntry}[3]{%
  \toks@{* #1}%
  \typeout{\the\toks@}%
}
%    \end{macrocode}
%    \end{macro}
%    \begin{macro}{\PrintClassListTitle}
%    \begin{macrocode}
\providecommand*{\PrintClassListTitle}{%
  \typeout{Class list:}%
}
%    \end{macrocode}
%    \end{macro}
%    \begin{macro}{\PrintClassList}
%    \begin{macrocode}
\providecommand*{\PrintClassList}{%
  \begingroup
    \let\ClassListEntry\PrintClassListEntry
    \PrintClassListTitle
    \ClassList
  \endgroup
}
%    \end{macrocode}
%    \end{macro}
%    \begin{macro}{\CL@InfoEntry}
%    \begin{macrocode}
\def\CL@InfoEntry#1#2#3{%
  \advance\count@ by \@ne
  \def\x{#2}%
  \@onelevel@sanitize\x
  \edef\CL@Info{%
    \CL@Info
    \noexpand\MessageBreak
    (\the\count@) %
    #1 [\x]%
    \ifx\\#3\\%
    \else
      \space[#3]% hash-ok
    \fi
  }%
}
%    \end{macrocode}
%    \end{macro}
%    \begin{macrocode}
\AtBeginDocument{%
  \begingroup
    \count@=\z@
    \def\CL@Info{Class List:}%
    \let\ClassListEntry\CL@InfoEntry
    \ClassList
    \let\on@line\@empty
    \PackageInfo{classlist}{\CL@Info}%
  \endgroup
}
%    \end{macrocode}
%
%    \begin{macrocode}
%</package>
%    \end{macrocode}
%
% \section{Installation}
%
% \subsection{Download}
%
% \paragraph{Package.} This package is available on
% CTAN\footnote{\CTANpkg{classlist}}:
% \begin{description}
% \item[\CTAN{macros/latex/contrib/oberdiek/classlist.dtx}] The source file.
% \item[\CTAN{macros/latex/contrib/oberdiek/classlist.pdf}] Documentation.
% \end{description}
%
%
% \paragraph{Bundle.} All the packages of the bundle `oberdiek'
% are also available in a TDS compliant ZIP archive. There
% the packages are already unpacked and the documentation files
% are generated. The files and directories obey the TDS standard.
% \begin{description}
% \item[\CTANinstall{install/macros/latex/contrib/oberdiek.tds.zip}]
% \end{description}
% \emph{TDS} refers to the standard ``A Directory Structure
% for \TeX\ Files'' (\CTANpkg{tds}). Directories
% with \xfile{texmf} in their name are usually organized this way.
%
% \subsection{Bundle installation}
%
% \paragraph{Unpacking.} Unpack the \xfile{oberdiek.tds.zip} in the
% TDS tree (also known as \xfile{texmf} tree) of your choice.
% Example (linux):
% \begin{quote}
%   |unzip oberdiek.tds.zip -d ~/texmf|
% \end{quote}
%
% \subsection{Package installation}
%
% \paragraph{Unpacking.} The \xfile{.dtx} file is a self-extracting
% \docstrip\ archive. The files are extracted by running the
% \xfile{.dtx} through \plainTeX:
% \begin{quote}
%   \verb|tex classlist.dtx|
% \end{quote}
%
% \paragraph{TDS.} Now the different files must be moved into
% the different directories in your installation TDS tree
% (also known as \xfile{texmf} tree):
% \begin{quote}
% \def\t{^^A
% \begin{tabular}{@{}>{\ttfamily}l@{ $\rightarrow$ }>{\ttfamily}l@{}}
%   classlist.sty & tex/latex/oberdiek/classlist.sty\\
%   classlist.pdf & doc/latex/oberdiek/classlist.pdf\\
%   classlist.dtx & source/latex/oberdiek/classlist.dtx\\
% \end{tabular}^^A
% }^^A
% \sbox0{\t}^^A
% \ifdim\wd0>\linewidth
%   \begingroup
%     \advance\linewidth by\leftmargin
%     \advance\linewidth by\rightmargin
%   \edef\x{\endgroup
%     \def\noexpand\lw{\the\linewidth}^^A
%   }\x
%   \def\lwbox{^^A
%     \leavevmode
%     \hbox to \linewidth{^^A
%       \kern-\leftmargin\relax
%       \hss
%       \usebox0
%       \hss
%       \kern-\rightmargin\relax
%     }^^A
%   }^^A
%   \ifdim\wd0>\lw
%     \sbox0{\small\t}^^A
%     \ifdim\wd0>\linewidth
%       \ifdim\wd0>\lw
%         \sbox0{\footnotesize\t}^^A
%         \ifdim\wd0>\linewidth
%           \ifdim\wd0>\lw
%             \sbox0{\scriptsize\t}^^A
%             \ifdim\wd0>\linewidth
%               \ifdim\wd0>\lw
%                 \sbox0{\tiny\t}^^A
%                 \ifdim\wd0>\linewidth
%                   \lwbox
%                 \else
%                   \usebox0
%                 \fi
%               \else
%                 \lwbox
%               \fi
%             \else
%               \usebox0
%             \fi
%           \else
%             \lwbox
%           \fi
%         \else
%           \usebox0
%         \fi
%       \else
%         \lwbox
%       \fi
%     \else
%       \usebox0
%     \fi
%   \else
%     \lwbox
%   \fi
% \else
%   \usebox0
% \fi
% \end{quote}
% If you have a \xfile{docstrip.cfg} that configures and enables \docstrip's
% TDS installing feature, then some files can already be in the right
% place, see the documentation of \docstrip.
%
% \subsection{Refresh file name databases}
%
% If your \TeX~distribution
% (\TeX\,Live, \mikTeX, \dots) relies on file name databases, you must refresh
% these. For example, \TeX\,Live\ users run \verb|texhash| or
% \verb|mktexlsr|.
%
% \subsection{Some details for the interested}
%
% \paragraph{Unpacking with \LaTeX.}
% The \xfile{.dtx} chooses its action depending on the format:
% \begin{description}
% \item[\plainTeX:] Run \docstrip\ and extract the files.
% \item[\LaTeX:] Generate the documentation.
% \end{description}
% If you insist on using \LaTeX\ for \docstrip\ (really,
% \docstrip\ does not need \LaTeX), then inform the autodetect routine
% about your intention:
% \begin{quote}
%   \verb|latex \let\install=y% \iffalse meta-comment
%
% File: classlist.dtx
% Version: 2016/05/16 v1.5
% Info: Record classes used in a document
%
% Copyright (C)
%    2005, 2006, 2008, 2011 Heiko Oberdiek
%    2016-2019 Oberdiek Package Support Group
%    https://github.com/ho-tex/oberdiek/issues
%
% This work may be distributed and/or modified under the
% conditions of the LaTeX Project Public License, either
% version 1.3c of this license or (at your option) any later
% version. This version of this license is in
%    https://www.latex-project.org/lppl/lppl-1-3c.txt
% and the latest version of this license is in
%    https://www.latex-project.org/lppl.txt
% and version 1.3 or later is part of all distributions of
% LaTeX version 2005/12/01 or later.
%
% This work has the LPPL maintenance status "maintained".
%
% The Current Maintainers of this work are
% Heiko Oberdiek and the Oberdiek Package Support Group
% https://github.com/ho-tex/oberdiek/issues
%
% This work consists of the main source file classlist.dtx
% and the derived files
%    classlist.sty, classlist.pdf, classlist.ins, classlist.drv.
%
% Distribution:
%    CTAN:macros/latex/contrib/oberdiek/classlist.dtx
%    CTAN:macros/latex/contrib/oberdiek/classlist.pdf
%
% Unpacking:
%    (a) If classlist.ins is present:
%           tex classlist.ins
%    (b) Without classlist.ins:
%           tex classlist.dtx
%    (c) If you insist on using LaTeX
%           latex \let\install=y% \iffalse meta-comment
%
% File: classlist.dtx
% Version: 2016/05/16 v1.5
% Info: Record classes used in a document
%
% Copyright (C)
%    2005, 2006, 2008, 2011 Heiko Oberdiek
%    2016-2019 Oberdiek Package Support Group
%    https://github.com/ho-tex/oberdiek/issues
%
% This work may be distributed and/or modified under the
% conditions of the LaTeX Project Public License, either
% version 1.3c of this license or (at your option) any later
% version. This version of this license is in
%    https://www.latex-project.org/lppl/lppl-1-3c.txt
% and the latest version of this license is in
%    https://www.latex-project.org/lppl.txt
% and version 1.3 or later is part of all distributions of
% LaTeX version 2005/12/01 or later.
%
% This work has the LPPL maintenance status "maintained".
%
% The Current Maintainers of this work are
% Heiko Oberdiek and the Oberdiek Package Support Group
% https://github.com/ho-tex/oberdiek/issues
%
% This work consists of the main source file classlist.dtx
% and the derived files
%    classlist.sty, classlist.pdf, classlist.ins, classlist.drv.
%
% Distribution:
%    CTAN:macros/latex/contrib/oberdiek/classlist.dtx
%    CTAN:macros/latex/contrib/oberdiek/classlist.pdf
%
% Unpacking:
%    (a) If classlist.ins is present:
%           tex classlist.ins
%    (b) Without classlist.ins:
%           tex classlist.dtx
%    (c) If you insist on using LaTeX
%           latex \let\install=y\input{classlist.dtx}
%        (quote the arguments according to the demands of your shell)
%
% Documentation:
%    (a) If classlist.drv is present:
%           latex classlist.drv
%    (b) Without classlist.drv:
%           latex classlist.dtx; ...
%    The class ltxdoc loads the configuration file ltxdoc.cfg
%    if available. Here you can specify further options, e.g.
%    use A4 as paper format:
%       \PassOptionsToClass{a4paper}{article}
%
%    Program calls to get the documentation (example):
%       pdflatex classlist.dtx
%       makeindex -s gind.ist classlist.idx
%       pdflatex classlist.dtx
%       makeindex -s gind.ist classlist.idx
%       pdflatex classlist.dtx
%
% Installation:
%    TDS:tex/latex/oberdiek/classlist.sty
%    TDS:doc/latex/oberdiek/classlist.pdf
%    TDS:source/latex/oberdiek/classlist.dtx
%
%<*ignore>
\begingroup
  \catcode123=1 %
  \catcode125=2 %
  \def\x{LaTeX2e}%
\expandafter\endgroup
\ifcase 0\ifx\install y1\fi\expandafter
         \ifx\csname processbatchFile\endcsname\relax\else1\fi
         \ifx\fmtname\x\else 1\fi\relax
\else\csname fi\endcsname
%</ignore>
%<*install>
\input docstrip.tex
\Msg{************************************************************************}
\Msg{* Installation}
\Msg{* Package: classlist 2016/05/16 v1.5 Record classes used in a document (HO)}
\Msg{************************************************************************}

\keepsilent
\askforoverwritefalse

\let\MetaPrefix\relax
\preamble

This is a generated file.

Project: classlist
Version: 2016/05/16 v1.5

Copyright (C)
   2005, 2006, 2008, 2011 Heiko Oberdiek
   2016-2019 Oberdiek Package Support Group

This work may be distributed and/or modified under the
conditions of the LaTeX Project Public License, either
version 1.3c of this license or (at your option) any later
version. This version of this license is in
   https://www.latex-project.org/lppl/lppl-1-3c.txt
and the latest version of this license is in
   https://www.latex-project.org/lppl.txt
and version 1.3 or later is part of all distributions of
LaTeX version 2005/12/01 or later.

This work has the LPPL maintenance status "maintained".

The Current Maintainers of this work are
Heiko Oberdiek and the Oberdiek Package Support Group
https://github.com/ho-tex/oberdiek/issues


This work consists of the main source file classlist.dtx
and the derived files
   classlist.sty, classlist.pdf, classlist.ins, classlist.drv.

\endpreamble
\let\MetaPrefix\DoubleperCent

\generate{%
  \file{classlist.ins}{\from{classlist.dtx}{install}}%
  \file{classlist.drv}{\from{classlist.dtx}{driver}}%
  \usedir{tex/latex/oberdiek}%
  \file{classlist.sty}{\from{classlist.dtx}{package}}%
}

\catcode32=13\relax% active space
\let =\space%
\Msg{************************************************************************}
\Msg{*}
\Msg{* To finish the installation you have to move the following}
\Msg{* file into a directory searched by TeX:}
\Msg{*}
\Msg{*     classlist.sty}
\Msg{*}
\Msg{* To produce the documentation run the file `classlist.drv'}
\Msg{* through LaTeX.}
\Msg{*}
\Msg{* Happy TeXing!}
\Msg{*}
\Msg{************************************************************************}

\endbatchfile
%</install>
%<*ignore>
\fi
%</ignore>
%<*driver>
\NeedsTeXFormat{LaTeX2e}
\ProvidesFile{classlist.drv}%
  [2016/05/16 v1.5 Record classes used in a document (HO)]%
\documentclass{ltxdoc}
\usepackage{holtxdoc}[2011/11/22]
\begin{document}
  \DocInput{classlist.dtx}%
\end{document}
%</driver>
% \fi
%
%
%
% \GetFileInfo{classlist.drv}
%
% \title{The \xpackage{classlist} package}
% \date{2016/05/16 v1.5}
% \author{Heiko Oberdiek\thanks
% {Please report any issues at \url{https://github.com/ho-tex/oberdiek/issues}}}
%
% \maketitle
%
% \begin{abstract}
% This package records the loaded classes and stores
% them in a list.
% \end{abstract}
%
% \tableofcontents
%
% \section{Documentation}
%
% \subsection{Background}
%
% This packages is an answer of a newsgroup question:
% \begin{quote}
% \begin{tabular}{@{}ll@{}}
%   Newsgroup: & comp.text.tex\\
%   Subject: & Finding the Document Class\\
%   From: & Herber Schulz\\
%   Date: & 18 Jun 2005 13:16:49 -0500\\
%   Message-ID: &
%    \textless
%    \texttt{herbs-D55DB9.13170418062005@news.isp.giganews.com}^^A
%    \textgreater
% \end{tabular}
% \end{quote}
%
% \subsection{Usage}
%
% Load this package before \cs{documentclass}:
% \begin{quote}
%   |\RequirePackage{classlist}|\\
%   |\documentclass[some,options]{whatever}|
% \end{quote}
% It then records the classes with options.
%
% If used after \cs{documentclass}, \cs{@filelist} is
% parsed for classes. The additional data
% specified options and requested version is no
% longer available here.
%
% \begin{description}
% \item[\cs{MainClassName}] contains the first loaded class.
% \item[\cs{ClassList}] stores the class entries, eg.
%   \begin{quote}
%   \begin{tabular}{@{}l@{ }l@{}}
%     \cs{ClassList} $\rightarrow$&
%     |\ClassListEntry{myarticle}{a4paper}{}|\\
%     &|\ClassListEntry{article}{}{}|
%   \end{tabular}
%   \end{quote}
% \item[\cs{ClassListEntry}] has three arguments:
%   \begin{quote}
%   \begin{tabular}{@{}ll@{}}
%     |#1|: & class name\\
%     |#2|: & options given in \cs{documentclass}/\cs{LoadClass}\\
%     |#3|: & requested version, not the version of class
%   \end{tabular}
%   \end{quote}
% \item[\cs{PrintClassList}] prints the list on screen it can be
%    configured by
% \item[\cs{PrintClassListTitle}] for the title and
% \item[\cs{PrintClassListEntry}] for formatting the entries.
%    See the implementation for how to use these.
% \end{description}
%
% \StopEventually{
% }
%
% \section{Implementation}
%
%    \begin{macrocode}
%<*package>
%    \end{macrocode}
%    Package identification.
%    \begin{macrocode}
\NeedsTeXFormat{LaTeX2e}
\ProvidesPackage{classlist}%
  [2016/05/16 v1.5 Record classes used in a document (HO)]
%    \end{macrocode}
%
%    \begin{macrocode}
\let\ClassList\@empty
\let\MainClassName\relax
%    \end{macrocode}
%
%    Test, whether we are called before \cs{documentclass}.
%    \begin{macrocode}
\ifx\@classoptionslist\relax
  \let\CL@org@fileswith@pti@ns\@fileswith@pti@ns
  \def\@fileswith@pti@ns#1[#2]#3[#4]{%
%    \end{macrocode}
%    \begin{tabular}{@{}ll@{}}
%      |#1|: & \cs{@clsextension}\\
%      |#2|: & options of \cs{documentclass}/\cs{LoadClass}\\
%      |#3|: & class name\\
%      |#4|: & requested version
%    \end{tabular}
%    \begin{macrocode}
    \ifx#1\@clsextension
      \@ifl@aded#1{#3}{%
        \PackageInfo{classlist}{%
          Skipping class `#3', because\MessageBreak
          this class is already loaded%
        }%
      }{%
        \@ifundefined{MainClassName}{%
          \def\MainClassName{#3}%
        }{}%
        \@temptokena\expandafter{%
          \ClassList
          \ClassListEntry{#3}{#2}{#4}%
        }%
        \edef\ClassList{\the\@temptokena}%
      }%
    \fi
    \CL@org@fileswith@pti@ns{#1}[{#2}]{#3}[{#4}]%
  }%
  \let\@@fileswith@pti@ns\@fileswith@pti@ns
\else
%    \end{macrocode}
%    Called after \cs{documentclass}.
%    \begin{macrocode}
  \PackageInfo{classlist}{Use \string\@filelist\space method}%

  \let\ClassListEntry\relax
  \expandafter\def\expandafter\CL@test
      \expandafter#\expandafter1\@clsextension#2\@nil{%
    \ifx\\#2\\%
%    \end{macrocode}
%    Name does not contain \cs{@clsextension}
%    \begin{macrocode}
    \else
      \expandafter\CL@test@i\CL@entry\@nil
    \fi
  }%
  \expandafter\def\expandafter\CL@test@i
      \expandafter#\expandafter1\@clsextension#2\@nil{%
    \ifx\\#2\\%
      \@ifundefined{opt@\CL@entry}{%
      }{%
        \@ifundefined{MainClassName}{%
          \let\MainClassName\CL@entry
        }{%
        }%
        \edef\ClassList{%
          \ClassList
          \ClassListEntry{\CL@entry}{}{}%
        }%
      }%
    \else
%    \end{macrocode}
%    Names with more than one \cs{@clsextension} are not supported.
%    \begin{macrocode}
    \fi
  }%
  \@for\CL@entry:=\@filelist\do{%
    \expandafter\expandafter\expandafter\CL@test\expandafter
        \CL@entry\@clsextension\@nil
  }%
\fi
%    \end{macrocode}
%
%    \begin{macro}{\PrintClassListEntry}
%    \begin{macrocode}
\providecommand*{\PrintClassListEntry}[3]{%
  \toks@{* #1}%
  \typeout{\the\toks@}%
}
%    \end{macrocode}
%    \end{macro}
%    \begin{macro}{\PrintClassListTitle}
%    \begin{macrocode}
\providecommand*{\PrintClassListTitle}{%
  \typeout{Class list:}%
}
%    \end{macrocode}
%    \end{macro}
%    \begin{macro}{\PrintClassList}
%    \begin{macrocode}
\providecommand*{\PrintClassList}{%
  \begingroup
    \let\ClassListEntry\PrintClassListEntry
    \PrintClassListTitle
    \ClassList
  \endgroup
}
%    \end{macrocode}
%    \end{macro}
%    \begin{macro}{\CL@InfoEntry}
%    \begin{macrocode}
\def\CL@InfoEntry#1#2#3{%
  \advance\count@ by \@ne
  \def\x{#2}%
  \@onelevel@sanitize\x
  \edef\CL@Info{%
    \CL@Info
    \noexpand\MessageBreak
    (\the\count@) %
    #1 [\x]%
    \ifx\\#3\\%
    \else
      \space[#3]% hash-ok
    \fi
  }%
}
%    \end{macrocode}
%    \end{macro}
%    \begin{macrocode}
\AtBeginDocument{%
  \begingroup
    \count@=\z@
    \def\CL@Info{Class List:}%
    \let\ClassListEntry\CL@InfoEntry
    \ClassList
    \let\on@line\@empty
    \PackageInfo{classlist}{\CL@Info}%
  \endgroup
}
%    \end{macrocode}
%
%    \begin{macrocode}
%</package>
%    \end{macrocode}
%
% \section{Installation}
%
% \subsection{Download}
%
% \paragraph{Package.} This package is available on
% CTAN\footnote{\CTANpkg{classlist}}:
% \begin{description}
% \item[\CTAN{macros/latex/contrib/oberdiek/classlist.dtx}] The source file.
% \item[\CTAN{macros/latex/contrib/oberdiek/classlist.pdf}] Documentation.
% \end{description}
%
%
% \paragraph{Bundle.} All the packages of the bundle `oberdiek'
% are also available in a TDS compliant ZIP archive. There
% the packages are already unpacked and the documentation files
% are generated. The files and directories obey the TDS standard.
% \begin{description}
% \item[\CTANinstall{install/macros/latex/contrib/oberdiek.tds.zip}]
% \end{description}
% \emph{TDS} refers to the standard ``A Directory Structure
% for \TeX\ Files'' (\CTANpkg{tds}). Directories
% with \xfile{texmf} in their name are usually organized this way.
%
% \subsection{Bundle installation}
%
% \paragraph{Unpacking.} Unpack the \xfile{oberdiek.tds.zip} in the
% TDS tree (also known as \xfile{texmf} tree) of your choice.
% Example (linux):
% \begin{quote}
%   |unzip oberdiek.tds.zip -d ~/texmf|
% \end{quote}
%
% \subsection{Package installation}
%
% \paragraph{Unpacking.} The \xfile{.dtx} file is a self-extracting
% \docstrip\ archive. The files are extracted by running the
% \xfile{.dtx} through \plainTeX:
% \begin{quote}
%   \verb|tex classlist.dtx|
% \end{quote}
%
% \paragraph{TDS.} Now the different files must be moved into
% the different directories in your installation TDS tree
% (also known as \xfile{texmf} tree):
% \begin{quote}
% \def\t{^^A
% \begin{tabular}{@{}>{\ttfamily}l@{ $\rightarrow$ }>{\ttfamily}l@{}}
%   classlist.sty & tex/latex/oberdiek/classlist.sty\\
%   classlist.pdf & doc/latex/oberdiek/classlist.pdf\\
%   classlist.dtx & source/latex/oberdiek/classlist.dtx\\
% \end{tabular}^^A
% }^^A
% \sbox0{\t}^^A
% \ifdim\wd0>\linewidth
%   \begingroup
%     \advance\linewidth by\leftmargin
%     \advance\linewidth by\rightmargin
%   \edef\x{\endgroup
%     \def\noexpand\lw{\the\linewidth}^^A
%   }\x
%   \def\lwbox{^^A
%     \leavevmode
%     \hbox to \linewidth{^^A
%       \kern-\leftmargin\relax
%       \hss
%       \usebox0
%       \hss
%       \kern-\rightmargin\relax
%     }^^A
%   }^^A
%   \ifdim\wd0>\lw
%     \sbox0{\small\t}^^A
%     \ifdim\wd0>\linewidth
%       \ifdim\wd0>\lw
%         \sbox0{\footnotesize\t}^^A
%         \ifdim\wd0>\linewidth
%           \ifdim\wd0>\lw
%             \sbox0{\scriptsize\t}^^A
%             \ifdim\wd0>\linewidth
%               \ifdim\wd0>\lw
%                 \sbox0{\tiny\t}^^A
%                 \ifdim\wd0>\linewidth
%                   \lwbox
%                 \else
%                   \usebox0
%                 \fi
%               \else
%                 \lwbox
%               \fi
%             \else
%               \usebox0
%             \fi
%           \else
%             \lwbox
%           \fi
%         \else
%           \usebox0
%         \fi
%       \else
%         \lwbox
%       \fi
%     \else
%       \usebox0
%     \fi
%   \else
%     \lwbox
%   \fi
% \else
%   \usebox0
% \fi
% \end{quote}
% If you have a \xfile{docstrip.cfg} that configures and enables \docstrip's
% TDS installing feature, then some files can already be in the right
% place, see the documentation of \docstrip.
%
% \subsection{Refresh file name databases}
%
% If your \TeX~distribution
% (\TeX\,Live, \mikTeX, \dots) relies on file name databases, you must refresh
% these. For example, \TeX\,Live\ users run \verb|texhash| or
% \verb|mktexlsr|.
%
% \subsection{Some details for the interested}
%
% \paragraph{Unpacking with \LaTeX.}
% The \xfile{.dtx} chooses its action depending on the format:
% \begin{description}
% \item[\plainTeX:] Run \docstrip\ and extract the files.
% \item[\LaTeX:] Generate the documentation.
% \end{description}
% If you insist on using \LaTeX\ for \docstrip\ (really,
% \docstrip\ does not need \LaTeX), then inform the autodetect routine
% about your intention:
% \begin{quote}
%   \verb|latex \let\install=y\input{classlist.dtx}|
% \end{quote}
% Do not forget to quote the argument according to the demands
% of your shell.
%
% \paragraph{Generating the documentation.}
% You can use both the \xfile{.dtx} or the \xfile{.drv} to generate
% the documentation. The process can be configured by the
% configuration file \xfile{ltxdoc.cfg}. For instance, put this
% line into this file, if you want to have A4 as paper format:
% \begin{quote}
%   \verb|\PassOptionsToClass{a4paper}{article}|
% \end{quote}
% An example follows how to generate the
% documentation with pdf\LaTeX:
% \begin{quote}
%\begin{verbatim}
%pdflatex classlist.dtx
%makeindex -s gind.ist classlist.idx
%pdflatex classlist.dtx
%makeindex -s gind.ist classlist.idx
%pdflatex classlist.dtx
%\end{verbatim}
% \end{quote}
%
% \begin{History}
%   \begin{Version}{2005/06/19 v1.0}
%   \item
%     First published version: CTAN and newsgroup \xnewsgroup{comp.text.tex}:
%     \URL{``\link{Re: Finding the Document Class}''}^^A
%     {https://groups.google.com/group/comp.text.tex/msg/8ee9523c2dc13666}
%   \end{Version}
%   \begin{Version}{2005/06/19 v1.1}
%   \item
%     After \cs{documentclass} the package looks
%     at \cs{@filelist} instead of aborting with error.
%   \end{Version}
%   \begin{Version}{2006/02/20 v1.2}
%   \item
%     DTX framework.
%   \item
%     Fix for \cs{@@fileswith@pti@ns}.
%   \end{Version}
%   \begin{Version}{2008/08/11 v1.3}
%   \item
%     Code is not changed.
%   \item
%     URLs updated.
%   \end{Version}
%   \begin{Version}{2011/10/17 v1.4}
%   \item
%     Documentation fix: \cs{MainClass} $\rightarrow$ \cs{MainClassName}.
%   \end{Version}
%   \begin{Version}{2016/05/16 v1.5}
%   \item
%     Documentation updates.
%   \end{Version}
% \end{History}
%
% \PrintIndex
%
% \Finale
\endinput

%        (quote the arguments according to the demands of your shell)
%
% Documentation:
%    (a) If classlist.drv is present:
%           latex classlist.drv
%    (b) Without classlist.drv:
%           latex classlist.dtx; ...
%    The class ltxdoc loads the configuration file ltxdoc.cfg
%    if available. Here you can specify further options, e.g.
%    use A4 as paper format:
%       \PassOptionsToClass{a4paper}{article}
%
%    Program calls to get the documentation (example):
%       pdflatex classlist.dtx
%       makeindex -s gind.ist classlist.idx
%       pdflatex classlist.dtx
%       makeindex -s gind.ist classlist.idx
%       pdflatex classlist.dtx
%
% Installation:
%    TDS:tex/latex/oberdiek/classlist.sty
%    TDS:doc/latex/oberdiek/classlist.pdf
%    TDS:source/latex/oberdiek/classlist.dtx
%
%<*ignore>
\begingroup
  \catcode123=1 %
  \catcode125=2 %
  \def\x{LaTeX2e}%
\expandafter\endgroup
\ifcase 0\ifx\install y1\fi\expandafter
         \ifx\csname processbatchFile\endcsname\relax\else1\fi
         \ifx\fmtname\x\else 1\fi\relax
\else\csname fi\endcsname
%</ignore>
%<*install>
\input docstrip.tex
\Msg{************************************************************************}
\Msg{* Installation}
\Msg{* Package: classlist 2016/05/16 v1.5 Record classes used in a document (HO)}
\Msg{************************************************************************}

\keepsilent
\askforoverwritefalse

\let\MetaPrefix\relax
\preamble

This is a generated file.

Project: classlist
Version: 2016/05/16 v1.5

Copyright (C)
   2005, 2006, 2008, 2011 Heiko Oberdiek
   2016-2019 Oberdiek Package Support Group

This work may be distributed and/or modified under the
conditions of the LaTeX Project Public License, either
version 1.3c of this license or (at your option) any later
version. This version of this license is in
   https://www.latex-project.org/lppl/lppl-1-3c.txt
and the latest version of this license is in
   https://www.latex-project.org/lppl.txt
and version 1.3 or later is part of all distributions of
LaTeX version 2005/12/01 or later.

This work has the LPPL maintenance status "maintained".

The Current Maintainers of this work are
Heiko Oberdiek and the Oberdiek Package Support Group
https://github.com/ho-tex/oberdiek/issues


This work consists of the main source file classlist.dtx
and the derived files
   classlist.sty, classlist.pdf, classlist.ins, classlist.drv.

\endpreamble
\let\MetaPrefix\DoubleperCent

\generate{%
  \file{classlist.ins}{\from{classlist.dtx}{install}}%
  \file{classlist.drv}{\from{classlist.dtx}{driver}}%
  \usedir{tex/latex/oberdiek}%
  \file{classlist.sty}{\from{classlist.dtx}{package}}%
}

\catcode32=13\relax% active space
\let =\space%
\Msg{************************************************************************}
\Msg{*}
\Msg{* To finish the installation you have to move the following}
\Msg{* file into a directory searched by TeX:}
\Msg{*}
\Msg{*     classlist.sty}
\Msg{*}
\Msg{* To produce the documentation run the file `classlist.drv'}
\Msg{* through LaTeX.}
\Msg{*}
\Msg{* Happy TeXing!}
\Msg{*}
\Msg{************************************************************************}

\endbatchfile
%</install>
%<*ignore>
\fi
%</ignore>
%<*driver>
\NeedsTeXFormat{LaTeX2e}
\ProvidesFile{classlist.drv}%
  [2016/05/16 v1.5 Record classes used in a document (HO)]%
\documentclass{ltxdoc}
\usepackage{holtxdoc}[2011/11/22]
\begin{document}
  \DocInput{classlist.dtx}%
\end{document}
%</driver>
% \fi
%
%
%
% \GetFileInfo{classlist.drv}
%
% \title{The \xpackage{classlist} package}
% \date{2016/05/16 v1.5}
% \author{Heiko Oberdiek\thanks
% {Please report any issues at \url{https://github.com/ho-tex/oberdiek/issues}}}
%
% \maketitle
%
% \begin{abstract}
% This package records the loaded classes and stores
% them in a list.
% \end{abstract}
%
% \tableofcontents
%
% \section{Documentation}
%
% \subsection{Background}
%
% This packages is an answer of a newsgroup question:
% \begin{quote}
% \begin{tabular}{@{}ll@{}}
%   Newsgroup: & comp.text.tex\\
%   Subject: & Finding the Document Class\\
%   From: & Herber Schulz\\
%   Date: & 18 Jun 2005 13:16:49 -0500\\
%   Message-ID: &
%    \textless
%    \texttt{herbs-D55DB9.13170418062005@news.isp.giganews.com}^^A
%    \textgreater
% \end{tabular}
% \end{quote}
%
% \subsection{Usage}
%
% Load this package before \cs{documentclass}:
% \begin{quote}
%   |\RequirePackage{classlist}|\\
%   |\documentclass[some,options]{whatever}|
% \end{quote}
% It then records the classes with options.
%
% If used after \cs{documentclass}, \cs{@filelist} is
% parsed for classes. The additional data
% specified options and requested version is no
% longer available here.
%
% \begin{description}
% \item[\cs{MainClassName}] contains the first loaded class.
% \item[\cs{ClassList}] stores the class entries, eg.
%   \begin{quote}
%   \begin{tabular}{@{}l@{ }l@{}}
%     \cs{ClassList} $\rightarrow$&
%     |\ClassListEntry{myarticle}{a4paper}{}|\\
%     &|\ClassListEntry{article}{}{}|
%   \end{tabular}
%   \end{quote}
% \item[\cs{ClassListEntry}] has three arguments:
%   \begin{quote}
%   \begin{tabular}{@{}ll@{}}
%     |#1|: & class name\\
%     |#2|: & options given in \cs{documentclass}/\cs{LoadClass}\\
%     |#3|: & requested version, not the version of class
%   \end{tabular}
%   \end{quote}
% \item[\cs{PrintClassList}] prints the list on screen it can be
%    configured by
% \item[\cs{PrintClassListTitle}] for the title and
% \item[\cs{PrintClassListEntry}] for formatting the entries.
%    See the implementation for how to use these.
% \end{description}
%
% \StopEventually{
% }
%
% \section{Implementation}
%
%    \begin{macrocode}
%<*package>
%    \end{macrocode}
%    Package identification.
%    \begin{macrocode}
\NeedsTeXFormat{LaTeX2e}
\ProvidesPackage{classlist}%
  [2016/05/16 v1.5 Record classes used in a document (HO)]
%    \end{macrocode}
%
%    \begin{macrocode}
\let\ClassList\@empty
\let\MainClassName\relax
%    \end{macrocode}
%
%    Test, whether we are called before \cs{documentclass}.
%    \begin{macrocode}
\ifx\@classoptionslist\relax
  \let\CL@org@fileswith@pti@ns\@fileswith@pti@ns
  \def\@fileswith@pti@ns#1[#2]#3[#4]{%
%    \end{macrocode}
%    \begin{tabular}{@{}ll@{}}
%      |#1|: & \cs{@clsextension}\\
%      |#2|: & options of \cs{documentclass}/\cs{LoadClass}\\
%      |#3|: & class name\\
%      |#4|: & requested version
%    \end{tabular}
%    \begin{macrocode}
    \ifx#1\@clsextension
      \@ifl@aded#1{#3}{%
        \PackageInfo{classlist}{%
          Skipping class `#3', because\MessageBreak
          this class is already loaded%
        }%
      }{%
        \@ifundefined{MainClassName}{%
          \def\MainClassName{#3}%
        }{}%
        \@temptokena\expandafter{%
          \ClassList
          \ClassListEntry{#3}{#2}{#4}%
        }%
        \edef\ClassList{\the\@temptokena}%
      }%
    \fi
    \CL@org@fileswith@pti@ns{#1}[{#2}]{#3}[{#4}]%
  }%
  \let\@@fileswith@pti@ns\@fileswith@pti@ns
\else
%    \end{macrocode}
%    Called after \cs{documentclass}.
%    \begin{macrocode}
  \PackageInfo{classlist}{Use \string\@filelist\space method}%

  \let\ClassListEntry\relax
  \expandafter\def\expandafter\CL@test
      \expandafter#\expandafter1\@clsextension#2\@nil{%
    \ifx\\#2\\%
%    \end{macrocode}
%    Name does not contain \cs{@clsextension}
%    \begin{macrocode}
    \else
      \expandafter\CL@test@i\CL@entry\@nil
    \fi
  }%
  \expandafter\def\expandafter\CL@test@i
      \expandafter#\expandafter1\@clsextension#2\@nil{%
    \ifx\\#2\\%
      \@ifundefined{opt@\CL@entry}{%
      }{%
        \@ifundefined{MainClassName}{%
          \let\MainClassName\CL@entry
        }{%
        }%
        \edef\ClassList{%
          \ClassList
          \ClassListEntry{\CL@entry}{}{}%
        }%
      }%
    \else
%    \end{macrocode}
%    Names with more than one \cs{@clsextension} are not supported.
%    \begin{macrocode}
    \fi
  }%
  \@for\CL@entry:=\@filelist\do{%
    \expandafter\expandafter\expandafter\CL@test\expandafter
        \CL@entry\@clsextension\@nil
  }%
\fi
%    \end{macrocode}
%
%    \begin{macro}{\PrintClassListEntry}
%    \begin{macrocode}
\providecommand*{\PrintClassListEntry}[3]{%
  \toks@{* #1}%
  \typeout{\the\toks@}%
}
%    \end{macrocode}
%    \end{macro}
%    \begin{macro}{\PrintClassListTitle}
%    \begin{macrocode}
\providecommand*{\PrintClassListTitle}{%
  \typeout{Class list:}%
}
%    \end{macrocode}
%    \end{macro}
%    \begin{macro}{\PrintClassList}
%    \begin{macrocode}
\providecommand*{\PrintClassList}{%
  \begingroup
    \let\ClassListEntry\PrintClassListEntry
    \PrintClassListTitle
    \ClassList
  \endgroup
}
%    \end{macrocode}
%    \end{macro}
%    \begin{macro}{\CL@InfoEntry}
%    \begin{macrocode}
\def\CL@InfoEntry#1#2#3{%
  \advance\count@ by \@ne
  \def\x{#2}%
  \@onelevel@sanitize\x
  \edef\CL@Info{%
    \CL@Info
    \noexpand\MessageBreak
    (\the\count@) %
    #1 [\x]%
    \ifx\\#3\\%
    \else
      \space[#3]% hash-ok
    \fi
  }%
}
%    \end{macrocode}
%    \end{macro}
%    \begin{macrocode}
\AtBeginDocument{%
  \begingroup
    \count@=\z@
    \def\CL@Info{Class List:}%
    \let\ClassListEntry\CL@InfoEntry
    \ClassList
    \let\on@line\@empty
    \PackageInfo{classlist}{\CL@Info}%
  \endgroup
}
%    \end{macrocode}
%
%    \begin{macrocode}
%</package>
%    \end{macrocode}
%
% \section{Installation}
%
% \subsection{Download}
%
% \paragraph{Package.} This package is available on
% CTAN\footnote{\CTANpkg{classlist}}:
% \begin{description}
% \item[\CTAN{macros/latex/contrib/oberdiek/classlist.dtx}] The source file.
% \item[\CTAN{macros/latex/contrib/oberdiek/classlist.pdf}] Documentation.
% \end{description}
%
%
% \paragraph{Bundle.} All the packages of the bundle `oberdiek'
% are also available in a TDS compliant ZIP archive. There
% the packages are already unpacked and the documentation files
% are generated. The files and directories obey the TDS standard.
% \begin{description}
% \item[\CTANinstall{install/macros/latex/contrib/oberdiek.tds.zip}]
% \end{description}
% \emph{TDS} refers to the standard ``A Directory Structure
% for \TeX\ Files'' (\CTANpkg{tds}). Directories
% with \xfile{texmf} in their name are usually organized this way.
%
% \subsection{Bundle installation}
%
% \paragraph{Unpacking.} Unpack the \xfile{oberdiek.tds.zip} in the
% TDS tree (also known as \xfile{texmf} tree) of your choice.
% Example (linux):
% \begin{quote}
%   |unzip oberdiek.tds.zip -d ~/texmf|
% \end{quote}
%
% \subsection{Package installation}
%
% \paragraph{Unpacking.} The \xfile{.dtx} file is a self-extracting
% \docstrip\ archive. The files are extracted by running the
% \xfile{.dtx} through \plainTeX:
% \begin{quote}
%   \verb|tex classlist.dtx|
% \end{quote}
%
% \paragraph{TDS.} Now the different files must be moved into
% the different directories in your installation TDS tree
% (also known as \xfile{texmf} tree):
% \begin{quote}
% \def\t{^^A
% \begin{tabular}{@{}>{\ttfamily}l@{ $\rightarrow$ }>{\ttfamily}l@{}}
%   classlist.sty & tex/latex/oberdiek/classlist.sty\\
%   classlist.pdf & doc/latex/oberdiek/classlist.pdf\\
%   classlist.dtx & source/latex/oberdiek/classlist.dtx\\
% \end{tabular}^^A
% }^^A
% \sbox0{\t}^^A
% \ifdim\wd0>\linewidth
%   \begingroup
%     \advance\linewidth by\leftmargin
%     \advance\linewidth by\rightmargin
%   \edef\x{\endgroup
%     \def\noexpand\lw{\the\linewidth}^^A
%   }\x
%   \def\lwbox{^^A
%     \leavevmode
%     \hbox to \linewidth{^^A
%       \kern-\leftmargin\relax
%       \hss
%       \usebox0
%       \hss
%       \kern-\rightmargin\relax
%     }^^A
%   }^^A
%   \ifdim\wd0>\lw
%     \sbox0{\small\t}^^A
%     \ifdim\wd0>\linewidth
%       \ifdim\wd0>\lw
%         \sbox0{\footnotesize\t}^^A
%         \ifdim\wd0>\linewidth
%           \ifdim\wd0>\lw
%             \sbox0{\scriptsize\t}^^A
%             \ifdim\wd0>\linewidth
%               \ifdim\wd0>\lw
%                 \sbox0{\tiny\t}^^A
%                 \ifdim\wd0>\linewidth
%                   \lwbox
%                 \else
%                   \usebox0
%                 \fi
%               \else
%                 \lwbox
%               \fi
%             \else
%               \usebox0
%             \fi
%           \else
%             \lwbox
%           \fi
%         \else
%           \usebox0
%         \fi
%       \else
%         \lwbox
%       \fi
%     \else
%       \usebox0
%     \fi
%   \else
%     \lwbox
%   \fi
% \else
%   \usebox0
% \fi
% \end{quote}
% If you have a \xfile{docstrip.cfg} that configures and enables \docstrip's
% TDS installing feature, then some files can already be in the right
% place, see the documentation of \docstrip.
%
% \subsection{Refresh file name databases}
%
% If your \TeX~distribution
% (\TeX\,Live, \mikTeX, \dots) relies on file name databases, you must refresh
% these. For example, \TeX\,Live\ users run \verb|texhash| or
% \verb|mktexlsr|.
%
% \subsection{Some details for the interested}
%
% \paragraph{Unpacking with \LaTeX.}
% The \xfile{.dtx} chooses its action depending on the format:
% \begin{description}
% \item[\plainTeX:] Run \docstrip\ and extract the files.
% \item[\LaTeX:] Generate the documentation.
% \end{description}
% If you insist on using \LaTeX\ for \docstrip\ (really,
% \docstrip\ does not need \LaTeX), then inform the autodetect routine
% about your intention:
% \begin{quote}
%   \verb|latex \let\install=y% \iffalse meta-comment
%
% File: classlist.dtx
% Version: 2016/05/16 v1.5
% Info: Record classes used in a document
%
% Copyright (C)
%    2005, 2006, 2008, 2011 Heiko Oberdiek
%    2016-2019 Oberdiek Package Support Group
%    https://github.com/ho-tex/oberdiek/issues
%
% This work may be distributed and/or modified under the
% conditions of the LaTeX Project Public License, either
% version 1.3c of this license or (at your option) any later
% version. This version of this license is in
%    https://www.latex-project.org/lppl/lppl-1-3c.txt
% and the latest version of this license is in
%    https://www.latex-project.org/lppl.txt
% and version 1.3 or later is part of all distributions of
% LaTeX version 2005/12/01 or later.
%
% This work has the LPPL maintenance status "maintained".
%
% The Current Maintainers of this work are
% Heiko Oberdiek and the Oberdiek Package Support Group
% https://github.com/ho-tex/oberdiek/issues
%
% This work consists of the main source file classlist.dtx
% and the derived files
%    classlist.sty, classlist.pdf, classlist.ins, classlist.drv.
%
% Distribution:
%    CTAN:macros/latex/contrib/oberdiek/classlist.dtx
%    CTAN:macros/latex/contrib/oberdiek/classlist.pdf
%
% Unpacking:
%    (a) If classlist.ins is present:
%           tex classlist.ins
%    (b) Without classlist.ins:
%           tex classlist.dtx
%    (c) If you insist on using LaTeX
%           latex \let\install=y\input{classlist.dtx}
%        (quote the arguments according to the demands of your shell)
%
% Documentation:
%    (a) If classlist.drv is present:
%           latex classlist.drv
%    (b) Without classlist.drv:
%           latex classlist.dtx; ...
%    The class ltxdoc loads the configuration file ltxdoc.cfg
%    if available. Here you can specify further options, e.g.
%    use A4 as paper format:
%       \PassOptionsToClass{a4paper}{article}
%
%    Program calls to get the documentation (example):
%       pdflatex classlist.dtx
%       makeindex -s gind.ist classlist.idx
%       pdflatex classlist.dtx
%       makeindex -s gind.ist classlist.idx
%       pdflatex classlist.dtx
%
% Installation:
%    TDS:tex/latex/oberdiek/classlist.sty
%    TDS:doc/latex/oberdiek/classlist.pdf
%    TDS:source/latex/oberdiek/classlist.dtx
%
%<*ignore>
\begingroup
  \catcode123=1 %
  \catcode125=2 %
  \def\x{LaTeX2e}%
\expandafter\endgroup
\ifcase 0\ifx\install y1\fi\expandafter
         \ifx\csname processbatchFile\endcsname\relax\else1\fi
         \ifx\fmtname\x\else 1\fi\relax
\else\csname fi\endcsname
%</ignore>
%<*install>
\input docstrip.tex
\Msg{************************************************************************}
\Msg{* Installation}
\Msg{* Package: classlist 2016/05/16 v1.5 Record classes used in a document (HO)}
\Msg{************************************************************************}

\keepsilent
\askforoverwritefalse

\let\MetaPrefix\relax
\preamble

This is a generated file.

Project: classlist
Version: 2016/05/16 v1.5

Copyright (C)
   2005, 2006, 2008, 2011 Heiko Oberdiek
   2016-2019 Oberdiek Package Support Group

This work may be distributed and/or modified under the
conditions of the LaTeX Project Public License, either
version 1.3c of this license or (at your option) any later
version. This version of this license is in
   https://www.latex-project.org/lppl/lppl-1-3c.txt
and the latest version of this license is in
   https://www.latex-project.org/lppl.txt
and version 1.3 or later is part of all distributions of
LaTeX version 2005/12/01 or later.

This work has the LPPL maintenance status "maintained".

The Current Maintainers of this work are
Heiko Oberdiek and the Oberdiek Package Support Group
https://github.com/ho-tex/oberdiek/issues


This work consists of the main source file classlist.dtx
and the derived files
   classlist.sty, classlist.pdf, classlist.ins, classlist.drv.

\endpreamble
\let\MetaPrefix\DoubleperCent

\generate{%
  \file{classlist.ins}{\from{classlist.dtx}{install}}%
  \file{classlist.drv}{\from{classlist.dtx}{driver}}%
  \usedir{tex/latex/oberdiek}%
  \file{classlist.sty}{\from{classlist.dtx}{package}}%
}

\catcode32=13\relax% active space
\let =\space%
\Msg{************************************************************************}
\Msg{*}
\Msg{* To finish the installation you have to move the following}
\Msg{* file into a directory searched by TeX:}
\Msg{*}
\Msg{*     classlist.sty}
\Msg{*}
\Msg{* To produce the documentation run the file `classlist.drv'}
\Msg{* through LaTeX.}
\Msg{*}
\Msg{* Happy TeXing!}
\Msg{*}
\Msg{************************************************************************}

\endbatchfile
%</install>
%<*ignore>
\fi
%</ignore>
%<*driver>
\NeedsTeXFormat{LaTeX2e}
\ProvidesFile{classlist.drv}%
  [2016/05/16 v1.5 Record classes used in a document (HO)]%
\documentclass{ltxdoc}
\usepackage{holtxdoc}[2011/11/22]
\begin{document}
  \DocInput{classlist.dtx}%
\end{document}
%</driver>
% \fi
%
%
%
% \GetFileInfo{classlist.drv}
%
% \title{The \xpackage{classlist} package}
% \date{2016/05/16 v1.5}
% \author{Heiko Oberdiek\thanks
% {Please report any issues at \url{https://github.com/ho-tex/oberdiek/issues}}}
%
% \maketitle
%
% \begin{abstract}
% This package records the loaded classes and stores
% them in a list.
% \end{abstract}
%
% \tableofcontents
%
% \section{Documentation}
%
% \subsection{Background}
%
% This packages is an answer of a newsgroup question:
% \begin{quote}
% \begin{tabular}{@{}ll@{}}
%   Newsgroup: & comp.text.tex\\
%   Subject: & Finding the Document Class\\
%   From: & Herber Schulz\\
%   Date: & 18 Jun 2005 13:16:49 -0500\\
%   Message-ID: &
%    \textless
%    \texttt{herbs-D55DB9.13170418062005@news.isp.giganews.com}^^A
%    \textgreater
% \end{tabular}
% \end{quote}
%
% \subsection{Usage}
%
% Load this package before \cs{documentclass}:
% \begin{quote}
%   |\RequirePackage{classlist}|\\
%   |\documentclass[some,options]{whatever}|
% \end{quote}
% It then records the classes with options.
%
% If used after \cs{documentclass}, \cs{@filelist} is
% parsed for classes. The additional data
% specified options and requested version is no
% longer available here.
%
% \begin{description}
% \item[\cs{MainClassName}] contains the first loaded class.
% \item[\cs{ClassList}] stores the class entries, eg.
%   \begin{quote}
%   \begin{tabular}{@{}l@{ }l@{}}
%     \cs{ClassList} $\rightarrow$&
%     |\ClassListEntry{myarticle}{a4paper}{}|\\
%     &|\ClassListEntry{article}{}{}|
%   \end{tabular}
%   \end{quote}
% \item[\cs{ClassListEntry}] has three arguments:
%   \begin{quote}
%   \begin{tabular}{@{}ll@{}}
%     |#1|: & class name\\
%     |#2|: & options given in \cs{documentclass}/\cs{LoadClass}\\
%     |#3|: & requested version, not the version of class
%   \end{tabular}
%   \end{quote}
% \item[\cs{PrintClassList}] prints the list on screen it can be
%    configured by
% \item[\cs{PrintClassListTitle}] for the title and
% \item[\cs{PrintClassListEntry}] for formatting the entries.
%    See the implementation for how to use these.
% \end{description}
%
% \StopEventually{
% }
%
% \section{Implementation}
%
%    \begin{macrocode}
%<*package>
%    \end{macrocode}
%    Package identification.
%    \begin{macrocode}
\NeedsTeXFormat{LaTeX2e}
\ProvidesPackage{classlist}%
  [2016/05/16 v1.5 Record classes used in a document (HO)]
%    \end{macrocode}
%
%    \begin{macrocode}
\let\ClassList\@empty
\let\MainClassName\relax
%    \end{macrocode}
%
%    Test, whether we are called before \cs{documentclass}.
%    \begin{macrocode}
\ifx\@classoptionslist\relax
  \let\CL@org@fileswith@pti@ns\@fileswith@pti@ns
  \def\@fileswith@pti@ns#1[#2]#3[#4]{%
%    \end{macrocode}
%    \begin{tabular}{@{}ll@{}}
%      |#1|: & \cs{@clsextension}\\
%      |#2|: & options of \cs{documentclass}/\cs{LoadClass}\\
%      |#3|: & class name\\
%      |#4|: & requested version
%    \end{tabular}
%    \begin{macrocode}
    \ifx#1\@clsextension
      \@ifl@aded#1{#3}{%
        \PackageInfo{classlist}{%
          Skipping class `#3', because\MessageBreak
          this class is already loaded%
        }%
      }{%
        \@ifundefined{MainClassName}{%
          \def\MainClassName{#3}%
        }{}%
        \@temptokena\expandafter{%
          \ClassList
          \ClassListEntry{#3}{#2}{#4}%
        }%
        \edef\ClassList{\the\@temptokena}%
      }%
    \fi
    \CL@org@fileswith@pti@ns{#1}[{#2}]{#3}[{#4}]%
  }%
  \let\@@fileswith@pti@ns\@fileswith@pti@ns
\else
%    \end{macrocode}
%    Called after \cs{documentclass}.
%    \begin{macrocode}
  \PackageInfo{classlist}{Use \string\@filelist\space method}%

  \let\ClassListEntry\relax
  \expandafter\def\expandafter\CL@test
      \expandafter#\expandafter1\@clsextension#2\@nil{%
    \ifx\\#2\\%
%    \end{macrocode}
%    Name does not contain \cs{@clsextension}
%    \begin{macrocode}
    \else
      \expandafter\CL@test@i\CL@entry\@nil
    \fi
  }%
  \expandafter\def\expandafter\CL@test@i
      \expandafter#\expandafter1\@clsextension#2\@nil{%
    \ifx\\#2\\%
      \@ifundefined{opt@\CL@entry}{%
      }{%
        \@ifundefined{MainClassName}{%
          \let\MainClassName\CL@entry
        }{%
        }%
        \edef\ClassList{%
          \ClassList
          \ClassListEntry{\CL@entry}{}{}%
        }%
      }%
    \else
%    \end{macrocode}
%    Names with more than one \cs{@clsextension} are not supported.
%    \begin{macrocode}
    \fi
  }%
  \@for\CL@entry:=\@filelist\do{%
    \expandafter\expandafter\expandafter\CL@test\expandafter
        \CL@entry\@clsextension\@nil
  }%
\fi
%    \end{macrocode}
%
%    \begin{macro}{\PrintClassListEntry}
%    \begin{macrocode}
\providecommand*{\PrintClassListEntry}[3]{%
  \toks@{* #1}%
  \typeout{\the\toks@}%
}
%    \end{macrocode}
%    \end{macro}
%    \begin{macro}{\PrintClassListTitle}
%    \begin{macrocode}
\providecommand*{\PrintClassListTitle}{%
  \typeout{Class list:}%
}
%    \end{macrocode}
%    \end{macro}
%    \begin{macro}{\PrintClassList}
%    \begin{macrocode}
\providecommand*{\PrintClassList}{%
  \begingroup
    \let\ClassListEntry\PrintClassListEntry
    \PrintClassListTitle
    \ClassList
  \endgroup
}
%    \end{macrocode}
%    \end{macro}
%    \begin{macro}{\CL@InfoEntry}
%    \begin{macrocode}
\def\CL@InfoEntry#1#2#3{%
  \advance\count@ by \@ne
  \def\x{#2}%
  \@onelevel@sanitize\x
  \edef\CL@Info{%
    \CL@Info
    \noexpand\MessageBreak
    (\the\count@) %
    #1 [\x]%
    \ifx\\#3\\%
    \else
      \space[#3]% hash-ok
    \fi
  }%
}
%    \end{macrocode}
%    \end{macro}
%    \begin{macrocode}
\AtBeginDocument{%
  \begingroup
    \count@=\z@
    \def\CL@Info{Class List:}%
    \let\ClassListEntry\CL@InfoEntry
    \ClassList
    \let\on@line\@empty
    \PackageInfo{classlist}{\CL@Info}%
  \endgroup
}
%    \end{macrocode}
%
%    \begin{macrocode}
%</package>
%    \end{macrocode}
%
% \section{Installation}
%
% \subsection{Download}
%
% \paragraph{Package.} This package is available on
% CTAN\footnote{\CTANpkg{classlist}}:
% \begin{description}
% \item[\CTAN{macros/latex/contrib/oberdiek/classlist.dtx}] The source file.
% \item[\CTAN{macros/latex/contrib/oberdiek/classlist.pdf}] Documentation.
% \end{description}
%
%
% \paragraph{Bundle.} All the packages of the bundle `oberdiek'
% are also available in a TDS compliant ZIP archive. There
% the packages are already unpacked and the documentation files
% are generated. The files and directories obey the TDS standard.
% \begin{description}
% \item[\CTANinstall{install/macros/latex/contrib/oberdiek.tds.zip}]
% \end{description}
% \emph{TDS} refers to the standard ``A Directory Structure
% for \TeX\ Files'' (\CTANpkg{tds}). Directories
% with \xfile{texmf} in their name are usually organized this way.
%
% \subsection{Bundle installation}
%
% \paragraph{Unpacking.} Unpack the \xfile{oberdiek.tds.zip} in the
% TDS tree (also known as \xfile{texmf} tree) of your choice.
% Example (linux):
% \begin{quote}
%   |unzip oberdiek.tds.zip -d ~/texmf|
% \end{quote}
%
% \subsection{Package installation}
%
% \paragraph{Unpacking.} The \xfile{.dtx} file is a self-extracting
% \docstrip\ archive. The files are extracted by running the
% \xfile{.dtx} through \plainTeX:
% \begin{quote}
%   \verb|tex classlist.dtx|
% \end{quote}
%
% \paragraph{TDS.} Now the different files must be moved into
% the different directories in your installation TDS tree
% (also known as \xfile{texmf} tree):
% \begin{quote}
% \def\t{^^A
% \begin{tabular}{@{}>{\ttfamily}l@{ $\rightarrow$ }>{\ttfamily}l@{}}
%   classlist.sty & tex/latex/oberdiek/classlist.sty\\
%   classlist.pdf & doc/latex/oberdiek/classlist.pdf\\
%   classlist.dtx & source/latex/oberdiek/classlist.dtx\\
% \end{tabular}^^A
% }^^A
% \sbox0{\t}^^A
% \ifdim\wd0>\linewidth
%   \begingroup
%     \advance\linewidth by\leftmargin
%     \advance\linewidth by\rightmargin
%   \edef\x{\endgroup
%     \def\noexpand\lw{\the\linewidth}^^A
%   }\x
%   \def\lwbox{^^A
%     \leavevmode
%     \hbox to \linewidth{^^A
%       \kern-\leftmargin\relax
%       \hss
%       \usebox0
%       \hss
%       \kern-\rightmargin\relax
%     }^^A
%   }^^A
%   \ifdim\wd0>\lw
%     \sbox0{\small\t}^^A
%     \ifdim\wd0>\linewidth
%       \ifdim\wd0>\lw
%         \sbox0{\footnotesize\t}^^A
%         \ifdim\wd0>\linewidth
%           \ifdim\wd0>\lw
%             \sbox0{\scriptsize\t}^^A
%             \ifdim\wd0>\linewidth
%               \ifdim\wd0>\lw
%                 \sbox0{\tiny\t}^^A
%                 \ifdim\wd0>\linewidth
%                   \lwbox
%                 \else
%                   \usebox0
%                 \fi
%               \else
%                 \lwbox
%               \fi
%             \else
%               \usebox0
%             \fi
%           \else
%             \lwbox
%           \fi
%         \else
%           \usebox0
%         \fi
%       \else
%         \lwbox
%       \fi
%     \else
%       \usebox0
%     \fi
%   \else
%     \lwbox
%   \fi
% \else
%   \usebox0
% \fi
% \end{quote}
% If you have a \xfile{docstrip.cfg} that configures and enables \docstrip's
% TDS installing feature, then some files can already be in the right
% place, see the documentation of \docstrip.
%
% \subsection{Refresh file name databases}
%
% If your \TeX~distribution
% (\TeX\,Live, \mikTeX, \dots) relies on file name databases, you must refresh
% these. For example, \TeX\,Live\ users run \verb|texhash| or
% \verb|mktexlsr|.
%
% \subsection{Some details for the interested}
%
% \paragraph{Unpacking with \LaTeX.}
% The \xfile{.dtx} chooses its action depending on the format:
% \begin{description}
% \item[\plainTeX:] Run \docstrip\ and extract the files.
% \item[\LaTeX:] Generate the documentation.
% \end{description}
% If you insist on using \LaTeX\ for \docstrip\ (really,
% \docstrip\ does not need \LaTeX), then inform the autodetect routine
% about your intention:
% \begin{quote}
%   \verb|latex \let\install=y\input{classlist.dtx}|
% \end{quote}
% Do not forget to quote the argument according to the demands
% of your shell.
%
% \paragraph{Generating the documentation.}
% You can use both the \xfile{.dtx} or the \xfile{.drv} to generate
% the documentation. The process can be configured by the
% configuration file \xfile{ltxdoc.cfg}. For instance, put this
% line into this file, if you want to have A4 as paper format:
% \begin{quote}
%   \verb|\PassOptionsToClass{a4paper}{article}|
% \end{quote}
% An example follows how to generate the
% documentation with pdf\LaTeX:
% \begin{quote}
%\begin{verbatim}
%pdflatex classlist.dtx
%makeindex -s gind.ist classlist.idx
%pdflatex classlist.dtx
%makeindex -s gind.ist classlist.idx
%pdflatex classlist.dtx
%\end{verbatim}
% \end{quote}
%
% \begin{History}
%   \begin{Version}{2005/06/19 v1.0}
%   \item
%     First published version: CTAN and newsgroup \xnewsgroup{comp.text.tex}:
%     \URL{``\link{Re: Finding the Document Class}''}^^A
%     {https://groups.google.com/group/comp.text.tex/msg/8ee9523c2dc13666}
%   \end{Version}
%   \begin{Version}{2005/06/19 v1.1}
%   \item
%     After \cs{documentclass} the package looks
%     at \cs{@filelist} instead of aborting with error.
%   \end{Version}
%   \begin{Version}{2006/02/20 v1.2}
%   \item
%     DTX framework.
%   \item
%     Fix for \cs{@@fileswith@pti@ns}.
%   \end{Version}
%   \begin{Version}{2008/08/11 v1.3}
%   \item
%     Code is not changed.
%   \item
%     URLs updated.
%   \end{Version}
%   \begin{Version}{2011/10/17 v1.4}
%   \item
%     Documentation fix: \cs{MainClass} $\rightarrow$ \cs{MainClassName}.
%   \end{Version}
%   \begin{Version}{2016/05/16 v1.5}
%   \item
%     Documentation updates.
%   \end{Version}
% \end{History}
%
% \PrintIndex
%
% \Finale
\endinput
|
% \end{quote}
% Do not forget to quote the argument according to the demands
% of your shell.
%
% \paragraph{Generating the documentation.}
% You can use both the \xfile{.dtx} or the \xfile{.drv} to generate
% the documentation. The process can be configured by the
% configuration file \xfile{ltxdoc.cfg}. For instance, put this
% line into this file, if you want to have A4 as paper format:
% \begin{quote}
%   \verb|\PassOptionsToClass{a4paper}{article}|
% \end{quote}
% An example follows how to generate the
% documentation with pdf\LaTeX:
% \begin{quote}
%\begin{verbatim}
%pdflatex classlist.dtx
%makeindex -s gind.ist classlist.idx
%pdflatex classlist.dtx
%makeindex -s gind.ist classlist.idx
%pdflatex classlist.dtx
%\end{verbatim}
% \end{quote}
%
% \begin{History}
%   \begin{Version}{2005/06/19 v1.0}
%   \item
%     First published version: CTAN and newsgroup \xnewsgroup{comp.text.tex}:
%     \URL{``\link{Re: Finding the Document Class}''}^^A
%     {https://groups.google.com/group/comp.text.tex/msg/8ee9523c2dc13666}
%   \end{Version}
%   \begin{Version}{2005/06/19 v1.1}
%   \item
%     After \cs{documentclass} the package looks
%     at \cs{@filelist} instead of aborting with error.
%   \end{Version}
%   \begin{Version}{2006/02/20 v1.2}
%   \item
%     DTX framework.
%   \item
%     Fix for \cs{@@fileswith@pti@ns}.
%   \end{Version}
%   \begin{Version}{2008/08/11 v1.3}
%   \item
%     Code is not changed.
%   \item
%     URLs updated.
%   \end{Version}
%   \begin{Version}{2011/10/17 v1.4}
%   \item
%     Documentation fix: \cs{MainClass} $\rightarrow$ \cs{MainClassName}.
%   \end{Version}
%   \begin{Version}{2016/05/16 v1.5}
%   \item
%     Documentation updates.
%   \end{Version}
% \end{History}
%
% \PrintIndex
%
% \Finale
\endinput
|
% \end{quote}
% Do not forget to quote the argument according to the demands
% of your shell.
%
% \paragraph{Generating the documentation.}
% You can use both the \xfile{.dtx} or the \xfile{.drv} to generate
% the documentation. The process can be configured by the
% configuration file \xfile{ltxdoc.cfg}. For instance, put this
% line into this file, if you want to have A4 as paper format:
% \begin{quote}
%   \verb|\PassOptionsToClass{a4paper}{article}|
% \end{quote}
% An example follows how to generate the
% documentation with pdf\LaTeX:
% \begin{quote}
%\begin{verbatim}
%pdflatex classlist.dtx
%makeindex -s gind.ist classlist.idx
%pdflatex classlist.dtx
%makeindex -s gind.ist classlist.idx
%pdflatex classlist.dtx
%\end{verbatim}
% \end{quote}
%
% \begin{History}
%   \begin{Version}{2005/06/19 v1.0}
%   \item
%     First published version: CTAN and newsgroup \xnewsgroup{comp.text.tex}:
%     \URL{``\link{Re: Finding the Document Class}''}^^A
%     {https://groups.google.com/group/comp.text.tex/msg/8ee9523c2dc13666}
%   \end{Version}
%   \begin{Version}{2005/06/19 v1.1}
%   \item
%     After \cs{documentclass} the package looks
%     at \cs{@filelist} instead of aborting with error.
%   \end{Version}
%   \begin{Version}{2006/02/20 v1.2}
%   \item
%     DTX framework.
%   \item
%     Fix for \cs{@@fileswith@pti@ns}.
%   \end{Version}
%   \begin{Version}{2008/08/11 v1.3}
%   \item
%     Code is not changed.
%   \item
%     URLs updated.
%   \end{Version}
%   \begin{Version}{2011/10/17 v1.4}
%   \item
%     Documentation fix: \cs{MainClass} $\rightarrow$ \cs{MainClassName}.
%   \end{Version}
%   \begin{Version}{2016/05/16 v1.5}
%   \item
%     Documentation updates.
%   \end{Version}
% \end{History}
%
% \PrintIndex
%
% \Finale
\endinput
|
% \end{quote}
% Do not forget to quote the argument according to the demands
% of your shell.
%
% \paragraph{Generating the documentation.}
% You can use both the \xfile{.dtx} or the \xfile{.drv} to generate
% the documentation. The process can be configured by the
% configuration file \xfile{ltxdoc.cfg}. For instance, put this
% line into this file, if you want to have A4 as paper format:
% \begin{quote}
%   \verb|\PassOptionsToClass{a4paper}{article}|
% \end{quote}
% An example follows how to generate the
% documentation with pdf\LaTeX:
% \begin{quote}
%\begin{verbatim}
%pdflatex classlist.dtx
%makeindex -s gind.ist classlist.idx
%pdflatex classlist.dtx
%makeindex -s gind.ist classlist.idx
%pdflatex classlist.dtx
%\end{verbatim}
% \end{quote}
%
% \begin{History}
%   \begin{Version}{2005/06/19 v1.0}
%   \item
%     First published version: CTAN and newsgroup \xnewsgroup{comp.text.tex}:
%     \URL{``\link{Re: Finding the Document Class}''}^^A
%     {https://groups.google.com/group/comp.text.tex/msg/8ee9523c2dc13666}
%   \end{Version}
%   \begin{Version}{2005/06/19 v1.1}
%   \item
%     After \cs{documentclass} the package looks
%     at \cs{@filelist} instead of aborting with error.
%   \end{Version}
%   \begin{Version}{2006/02/20 v1.2}
%   \item
%     DTX framework.
%   \item
%     Fix for \cs{@@fileswith@pti@ns}.
%   \end{Version}
%   \begin{Version}{2008/08/11 v1.3}
%   \item
%     Code is not changed.
%   \item
%     URLs updated.
%   \end{Version}
%   \begin{Version}{2011/10/17 v1.4}
%   \item
%     Documentation fix: \cs{MainClass} $\rightarrow$ \cs{MainClassName}.
%   \end{Version}
%   \begin{Version}{2016/05/16 v1.5}
%   \item
%     Documentation updates.
%   \end{Version}
% \end{History}
%
% \PrintIndex
%
% \Finale
\endinput
