%%
%% $Id: 2up-doc.tex 1180 2020-05-09 10:02:00Z herbert $
\documentclass[12pt]{article}
%\usepackage[T1]{fontenc}
\usepackage{libertinus}
\usepackage{url}

\usepackage{2up}
\special{landscape}

\let\TUfv\fileversion

%% INSTRUCTIONS FOR TWO-UP PRINTING.
%% 3. Uncomment one of the booklet options, if desired.
%%
%\targetlayout{twosided}
\TwoupWrites

% \targetlayout{Booklet}  %% Booklet printing with 2-sided printer/copier.
% \targetlayout{booklet}  %% Booklet printing with 1-sided printer/copier.

\twouplegaltarget
\target{\magstepminus1}{29.6cm}{21cm}
\source{\magstep0}{21cm}{29.6cm}

\makeatletter
%% PAGE NUMBERING:
%% Adjust page numbering if using booklet layout option.
\def\next{\if@leftpage\else\stepcounter{page}\fi}
\@ifundefined{target@stop}{}{\next}

%% Special list:
\newcommand\namelistlabel[1]{\mbox{\bf #1}\hfil}
\newenvironment{namelist}[1]%
  {\begin{list}{}%
    {\let\makelabel\namelistlabel
    \settowidth{\labelwidth}{#1}
    \setlength{\leftmargin}{1.1\labelwidth}}}%
 {\end{list}}

%% VERBATIM:
%%
\begingroup
  \catcode`\{=12 \catcode`\}=12
  \catcode`\(=1 \catcode`\)=2
  \catcode`\+=0 \catcode`\\=12
  +gdef+|(\)          % \| = character \
  +gdef+{({)          % \{ = character {
  +gdef+}(})          % \} = character }
+endgroup
\def\Backslash{\protect\|}
%%
%% Define some functional font commands:
%%
\def\MainFont{\tt}      % For macro definitions.
\def\UsageFont{\tt}           % For in-line macro names.
\def\InlineFont{\tt}          % For other in-line snipets of code.
\def\MetaFont{\rm\it}            % For meta arguments.
%%
%% Short-cuts for font commands:
%%
\def\s#1{\string#1}
\def\t#1{{\InlineFont\string#1}}   % For other in-line code.
\def\m#1{{\MetaFont #1\/}}         % For just putting things in italics
\def\M#1{{\InlineFont\{\m{#1}\}}}  % Arg is in italic, enclosed in tt braces.
\let\N\t                           % For in-line macro names.
\catcode`\<=13 \def<#1>{\m{#1}}    % <meta>
\catcode`\"=13\def"{\verb"}        % Short verb
%%
%% This is for listing macro definitions in a quote-like environment.
%%
\begingroup
  \catcode`\[=1 \catcode`\]=2
  \catcode`\{=13 \catcode`\}=12
  \gdef\@MD[%
    \catcode`\{=13 \catcode`\}=12
    \def{##1}[\{[\MetaFont ##1]\}]]
\endgroup
\def\MD{%
  \quote
  \begingroup
    \@MD
    \def\end{\endgroup\end}%
    \def\\{\@centercr\s}%
    \MainFont
    \s}
\def\endMD{\endquote}
%%
%% End preamble.
\catcode`\@=12

\begin{document}

\title{Documentation for 2up.tex:\\
  Two-up printing for Generic TeX}
\author{Timothy Van Zandt\thanks{Version 1.3a prepared by Herbert Voß. The documentation was put into
  \LaTeX\ format by David Todd and Herbert Voß}\\ \url{tvz@Princeton.EDU}}
\date{Version \TUfv\\[3pt] \today}

\maketitle
\thispagestyle{empty}

\begin{abstract}
  "2up.tex"/"2up.sty" allows one to print a document two-up, with
considerable flexibility as to paper size and layout. It produces a standard
dvi file, and does not involve an additional dvi or PostScript filter. It
should work with most \TeX\ macro packages.

\end{abstract}

\clearpage
\tableofcontents
\clearpage

\section{Usage} 
Load 2up as a \LaTeX\ package with \texttt{\textbackslash usepackage}.
There is a good chance you will get the desired layout. See the rest of
this documentation for detailed information on controlling 2up.

\section{Comparison with other methods}

There are other tools for two-up printing:
\begin{description}
\item[PostScript filters]
These work well as long as you only use PostScript fonts (as opposed to the
usual Metafont fonts). Otherwise, the PostScript filter scales bitmapped
fonts, and the quality is lower than with "2up.tex".

\item[Rokicki's dvidvi]
"dvidvi" is a versatil tool that does much more than
2up printing. However, if only rearranges the pages, and does not adjust the
magnification to ensure that the pages fit properly in a two-up format. It
also adds one more step to the \TeX-dvi-printer cycle. "2up.tex", on the other
hand, combines the changes to \TeX's magnification and page layout with the
rearranging of the pages, all with an easy-to-use interface and from within
your \TeX\ document. (See Section \ref{dvidvi} on using both "2up.tex" and
"dvidvi" together.)
\end{description}

\section{A warning about fonts}

Two-up printing generally involves reducing a document, and thus using fonts
sizes that are not typically found on your system. If you are using bitmapped
fonts (standard for \TeX\ users), and you do not have and cannot generate the
extra fonts bitmaps, you will find 2up.sty (and any other method of two-up
printing) to give very poor quality output.

If you are using a program like Rokicki's "dvips" that can generate extra font
bitmaps as needed, you will find that it takes a long time to print your 2up
document the first few times, as "dvips" is busy making the extra fonts. This
problem will go away. Be thankful you have such an easy way to get extra font
bitmaps.

\section{Source and target dimensions}

Set the target and source layout:
\begin{MD}
  \source{mag}{width}{height}\\
  \target{mag}{width}{height}
\end{MD}

\begin{itemize}
\item The ``source'' layout is the layout of your document
the way it is usually printed.  The ``target'' layout is the layout when it
is printed two-up.

\item <width> and <height> are the true width and height of the paper,
when looking at the output right-side-up.

\item <mag> refers to the source and target magnification. It should be an
integer, equal to 1000 times the magnification factor. There are also some
magic magnification numbers:
\begin{description}
\item[$\setminus$magstep<n>] for a magnification of 1.2$^n$ (n=0,...,5), or

\item[$\setminus$magstepminus<n>] for a magnification of 1.2$^{-n}$
(n=0,...,5).
\end{description}
Using these magnification values will make best use of the available fonts on
your system.
\end{itemize}

"2up.tex" takes each source page, and centers in half a target page. To get
the pages to fit, the magnification of the document typically has to be
reduced. If you find that the pages are too crowded, try reducing the target
magnification. If they are too lonely, try increasing the target
magnification.

If you want to adjust the positioning of the scaled pages, try changing the
values of "\hoffset" or "\voffset" with "\advance" or \LaTeX's "\addtolength".

Don't change the magnification of your document after using the "\target"
command, as you will override the settings made by 2up.tex.

You can also specify some options:
\begin{MD}
  \targetlayout{options}
\end{MD}
where <options> is a comma separated list with no spaces. Valid
options are:
\begin{namelist}{topbottomxxx}

\item [topbottom] The two source pages are printed one on top of the other.
\item [twosided]  Odd pages are always printed on the right or the bottom.
\item [booklet]   Pages are printed like a booklet. See Section \ref{booklet}.
\item [Booklet]   Like "booklet", but for two-sided printers/copiers.
\item [dvidvi]    For arranging in two-up layout with "dvidvi". See Section
\ref{dvidvi}.

\end{namelist}

The only other changes you might want to make to your document are the
following:
\begin{itemize}
\item You have to see to it that the target output is printed with the
correct orientation (landscape or portrait). See the documentation for your
dvi driver for details.

\item "2up.tex" will inserts a line between pages if you change the following
dimensions:
\begin{namelist}{pagesepwidthxxx}
\item [$\setminus$pagesepwidth]   Thickness of line. (Default 0pt)
\item [$\setminus$pageseplength]  Length of line. (Default 6.5in)
\item [$\setminus$pagesepoffset]  Distance from bottom of page to bottom of
line, or (with "topbottom" layout) from side of page to beginning of line.
(Default 1in)
\end{namelist}

\item You can, but do not need to, use the following commands:
\begin{namelist}{twoupemptypagexxx}

\item [$\setminus$twoupemptypage] inserts a completely blank half page
in the target.

\item [$\setminus$twoupclearpage] \LaTeX\ users: ejects a complete target
page.

\item [$\setminus$twoupeject] Plain \TeX users: ejects a complete target
page.

\end{namelist}
\end{itemize}

\section{Defaults}

  The following commands set the parameters for some common sources and
targets. The definitions of these commands are also useful examples.

  For A4 paper, just replace "8.5in" and "11in" in the definitions below by
the width and height of A4 paper.

\begin{description}

\item [$\setminus$twouparticle]

This is the default for \LaTeX.
\begin{namelist}{Sourcexx}

\item [Source:] A portrait document on 8.5x11 inch paper with wide margins as
used in \LaTeX's article style, and without marginal notes.

\item [Target:] A landscape document on the same size paper, with the source
pages printed side by side.

\item [Definition:]\hspace{1pt}
\begin{verbatim}
\target{\magstepminus1}{11in}{8.5in}
\source{\magstep0}{8.5in}{11in}}
\end{verbatim}
\end{namelist}

If you use narrower side margins, or if you use marginal notes, then you
will need to use "\twoupplain" or \\ "\twouplegaltarget", defined below.

\item [$\setminus$twoupplain]

This is the default for non-\LaTeX.

It is the same as "\twouparticle", but for source documents with narrower
margins.
\begin{namelist}{Sourcexxx}

\item [Definition:]\hspace{1pt}
\begin{verbatim}
\target{\magstepminus2}{11in}{8.5in}
\source{\magstep0}{8.5in}{11in}
\end{verbatim}
\end{namelist}

\item [$\setminus$twouplegaltarget]

\begin{namelist}{Sourcexxx}
\item [Source:] A portrait document on 8.5x11 inch paper with just about any
margins and perhaps with marginal notes.

\item [Target:] A landscape document on legal size paper (8.5x14 inches),
with the source pages printed side by side.

\item [Definition:]\hspace{1pt}
\begin{verbatim}
\target{\magstepminus1}{14in}{8.5in}
\source{\magstep0}{8.5in}{11in}
\end{verbatim}
\end{namelist}

\item [$\setminus$twouplandscape]\hspace{1pt}
\begin{namelist}{Sourcexxx}

\item [Source:] A landscape document on 8.5x11 inch paper with just about any
margins.

\item [Target:] A portrait document on the same size paper, with the
source pages stacked vertically.

\item [Definition:]\hspace{1pt}
\begin{verbatim}
\target{\magstepminus2}{8.5in}{11in}
\source{\magstep0}{11in}{8.5in}
\targetlayout{topbottom}
\end{verbatim}
\end{namelist}

\end{description}


\section{True Dimensions}

\TeX\ lets you use `true' dimensions, as in
\begin{verbatim}
    \topmargin 1truein
    \setlength{\topmargin}{1truein}
\end{verbatim}
This sets the length to 1 inch
as it appears on the page, no matter what the magnification.

Once you use `true' dimensions, it is impossible to change the
magnification of your document, and hence it is almost impossible
to use 2up.tex. You will get an error like:
\begin{verbatim}
    ! Incompatible magnification (833);
      the previous value will be retained (2000).
\end{verbatim}%

The use of `true' dimensions might be hiding in macros or style files
you are using. For example, Plain \TeX's "\magnification" command uses
true dimensions; use "\mag" instead, and then set "\hsize" and
"\vsize" appropriately scaled. If you can't find the culprit `true'
dimension in your document or input files, try putting this on the
(very) first line of your document:
\begin{verbatim}
    \mag 143 \voffset 1truein \mag 1000
\end{verbatim}

You will get the error listed above wherever true dimensions are used
(and in some other places as well).


\section{Page Cross-References}

  Page cross-references (e.g., in an index or table of contents) will not be
accurate with "2up.tex", because finished pages are not immediately shipped
out. This can also cause problems with some special \LaTeX\ styles that have
multiple ".aux" files (e.g., for each chapter).

  To get better, but not perfect, page cross-references, try putting the
command
\begin{verbatim}
    \TwoupWrites
\end{verbatim}
at the beginning of your document, after inputting "2up.tex". This is good for
drafts, and may fix problems with \LaTeX\ styles that have multiple ".aux"
files. "\TwoupWrites" makes "\write" and "\read" "\immediate", if you know
what that means. It might cause problems of its own, but try it and see.

  To get accurate page cross-references:
\begin{enumerate}
\item Run the job without "2up.tex", until all cross-references
are properly resolved. Run any index utility at this point.

\item Run the job ONE MORE TIME with "2up.tex"; it will use the
correct cross-references from the previous runs. (In \LaTeX\ documents,
it is useful at this point to include the "\nofiles" command, so that
the correct cross-references are preserved.)
\end{enumerate}

\section{Booklets\label{booklet}}

Here are some special considerations for making booklets:

\begin{itemize}

\item  Use the "booklet" target layout if your printer or copier cannot print
two
sides directly. "2up.tex" prints out the ``tops'' of the pages, and then the
``bottoms'', so that you can copy or print on both sides by doing one side,
flipping the output over, and doing the other side.

\item Use "Booklet" if your printer or copier can print directly on two sides.

\item You must have enough memory and box registers to hold all the pages in
the document. A Big\TeX\ should be able to handle at least a 32-page booklet.

\item Sometimes booklet printing is part of the design of a document, rather
than an afterthought. In this case, you should set up the margins for the
target half-page rather than using "2up.tex" to scale the document. For
example, if using \LaTeX, 8.5x11 inch paper, \LaTeX's twoside style option,
and headings, try the following page parameter changes:
\begin{verbatim}
    \setlength{\oddsidemargin}{-0.2in}
    \setlength{\evensidemargin}{-0.5in}
    \setlength{\textwidth}{4.2in}
    \setlength{\textheight}{6.5in}
    \setlength{\topmargin}{-.4in}
\end{verbatim}
If you do this, "2up.tex" isn't adjusting margins or scaling the
document. You coould therefore get by without "2up.tex" by using "dvidvi".
\end{itemize}

\section{Using dvidvi\label{dvidvi}}

  With the "dvidvi" layout option, "2up.tex" takes care of adjusting the
margins, and you then use "dvidvi" to arrange the pages in a two-up
format. The advantage is that cross referencing comes out right.
The disadvantage is that it involves one more step.

Start by \TeX ing your document with "2up.tex" and the "dvidvi" layout option.
Then run the dvi file through "dvidvi". Specify the offsets for "dvidvi" at
the middle of the page. E.g., for
a side-by-side layout on 8.5x11in paper, use
\begin{verbatim}
    dvidvi '2:0,1(5.5in,0in)' foo bar
\end{verbatim}
For other paper sizes, replace "5.5in" by half the target paper width. For the
"topbottom" layout, reverse the dimensions, e.g.,
(0in,5.5in). For booklet printing, see the
the documentation for "dvidvi".


\section{Compatibility\label{compatibility}}

  These macros should work with most \TeX\ macro packages, except for
the following possible but unlikely compatibility problems:

\begin{itemize}

\item If you use a macro package other than \LaTeX\ that does unusual things
to end a document and/or that modifies the definition of "\end", you may
find that the last page is nor printed or that an extra page with headings
is inserted at the end of the document. This is fixable on a case-by-case
basis. See the redefinition of "\end" in "2up.tex".

\item "2up.tex" might not work if you meddle with \TeX's "\shipout" primitive
after loading "2up.tex".

\item If the first page of the source document is not on the first two-up
target
page (possible, for example, with two-sided printing and the "booklet"
option, but generally unlikely), "\special"'s that come at the beginning
of the document also do not get onto the first page of output. This can mess
up "\special"'s for landscape printing or
including a PostScript header file. In this case, you have to specify
these things as command line options when using your dvi driver.

\end{itemize}


\section{Changes}
\begin{namelist}{V0.9xxx}
\raggedright

\item [V0.9a] Everything redone. Too many differences to point out.

\item [V0.9b] Interface redone again. "booklet" option added.

\item [V0.9c] Pages ship out a little earlier. A few bugs fixed.

  Parameters "\pageseplength" and "\pagesepthickness" added.

\item [V0.9d] Fixed bug with "\twoupclearpage" and "\twoupeject".

\item [V0.9e] Made pages be counted (from dvi driver's viewpoint) by the
physical page number.

\item [V0.9f] Parameter "\pagesepoffset" added, and "\pagesepthickness"
changed to "\pagesepwidth".

\item [V1.0]  None. Just seemed seemed like it was time to call it 1.0

\item [V1.1] Added "Booklet" and "dvidvi" target layouts.

  "\hoffset" and "\voffset" in source document now have expected effect.

\item[V1.2] Fixed incompatibility problem with AmS-\TeX/AmS-\LaTeX.
  Added "\TwoupWrites" command.
\item[V1.3] LPPL license and updated documentation (hv).

\end{namelist}
\end{document}
%% END 2up-doc
