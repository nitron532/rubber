\documentclass[10pt]{jarticle}
%%%
\def\pdf{TF}		%  TT (pdf for dvipdfmx) <-> TF (dvi for dviout)
%%%
\usepackage{myhyper,amsmath,amssymb,amsfonts,amscd,alltt}
\if\pdf
\usepackage[nokeyin]{keyin}
\else
\usepackage{keyin}
\fi
\usepackage[dviout]{graphicx,color}
\AtBeginDvi{\special{dviout -hyper=144 -hyperoff=0 v0}}
% -hyper=144 : Blue letters without underline (hot spot)
% -hyperoff=0: HyperTeX is valid
% v0         : Minimal bars
% v0vs       : Minimal bars + Status bar
% !3         : Copy to Editor without CTRL
\if\pdf
\AtBeginDvi{\special{papersize=5.5in,8in}}
\hoffset=-1.03cm
\voffset=-3.18cm
\paperwidth=5.6in
\paperheight=8in
\else
\AtBeginDvi{\special{papersize=4.6in,7.07in}}
\hoffset=-2.3cm
\voffset=-4.45cm
\paperwidth=4.6in
\paperheight=7in
\fi
\textwidth=4.4in
\textheight=7.0in
\renewcommand\baselinestretch{0.92}
%\paraindent=0cm
\setlength{\oddsidemargin}{0cm}
\setlength{\parindent}{0cm}
\def\BK{\texttt{\symbol{'134}}}
\def\D#1{\keyin{#1}{#1}}
\def\E#1{\keyin{\string#1}{#1}}
\def\Q#1{\keyin{\string#1}{$#1$}}
\def\R#1{\keyin{\noexpand#1}{$#1$}}
\def\F#1{\special{html:<a href="dviout:`in .5c#1">}\BK#1\special{html:</a>}}
\def\G#1{\special{html:<a href="dviout:`in .5c#1=">}\BK#1\special{html:</a>}}
\def\Goto#1#2{\goto{#1}{\colorbox{green}{\tiny\sf #2}}}
% \def\gotop#1{\makebox[0mm]{\hspace{#1}\Goto{top}{Top}}}
\def\red#1{{\color{red}#1}}
\def\blue#1{{\color{blue}#1}}
\def\green#1{{\color{green}#1}}
\def\RED#1{{\it\color{red}#1}}
\def\MGNT#1{{\it\color{magenta}#1}}
\def\GREEN#1{{\it\color{green}#1}}
\def\ops#1{\{\MGNT{#1}\}}
\def\Ops#1{[\MGNT{#1}]}
\def\opt#1{\{\RED{#1}\}}
\def\Opt#1{[\RED{#1}]}
\def\ST{\hspace{-.1em}\raisebox{-0.25em}*}
\if\pdf
\def\NEXT{\phantom{a}}
\def\BACK{\phantom{a}}
\else
\def\NEXT%
{\special{html:<a href="dviout: Je">}\colorbox{cyan}{\tiny$\Rightarrow$}\special{html:</a>}}
\def\BACK%
{\special{html:<a href="dviout: Jv">}\colorbox{cyan}{\tiny$\Leftarrow$}\special{html:</a>}}
\fi
\def\GOTO#1{

\vspace*{-2.7ex}
\hfill{\Goto{#1}{#1}}

}
%
\pagestyle{empty}
%%%
\newlength{\minitwocolumn}
\newenvironment{z2col}[1][0pt]%
{\def\kaidan{\end{minipage}%
             \hspace{\columnsep}%
             \addtolength{\minitwocolumn}{-#1}%
             \begin{minipage}[t]{\minitwocolumn}}%
 \setlength{\minitwocolumn}{0.48\textwidth}%
 \addtolength{\minitwocolumn}{-0.5\columnsep}%
 \addtolength{\minitwocolumn}{#1}%
 \begin{minipage}[t]{\minitwocolumn}}%
{\end{minipage}}
%%%
\if\pdf
\usepackage[all,ps,dvips]{xy}
\else
\usepackage[all]{xy}
\fi
\xyoption{poly}
\xyoption{arc}
\xyoption{2cell}
\xyoption{knot}
\xyoption{web}
\UseAllTwocells
%%%
\newenvironment{Alltt}{%
\renewcommand{\\}{\char`\\}
\begin{alltt}%
}{%
\end{alltt}
}
%%%%%%%%%%%%%%%%%%%%%%%%%%%%%%%%%%%%%%%%%%%%%%%%%%%%%%%%%%%%%%%%%%%%%
\def\ToDef#1{\goto{#1}{\colorbox{magenta}{\tiny\sf Def}}}
\def\ToEx#1{\goto{#1}{\colorbox{magenta}{\tiny\sf Ex}}}
\def\La{\,$\leftarrow$\,}
\def\pos{\MGNT{pos}}
\def\Pos{\goto{pos}\pos}
\def\cod{\MGNT{cod}}
\def\Cod{\goto{cod}\cod}
\def\obj{\MGNT{obj}}
\def\Obj{\goto{obj}\obj}
\def\vector{\MGNT{vector}}
\def\Vector{\goto{vector}\vector}
\def\place{\MGNT{place}\/}
\def\Place{\goto{place}\place}
\def\stack{\MGNT{stack}\/}
\def\sav{\MGNT{save}}
\def\FAC{{\it\color{cyan}factor}}
\def\dr{\MGNT{direction}}
\def\dm{{\it\color{cyan}dimen}}
\def\Dm{\goto{dm}\dm}
\def\CIR{{\it\color{magenta}cir}}
\def\Cir{\goto{cir}\CIR}
\def\anchor{\MGNT{anchor}}
\def\Anchor{\goto{anchor}\anchor}
\def\DIR{{\MGNT{dir}}}
\def\Dir{\goto{dir}\DIR}
\def\TEXT{{\it\color{cyan}text}}
\def\ID{{\it\color{cyan}id}}
\def\diag{{\it\color{magenta}diag}}
\def\Diag{\goto{diag}\diag}
%%%%%%%%%%%%%%%%%%%%%%%%%%%%%%%%%%%%%%%%%%%%%%%%%%%%%%%%%%%%%%%%%%%%%%%
\begin{document}
\if\pdf
\rightline{\href{http://akagi.ms.u-tokyo.ac.jp/input9.pdf}{\colorbox{cyan}{\scriptsize\sf Help \Xy-pic}}
\ \NEXT}
\else
\rightline{\href{file:input9.dvi}{\colorbox{cyan}{\scriptsize\sf Help \Xy-pic}}
\ \NEXT}
\fi
\centerline{\Large\bf\Xy-pic}

\vspace*{-4ex}
\section{はじめに}
\AmS-\LaTeX の\texttt{CD}環境で簡単な可換図式が描けますが,
矢印の形状が限られている上,斜めの矢印が描けないなど種々の制約があります.
$\Xy$-pic はこのような制約が無く,各種の図式を描くことができます.
%簡単に可換図式を描くことができるなど,
%以下のような優れた特徴を持っています.
\begin{itemize}
\item
任意の2点間に矢印が引け,矢先,軸,矢羽を組み合わせて種々の形状が可能で,
軸の部分も自由に曲げることができる.
\item
可換図式,フローチャート,有向グラフやツリーグラフ,
カテゴリー理論のセル2つの図式,多角形や楕円などの図形や自由曲線,
格子構造の図式,結び目やリンク,などを描くための機能が備わっている.
\item
矢印や曲線などは独自のフォントによって(specialsを用いない)標準の\TeX の機能
内で動作する.一方,出力デバイスに依存するが,tpic specials や PostScriptを使
った軽く綺麗な出力にも対応し,タイプセットは(綺麗さを除いて)両者で同一で
ある.これを利用し,拡大してもギザギザが出ないPDFに変換できる.
\item
座標の原点の移動のみならず,$X$-軸,$Y$-軸の単位ベクトルを任意に指定できるので,斜交座標の採用や図形のアフィン変換が容易.
\item
\LaTeX, plain\TeX, \AmS-\TeX など,種々の\TeX の下で同様に動く.
\end{itemize}
\quad
\Xy-pic はいくつかのファイル群から成り立っており,必要なもののみ読み込むことができますが,当面は
\begin{quote}
\keyin{\usepackage[all]{xy} 0d 0a}{{\sf \BK usepackage[all]\{xy\}}}
\end{quote}
を\LaTeX の冒頭の{\sf \BK usepackage\{\}}の最後に追加しておくのがよいでしょう.

\quad
以下で扱う \LaTeX や p\LaTeX の場合,\Xy-pic は,本文の中で
\begin{quote}
\keyin{\begin{xy} 0d 0a\end{xy} 0d 0a}{{\sf \BK begin\{xy\}\\
\quad$\cdots$\\
\BK end\{xy\}}}
\end{quote}
として使うのが基本です.\\
\quad
なお,\href{ftp://akagi.ms.u-tokyo.ac.jp/pub/TeX/macros/xypic.zip}{ここ}
をクリックして得たパッケージを {\tt.\BK texmf} の存在するディレクトリで展開すれば \Xy-pic  がインストールされます.
%%%%%%%%%%%%%%%%%%%%%%%%%%%%%%%%%%%%%%%%%%%%%%%%%%%%%%
\section{矢印}

\vspace*{-4ex}
\hfill\raisebox{2ex}{\BACK\ \NEXT}\\
{\bf 1}.
デフォルトの座標単位はmmで,%す.\\ \quad
$(0,0)$から$(20,0)$に矢印を引くには

\begin{z2col}
\begin{Alltt}
\\begin\{xy\}
  \\ar (20,0)
\\end\{xy\}
\end{Alltt}
\kaidan
\vspace*{3ex}
\begin{xy}
  \ar (20,0)
\end{xy}
\end{z2col}

\medskip
とします.
{\color{red}$(20,0)$} の代わりに実寸を用いて {\color{red}\verb|<2cm,0cm>|} の
ように書くこともできます(\TeX で有効な\MGNT{長さ}を用いる).

\quad 実際には「続けて{\color{red}\verb|\begin{xy} |}\verb|\ar (20,0)|{\color{red}\verb|\end{xy}|}と書く」と,
以下のようになります.

\centerline{「続けて\begin{xy} \ar (20,0)\end{xy}と書く」}

\smallskip
\quad
より短く「続けて{\color{red}\verb|\xybox{|}\verb|\ar(20,0)|{\color{red}\verb|}|}と書く」としても同じ結果です.

\medskip
{\bf 2}.
一般に

\centerline{\tt\{(\MGNT{a},\MGNT{b}) \BK ar \string@\{\MGNT{arrow}\} (\MGNT{c},\MGNT{d})\} または \{\BK ar \string@\{\MGNT{arrow}\} (\MGNT{a},\MGNT{b});(\MGNT{c},\MGNT{d})\}}

\medskip
によって,{\tt(\MGNT{a},\MGNT{b})}から{\tt(\MGNT{c},\MGNT{d})}へ 
\MGNT{arrow}で形状を指定した矢印が引かれます
(形状指定がないときは,上の標準の矢印):

\begin{z2col}
\begin{Alltt}
\\begin\{xy\}
 \{(0,20)  \\ar @\{\red{=>\}}        (15,23)\},
 \{(0,15)  \\ar @\{\red{|-->}\}      (15,15)\},
 \{(0,15)  \\ar @\{\red{~>}\}        (35,5)\},
 \{(0,5)   \\ar @\{\red{.}\}         (15,5)\},
 \{(0,0)   \\ar @\red{_}\{\red{||->>}\}    (10,0)\},
 \{(20,20) \\ar @\red{/^4mm/}      (35,20)\},
 \{(20,15) \\ar @\red{3}\{\red{<->}\}      (35,15)\},
 \{(20,0)  \\ar @\{\red{^\{(\}-_\{>\}}\} (35,0)\}
\\end\{xy\}
\end{Alltt}
\kaidan
\vspace*{5ex}
\hfill
\begin{xy}
 {(0,20)  \ar @{=>}        (15,23)},
 {(0,15)  \ar @{|-->}      (15,15)},
 {(0,15)  \ar @{~>}        (35,2)},
 {(0,5)   \ar @{.}         (15,5)},
 {(0,0)   \ar @_{||->>}    (15,0)},
 {(20,20) \ar @/^4mm/      (35,20)},
 {(20,15) \ar @3{<->}      (35,15)},
 {(20,0)  \ar @{^{(}-_{>}} (35,0)}
\end{xy}
\end{z2col}

\medskip
\quad
矢印の進む向きに対し,\red{\tt\string^} は左直交方向,
\red{\tt\string_} は右直交方向を表します(左から右へ向かう矢印に対しては,
前者は「上」,後者は「下」方向).

\quad
一般に {\color{red}\verb|/  /|} は\textbf{ずらし}を表し,
{\color{red}\verb|@/^4mm/|} は「左直交方向に4mmのずらし」となります.
矢印の始点と終点は固定されたままで,上の例では4mm上に曲がった曲線の軸を持つ\textbf{標準の形状}の矢印が引かれます.

\quad
矢先と矢羽には \red{\tt < > ( ) | ' ` + / x o \{*\}} などが使えます.

\quad
矢軸には \red{\tt - . \string~ = :} が使えます.

\quad
\red{\tt@} の直後に {\tt\color{red}\verb|^ _ 2 3|} (左/右半,二/三重)のいずれかを付加できます.

\quad
形状の\MGNT{arrow}における 上の例の{\color{red}\verb|^{(}|}は
「矢羽 \red{\tt(} の左半分」を表します.

\medskip
{\bf 3}.
矢印のフォントはComputer ModernやEulerに変更できます:
\hfill{\BACK\ \NEXT\!\!}

\begin{z2col}
\begin{Alltt}\\begin\{xy\}         \\ar (0,10);(10,10)
 \red{\\SelectTips\{cm\}\{\}} \\ar (0,5) ;(10,5)
 \red{\\SelectTips\{eu\}\{\}} \\ar (10,0)  \\end\{xy\} \end{Alltt}
\kaidan
\vspace*{-.5ex}
\hfill\hfill
\begin{xy}
 \ar (0,10);(10,10)
 \SelectTips{cm}{} \ar (0,5); (10,5)
 \SelectTips{eu}{} \ar (10,0)
\end{xy}
\hfill\phantom{.}
\end{z2col}

\smallskip
後の\red{\tt\{\ \}} の中で,ポイント(\red{10}/\red{11}/\red{12})が
指定できます.

\medskip
{\bf 4}.
矢印には,以下のように {\color{red}\verb:^ _ |:}(挿入) によって\textbf{ラベル}を付けられます:

\begin{z2col}
\begin{Alltt}
\\begin\{xy\}
 \{(0,20) \\ar @\{\red{o.x}\}(20,20)\red{\tt\string^\{}n\\to\\infty\red{\}\string_}p\},
 \{(0,15) \\ar @\red{/_/}@\{\red{\{*\}->}\} (20,10)\red{\tt\string^}A\red{\tt\string_}\\alpha\},
 \{(0,0)  \\ar @\{\red{||->>}\} (20,0){\tt\color{red}|\{}\\sum_n a_n\red{\tt\}}\}
\\end\{xy\}
\end{Alltt}
\kaidan
\vspace*{-1.5ex}
\hfill
\begin{xy}
 {(0,15) \ar @{o.x}(20,15)^{n\to\infty}_p},
 {(0,10) \ar @/_/@{{*}->} (20,7)^A_\alpha},
 {(0,0)  \ar @{||->>} (20,0)|{\sum_n a_n}}
\end{xy}\!\!\!\!
\end{z2col}
%%%%%%%%%%%%%%%%%%%%%%

\medskip
{\bf 5}.
位置やラベルに{\tt\color{red}=\{\it{\color{cyan}id}\}} によって\textbf{名前}をつけると,後から参照できます:
\begin{z2col}
\begin{Alltt}
\\begin\{xy\}
 (5,8.66)\red{="A"}, (0,0)\red{="B"}, (10,0)\red{="C"},
 \\ar@\{-\}\red{"A"};\red{"B"} \\ar@\{-\}\red{"B"};\red{"C"} \\ar@\{-\}\red{"C"};\red{"A"} 
\\end\{xy\}
\\begin\{xy\}
 \\ar @\{<-|\} (0,0);(10,0)^a\red{="A"}
 \\ar @\{|=>\}(15,0);(25,0)^b\red{="B"}
 \\ar @/^/ @\{<->\}\red{"A"};\red{"B"}^c
\\end\{xy\}
\end{Alltt}
\kaidan
\vspace*{1.ex}
\hfill
\begin{xy}
 (5,8.66)="A", (0,0)="B", (10,0)="C",
 \ar@{-}"A";"B" \ar@{-}"B";"C" \ar@{-}"C";"A" 
\end{xy}

\vspace*{8ex}
\hfill
\begin{xy}
 \ar @{<-|}(0,0);(10,0)^a="A"
 \ar @{|=>}(15,0);(25,0)^b="B"
 \ar @/^/ @{<->}"A";"B"^c
\end{xy}
\end{z2col}

\medskip
{\bf 6}.
矢(羽)の\textbf{出発方向}と,終点からみた矢先の\textbf{到着方向}とは,
この順に

\centerline{\color{red}\tt@(\MGNT{方向},\MGNT{方向})}

で指定できます.
方向には
\red{\tt r}(右),\red{\tt rd}(右下),\red{\tt d}(下),
\red{\tt ld}(左下),\red{\tt l}(左),\red{\tt lu}(左上),\
\red{\tt u}(上),\red{\tt ru}(右上)があります.

\begin{z2col}
\begin{Alltt}
\\begin\{xy\}
 (0,0)="A", (20,0)="B",
 \\ar \red{@(lu,ld)} "A";"A"|\{id\}
 \\ar \red{@(d,d)} "A";"B"^f \\ar \red{@(ld,ru)} "B";"A"_g
\\end\{xy\}
\end{Alltt}
\kaidan
\vspace*{-.1ex}
\hfill
\begin{xy}
 (0,0)="A", (20,0)="B",
 \ar @(lu,ld) "A";"A"|{id}
 \ar @(d,d)   "A";"B"^f
 \ar @(ld,ru) "B";"A"_g
\end{xy}
\end{z2col}

\medskip
{\bf 7}.
矢印は,指定した長さ(左方向が正)だけの平行の\textbf{ずらし}ができます:

\centerline{\color{red}\tt@<\MGNT{長さ}>}

\vspace*{-2ex}
\begin{z2col}
\begin{Alltt}
\\begin\{xy\}
 (0,0)="A", (20,0)="B",
 \\ar \red{@<1mm>} "A";"B"^\{f\}
 \\ar \red{@<1mm>} "B";"A"^\{f^\{-1\}\}
\\end\{xy\}
\end{Alltt}
\kaidan
\vspace{1ex}
\hfill
\begin{xy}
 (0,0)="A", (20,0)="B",
 \ar @<1mm> "A";"B"^{f}
 \ar @<1mm> "B";"A"^{f^{-1}}
\end{xy}
\end{z2col}

\newpage
\medskip
{\bf 8}.
ラベルの直前にラベルの\textbf{場所指定}を置くことができます.\hfill
\BACK\ \NEXT

\hspace{1cm}{\tt\red{< > -\ \ \ } :矢印の末尾/先頭/中央(\red{\tt-} は \red{<>(0.5)} と同等)}\\
\hspace{1cm}{\tt\red{(\MGNT{比})\ \ \ \ } :比で指定}\\
\hspace{1cm}{\tt\red{/\MGNT{長さ}/\ \ } :ずらし}\\
\hspace{1cm}{\tt\red{!\{\MGNT{位置}\}\ }} :位置で指定

\begin{z2col}
\begin{Alltt}
\\begin\{xy\}  (0,15)="A",(0,0)="B",
 \\ar        "A";"B"^\red{<}a
 \\ar@<6mm>  "A";"B"^\red{<<<}b
 \\ar@<12mm> "A";"B"^\red{>}c
 \\ar@<18mm> "A";"B"^\red{(.3)}d
 \\ar@<24mm> "A";"B"^\red{/3mm/}e  \\end\{xy\}
\end{Alltt}
\kaidan
\vspace{1ex}
\hfill
\begin{xy}  (0,15)="A", (0,0)="B",
 \ar         "A";"B"^<a
 \ar @<6mm>  "A";"B"^<<<b
 \ar @<12mm> "A";"B"^>c
 \ar @<18mm> "A";"B"^(.3)d
 \ar @<24mm> "A";"B"^/3mm/e
\end{xy}\!\!\!\!
\end{z2col}


\medskip
{\bf 9}.
``A" と ``B" を結ぶ直線との交点の位置は

\centerline{\color{red}\tt!\{"A";"B"\}}

によって得られます.

\begin{z2col}
\begin{Alltt}
\\begin\{xy\}
 (0,0)="A",(10,10)="B",(15,0)="C",(0,10)="D",
 \\ar "A";"B"
 \\ar "C";"D"|\red{!\{"A";"B"\}\\hole}
\\end\{xy\}
\end{Alltt}
\kaidan

\vspace*{1ex}
\hfill
\begin{xy}
 (0,0)="A",(10,10)="B",(15,0)="C",(0,10)="D",
 \ar "A";"B"
 \ar "C";"D"|!{"A";"B"}\hole
\end{xy}
\end{z2col}

\medskip
``C" から ``D" への矢印に対し,``A" と ``B" を結ぶ直線との交点に空白の穴 {\color{red}\verb|\hole|} を空けたものです.

\quad
一般に ``A" と ``B" を結ぶ直線と ``C" と ``D" を結ぶ直線の交点は\\
\centerline{\tt\red{!\{"A";"B":"C";"D",x\}}}
によって求められます.

\vspace*{-3ex}
\section{要素}
{\bf 1}.
図式に{\bf 要素}を配置し,それを矢印で結ぶことができます.要素は,
通常の記号や文字,さらに一般に\TeX で表現されたものとすることができます.
%要素の大きさが考慮されて矢印が引かれます.

\begin{z2col}
\begin{Alltt}
\\begin\{xy\}
 (0,0) \red{*\{A\}}="A",(10,10)\red{*\{B\}}="B",
 (15,0)\red{*\{C\}}="C",(0,10) \red{*\{D\}}="D",
 \\ar "A";"B"
 \\ar "C";"D"|!\{"A";"B"\}\\hole
\\end\{xy\}
\end{Alltt}
\kaidan
\vspace*{2ex}
\hfill
\begin{xy}
 (0,0) *{A}="A",(10,10)*{B}="B",
 (15,0)*{C}="C",(0,10)*{D}="D",
 \ar "A";"B"
 \ar "C";"D"|!{"A";"B"}\hole
\end{xy}
\end{z2col}

\smallskip
\quad
要素を配置するには,先頭に \red{\tt *} をつけて \red{\tt *\{\!要素\!\}} の
ように書きます.
%が,要素が通常の文字一つの時は \red{\tt\{\ \}} は省略できます.\\

\vspace{5ex}
\quad
次のようなものが要素として,先頭に \red{\tt *} を付加することで配置されます.

\hspace{.5cm}\red{\tt \{\!テキスト\!\}} :テキストは数式モードで組まれます.\\
\hspace{1cm}{\TeX}における通常の {\tt\BK hbox} のように組まれた\textbf{箱}
でもOKです.\\
\hspace{.5cm}\red{\tt\BK txt<幅>\{\!テキスト\!\}} :テキストは通常の文章として組まれます.\\
\hspace{1cm}\red{\tt<幅>} は省略可能で,テキスト中は強制改行
記号 \red{\tt\BK\BK} が使用できます.\\
\hspace{.5cm}\red{\tt \BK xybox\{{\rm\Xy-pic}による図式\}} 

これらの要素は,{\bf ラベル}としても使用できます.\hfill
\BACK\ \NEXT

\begin{z2col}
\begin{Alltt}
\\begin\{xy\}
 \\ar (0,0) *\red{\{}\\sum_\{n=1\}^N2^{-n}\red{\}}; 
 (30,0) 
 *\red{\\txt\{}増加して\red{\\\\}1に収束\red{\}}
 ^-\red{\{}N\\to\\infty\red{\}}
\\end\{xy\}
\end{Alltt}
\kaidan

\vspace*{4ex}
\hfill
\begin{xy}
 \ar (0,0) *{\sum_{n=1}^N2^{-n}}; 
    (30,0) *\txt{増加して\\1に収束}
     ^-{N\to\infty}   \end{xy}\!\!\!\!\!\!
\end{z2col}

\medskip
{\bf 2}.
要素は{\bf 大きさ}や{\bf 位置}が調整でき,{\bf 枠}を付けることなどもできます.

\hspace{.5cm}\MGNT{add-op}\MGNT{長さ} :大きさの調整.
\MGNT{add-op} には,\red{\tt + - = += -=} がある.\\
\hspace{1cm}\MGNT{長さ}を指定しない場合は,デフォルト値が用いられる.\\
\hspace{.5cm}\red{\tt !}\MGNT{ベクトル} :位置の調整(境界までのベクトル
\red{\tt R RD D} $\cdots$ なども可)\\
\hspace{.5cm}\red{\tt[\MGNT{形}]} :様々な{\bf 形}(\red{\tt[o]}は丸に)を設定可能で,以下の{\bf 枠}もそのひとつ.\\
\hspace{1cm}\red{\tt [F\MGNT{frame}:\MGNT{opt}]\ \,} :\red{\tt:\MGNT{opt}}は省略可.
\red{\tt F}に(空白無しで続けた)\MGNT{frame} は\\
\hspace{1.5cm}\red{\tt .\ - =\ \ \ \ \ } :矩形の枠,\MGNT{opt}は\red{\tt<\MGNT{長さ}>}で角の丸みの半径\\
\hspace{1.5cm}\red{\tt -- o-\ \ \ \ \ } :破線の枠,丸みを帯びた枠\\
\hspace{1.5cm}\red{\tt , -,\ \ \ \ \ \ } :影つきの枠(\MGNT{opt}は\red{\tt<\MGNT{長さ}>}で影のサイズ)\\
\hspace{1.5cm}\red{\tt o .o -o oo} :円形の枠,\MGNT{opt}は\red{\tt<\MGNT{長さ}>}で半径を指定\\
\hspace{1.5cm}\red{\tt (  ) \string^) \string_(\ } :\,{\tt (} の形の括弧(左/右/上/下)\\
\hspace{1.5cm}\red{\tt\BK\{ \BK\} \string^\{ \string_\}}:\,{\tt \{} の形の括弧(左/右/上/下)

枠は \red{\tt*\BK frm \MGNT{opt}\{\MGNT{frame}\}} または,
\red{\tt**\BK frm \MGNT{opt}\{\MGNT{frame}\}} によっても描ける.前者は直前の
\text{要素}を囲み,後者はその前の\text{要素}も囲むものとなる.

\begin{z2col}
\begin{Alltt}
\\begin\{xy\}
\{\\ar(0,20) *\{\\times\}*\red{+!D}\{a\};(15,20) *\{\\bigcirc\}
 \\ar(0,15) *\red{=0}\{\\times\};(15,15) *\red{=0}\{\\bigcirc\}
 \\ar(0,10) *\red{+[F]}\{A\};   (15,10) *\red{+[Fo]}\{B\}
 \\ar *\red{++=}\{A\} \red{*\\frm\{o\}};
    (15,3) \red{*+++}=\{B\} \red{*\\frm\{oo\}} \red{**\\frm\{-\}}\},
 (8,-8) \red{*+[Fo]}\{AB\} ="A",
    \\ar @(lu,ld) "A";"A"
\\end\{xy\}
\end{Alltt}
\kaidan
\vspace*{.2ex}
\hfill
\begin{xy}
{\ar(0,20) *{\times}*+!D{a};(15,20) *{\bigcirc}
 \ar(0,15)  *=0{\times};(15,15)*=0{\bigcirc}
 \ar(0,10)  *+[F]{A};  (15,10) *+[Fo]{B}
 \ar *++={A} *\frm{o};(15,3) *+++={B} *\frm{oo} **\frm{-}},
 (8,-8) *+[Fo]{AB} ="A",
 \ar @(lu,ld) "A";"A"
\end{xy}\!\!\!\!
\end{z2col}

\medskip
{\bf 3}.
\textbf{行列形式の図式}.\hfill{\BACK\ \NEXT}\\
\textbf{要素}を,直接座標を指定せずに行列の\textbf{成分}として
配置します.\\
座標を考慮する必要がなく,可換図式のような図式を描くのに便利です.

\quad
列は \red{\tt\string&} で,行は \red{\tt\BK\BK} で区切り,以下の
ように書きます.

{\tt
\BK xymatrix\{\\[-.2ex]
\quad \textbf{成分} \string& \textbf{成分} \string& $\cdots$ \BK\BK\\[-.4ex]
\quad \textbf{成分} \string& \textbf{成分} \string& $\cdots$ \BK\BK\\[-.7ex]
\qquad$\cdots$\quad\}}

\quad
各\textbf{成分}には,配置された\textbf{要素}と,そこを始点と
する\textbf{矢印}の情報があり

・ \textbf{要素}がテキストの場合,\red{\tt *} は不要.%\\
%\phantom{・ }
先頭が \red{\tt\BK} でないなら \red{\tt\{ \}} も不要.\\
・ 矢印の行き先は,成分の番号で指定できる:\\
\quad \red{\tt[\MGNT{hop}]} :\MGNT{hop}は,\red{\tt r l u d} (右/左/上/下) を
いくつか並べたもの\\
\qquad たとえば \red{\tt[rrd]} は,
その場所から2つ右で一つ下の成分を意味する.\\
\quad \red{\tt "\MGNT{行},\MGNT{列}"} :例えば \red{\tt"1,1"} は左上
の$(1,1)$成分

\smallskip
\begin{z2col}
\begin{Alltt}
\red{\\xymatrix\{} \\ar@\{\}\red{[rd]}|\{\\circlearrowright\}
 X\\times_Z Y \\ar\red{[r]}_q \\ar\red{[d]}^p \red{&} Y \\ar\red{[d]}_q\red{\\\\}
 X \\ar\red{[r]}^q \red{&} Z \red{\}}
\end{Alltt}
\kaidan

\vspace*{-3ex}
\hfill
\xymatrix{ \ar@{}[rd]|{\circlearrowright}
 X\times_Z Y \ar[r]_q \ar[d]^p & Y \ar[d]_p\\
 X \ar[r]^q & Z }\!\!\!\!\!\!\!\!
\end{z2col}

\begin{z2col}
\begin{Alltt}
\red{\\xymatrix\{}
 U \\ar@/_/\red{[ddr]}_y \\ar@\{.>\}\red{[dr]}
     |\{x\\otimes y\} \\ar@/^/\red{[drr]}^x \red{\\\\}
  \red{&} *+[F-:<3pt>]\{X \\times_Z Y\}
   \\ar\red{[d]}^q \\ar\red{[r]}_p \red{&} X \\ar\red{[d]}_f \red{\\\\}
  \red{&} Y \\ar\red{[r]}^g \red{&} Z \red{\}}
\end{Alltt}
\kaidan
\vspace*{-1ex}
\hfill
\xymatrix{
 U \ar@/_/[ddr]_y \ar@{.>}[dr]|{x\otimes y} \ar@/^/[drr]^x \\
  & *+[F-:<4pt>]{X \times_Z Y} \ar[d]^q \ar[r]_p & X \ar[d]_f \\
  & Y \ar[r]^g & Z }\!\!\!\!
\end{z2col}

\vspace*{-3ex}
\section{いろいろな図式}
{\bf 1}.曲がった軸をもつ矢印\\
円弧と線分を組み合わせて,\textbf{迂回する矢印}を描くことができる.経路は\\
\quad\red{\tt '\MGNT{目標}} :目標までの線分\\
\quad\red{\tt `\MGNT{方向} \MGNT{目標}} :出発\MGNT{方向}から\MGNT{目標}に向かって$\frac14$回転する.\\
\qquad 最初の\MGNT{方向}指定がなければ,前回の方向からの回転となる(次項も).
\\
\quad\red{\tt `\MGNT{方向} \MGNT{回転の向き} \MGNT{方向} \MGNT{目標}} :最初の\MGNT{方向}へ出発して直線的に移動し,後

\qquad の\MGNT{方向}で進むと\MGNT{目標}に到達できる地点で必要な回転をする.\\
\qquad 方向には \red{\tt r rd d ld l lu u ru} がある.\\
\qquad\MGNT{回転の向き}は,\red{\tt\string^} (反時計回り) また
は \red{\tt\string_} (時計回り)
\\
\quad
最後は,同じ向きの矢印を引いて,最終\MGNT{目標}に到達させる.
\\
\quad
円弧の半径は 10pt であるが \red{\tt/\MGNT{長さ}} により以降変更可能.

\begin{z2col}
\begin{Alltt}
\\xymatrix\{
 \{\\circ\} \\ar 
   \red{`r[d] `[rr] `^lu/3pt[rr] [rr]}
        &\{\\circ\}&\{\\circ\}\\\\
 \{\\circ\}&\{\\circ\}&\{\\circ\}\\\\ \}
\end{Alltt}
\kaidan
\vspace*{1ex}
\hfill
\xymatrix{
 {\circ} \ar 
    `r[d]       `[rr]
    `^lu/3pt[rr] [rr]
  &{\circ} &{\circ}\\
 {\circ}&{\circ}&{\circ}\\ }
\end{z2col}

\medskip
\quad
より一般に,始点と終点とを与えたとき,途中の\textbf{制御点}を指定したB\'ezier曲線
を使って\textbf{自由曲線}が描けます.

\centerline{\tt\red{**\BK crv\{\MGNT{制御点} \string& \MGNT{制御点} \string& $\cdots$\}}}

・ 制御点がないときは,二点を結ぶ線分になります.

・ 矢印のときと同様 \red{\tt\string~*\{.\}} などを入れて\textbf{線種}指定できます.\\
\begin{z2col}
\begin{Alltt}
\\begin\{xy\}
 (0,10)*[o]+\{A\};(20,0)*[o]+\{B\}="B"
 \red{**\\crv\{\}}
 \red{**\\crv\{~*\{.\}}(10,15)\red{\}}
 \red{**\\crv\{}(10,20)\red{&}(20,20)\red{\}} ?>*\\dir\{>\}, "B"
 \red{**\\crv\{}(5,10)\red{&}(10,10)\red{&}(15,-10)\red{&}(20,-5)\red{\}}
\\end\{xy\}
\end{Alltt}
\kaidan
\vspace*{-1ex}
\hfill
\begin{xy}
 (0,10)*[o]+{A};(20,0)*[o]+{B}="B"
 **\crv{}
 **\crv{~*{.}(10,15)}
 **\crv{(10,20)&(20,20)} ?>*\dir{>}, "B"
 **\crv{(5,10)&(10,10)&(15,-10)&(20,-5)}
\end{xy}
\end{z2col}

\medskip
{\bf 2}.座標軸の回転\hfill{\BACK\ \NEXT}\\
\MGNT{ベクトル\tt:} $X$軸の単位ベクトルを定める(回転+スケール変換を行う).

\begin{z2col}
\begin{Alltt}
\\begin\{xy\} \{\\ar (8,0)\}, 
 (-.5,\red{\\halfrootthree})\red{:} \{\\ar(0,0);(8,0)\},
 (0,0),(-.5,\red{\\halfrootthree})\red{:} \\ar(0,0);(8,0)
\\end\{xy\}
\end{Alltt}
\kaidan
\vspace*{-.5ex}
\hfill
\begin{xy}
{\ar (8,0)}, (-.5,\halfrootthree): {\ar(0,0);(8,0)},
(0,0),(-.5,\halfrootthree): \ar(0,0);(8,0)
\end{xy}
\end{z2col}

\medskip
%座標軸の単位ベクトルを指定することにより,自由なアフィン変換ができますが,
次に,{\tt \BK xymatrix} を使った$\frac\pi4$の整数倍の回転の例を挙げます.\\
\quad {\tt \BK xymatrix} の直後の指定で\\
\quad \red{\tt\ @}\MGNT{方向}  :\MGNT{方向}に向けた回転をする
(\red{\tt @r}\,:回転なし)\\
\quad \red{\tt\ @}\MGNT{\it add-op 長さ}:行と列の空きの設定
(\red{\tt@!}\,:行と列の空きを等しくする)\\
\quad \red{\tt@C}\MGNT{\it add-op 長さ}:列の空き
(\red{\tt@C!}\,:列の空きを等しくする)\\
\quad \red{\tt@R}\MGNT{\it add-op 長さ}:行の空き
(\red{\tt@R!}\,:行の空きを等しくする)\\
\quad \red{\tt@M}\MGNT{\it add-op 長さ}:要素のデフォルトの空き

\begin{z2col}
\begin{Alltt}
\\xymatrix\red{@=0pt@M=0pt}\{X&Y\\\\Z&W\}\\qquad
\\xymatrix\red{@dr@=0pt@M=0pt}\{X&Y\\\\Z&W\}
\end{Alltt}
\vspace*{-4ex}
\begin{Alltt}
\\xymatrix\red{@dr@C=4mm}\{
 A \\ar[r]^a &  B \\ar@\{.>\}[d]^b\\\\
 C \\ar[u]^c &  D \\ar@\{.>\}[l]^d \}
\end{Alltt}
\kaidan
\vspace*{-.5ex}
\hfill
\xymatrix@=0pt@M=0pt{X&Y\\Z&W}\qquad
\xymatrix@dr@=0pt@M=0pt{X&Y\\Z&W}

\vspace*{-1ex}
\hfill
\xymatrix@dr@C=4mm{
 A \ar[r]^a &  B \ar@{.>}[d]^b\\
 C \ar[u]^c &  D \ar@{.>}[l]^d
}\quad\phantom{.}
\end{z2col}

\medskip
{\bf 3}.仕様と例\hfill{\BACK}\\[-4ex]
\[\begin{xy}
  \ar@{=>} *++!D{\alpha_1} *\cir<4pt>{}; 
   (10,0)  *++!D{\alpha_2} *\cir<4pt>{}
%
  \ar@{-} (20,0) *++!D{\alpha_1} *\cir<4pt>{};
  (30,0) *++!D!R(0.4){\alpha_2} *\cir<4pt>{}="A",
  \ar@{-} "A";(35,8.61) *++!L{\alpha_3} *\cir<4pt>{}
  \ar@{-} "A";(35,-8.61) *++!L{\alpha_4} *\cir<4pt>{},
  \ar@{-} "A";(35,-8.61) *++!L{\alpha_4} *\cir<4pt>{},
%
  \ar@{-} (50,0) *++!D{\alpha_1} *\cir<4pt>{};
  (60,0) *++!D{\alpha_2} *\cir<4pt>{}="B"
  \ar@{-} "B";(70,0) *++!D{\alpha_3} *\cir<4pt>{}="C"
  \ar@{-} "C";(75,0) \ar@{.} (75,0);(80,0)^*!U{\cdots}
  \ar@{-} (80,0);(85,0) *++!D{\alpha_n} *\cir<4pt>{}
\end{xy}
\]
\if\pdf
\begin{verbatim}
\begin{xy}
  \ar@{=>} *++!D{\alpha_1} *\cir<4pt>{}; 
    (10,0)    *++!D{\alpha_2} *\cir<4pt>{}
  \ar@{-} (20,0) *++!D{\alpha_1} *\cir<4pt>{};
    (30,0)   *++!D!R(0.4){\alpha_2} *\cir<4pt>{}="A",
  \ar@{-} "A";(35,8.61) *++!L{\alpha_3} *\cir<4pt>{}
  \ar@{-} "A";(35,-8.61) *++!L{\alpha_4} *\cir<4pt>{},
  \ar@{-} "A";(35,-8.61) *++!L{\alpha_4} *\cir<4pt>{},
  \ar@{-} (50,0) *++!D{\alpha_1} *\cir<4pt>{};
    (60,0)   *++!D{\alpha_2} *\cir<4pt>{}="B"
  \ar@{-} "B";(70,0) *++!D{\alpha_3} *\cir<4pt>{}="C"
  \ar@{-} "C";(75,0) \ar@{.} (75,0);(80,0)^*!U{\cdots}
  \ar@{-} (80,0);(85,0) *++!D{\alpha_n} *\cir<4pt>{}
\end{xy}
\end{verbatim}
\else
このDynkin図式の例(
\keyin{\begin{xy}
  \ar@{=>} *++!D{\alpha_1} *\cir<4pt>{}; 
    (10,0)    *++!D{\alpha_2} *\cir<4pt>{}
%
  \ar@{-} (20,0) *++!D{\alpha_1} *\cir<4pt>{};
    (30,0)   *++!D!R(0.4){\alpha_2} *\cir<4pt>{}="A",
  \ar@{-} "A";(35,8.61) *++!L{\alpha_3} *\cir<4pt>{}
  \ar@{-} "A";(35,-8.61) *++!L{\alpha_4} *\cir<4pt>{},
  \ar@{-} "A";(35,-8.61) *++!L{\alpha_4} *\cir<4pt>{},
%
  \ar@{-} (50,0) *++!D{\alpha_1} *\cir<4pt>{};
    (60,0)   *++!D{\alpha_2} *\cir<4pt>{}="B"
  \ar@{-} "B";(70,0) *++!D{\alpha_3} *\cir<4pt>{}="C"
  \ar@{-} "C";(75,0) \ar@{.} (75,0);(80,0)^*!U{\cdots}
  \ar@{-} (80,0);(85,0) *++!D{\alpha_n} *\cir<4pt>{}
\end{xy}
}{ここ}をクリックすると,ソースがクリップボードにコピーされます)
\fi
では,以下の\textbf{円弧}を描くコマンドを使っています.
\\
%\begin{verbatim}
%\[\begin{xy}
%  \ar@{=>} *++!D{\alpha_1} *\cir<4pt>{}; 
%   (10,0)  *++!D{\alpha_2} *\cir<4pt>{}
%%
%  \ar@{-} (20,0) *++!D{\alpha_1} *\cir<4pt>{};
%  (30,0) *++!D!R(0.4){\alpha_2} *\cir<4pt>{}="A",
%  \ar@{-} "A";(35,8.61) *++!L{\alpha_3} *\cir<4pt>{}
%  \ar@{-} "A";(35,-8.61) *++!L{\alpha_4} *\cir<4pt>{},
%  \ar@{-} "A";(35,-8.61) *++!L{\alpha_4} *\cir<4pt>{},
%%
%  \ar@{-} (50,0) *++!D{\alpha_1} *\cir<4pt>{};
%  (60,0) *++!D{\alpha_2} *\cir<4pt>{}="B"
%  \ar@{-} "B";(70,0) *++!D{\alpha_3} *\cir<4pt>{}="C"
%  \ar@{-} "C";(75,0) \ar@{.} (75,0);(80,0)^*!U{\cdots}
%  \ar@{-} (80,0);(85,0) *++!D{\alpha_n} *\cir<4pt>{}
%\end{xy}
%\]
%\end{verbatim}
\quad
\red{\tt \BK cir<\MGNT{半径}>\{\!\MGNT{方向} \MGNT{回転の向き} \MGNT{方向}\!\}}

最初の\MGNT{方向}で円周上を\MGNT{回転の向き}(\red{\tt\string^} は反時計回り,
\red{\tt\string_} は時計回り)にスタートし,後の\MGNT{方向}まで円弧を
描きます.\red{\tt\{\ \}}の中を空として,これらを指定しない場合は円を描きます.

\begin{z2col}
\begin{Alltt}
\\begin\{xy\}
 *\red{\\cir<5mm>\{l_d\}}, (10,0) *\red{\\cir<5mm>\{r^d\}},
 (22,0) *+\{M\} *\red{\\cir\{dr_ur\}}
\\end\{xy\}
\end{Alltt}
\kaidan
\vspace*{.1ex}
\hfill
\begin{xy}
 *\cir<5mm>{l_d}, (10,0) *\cir<5mm>{r^d},
 (22,0) *+{M} *\cir{dr_ur}
\end{xy}\!\!\!\!\!\!
\end{z2col}

\medskip
\quad
\Xy-pic は {\TeX} のメモリーを大量に使うため,複雑な図式を扱うと
{\tt TeX capacity exceeded} のエラーが発生することがあります.
それが解消できない場合,より軽い

\keyin{\usepackage[all,ps,dvisp]{xy}
}{\quad{\tt\BK usepackage[all,ps,dvisp]\{xy\}}}

としてDVIファイルを作成して下さい.その後 PS ファイルから EPS ファイル,
あるいは PDF ファイルに変換({\tt Ghostscript}+{\tt dvipsk}+{\tt ps2pdf}な
ど)することができます.

\medskip
\quad
これまで \Xy-pic の基本的な機能をいくつか紹介してきましたが,
その他にも,楕円や多角形,各種ダイアグラムや結び目などに関する
多くの描画機能を持っています.
より詳しい仕様やこれらの機能については,
\if\pdf
dvioutに添付の
\href{http://akagi.ms.u-tokyo.ac.jp/input9.pdf}{Help},
\else
\href{file:input9.dvi}{添付のHelp},
\fi
あるいはそこに挙げてある文献などを参照してください.
\end{document}
