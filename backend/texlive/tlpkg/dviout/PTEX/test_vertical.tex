% pTeX サンプル for Testing dviout&dviprt (ver2.38以降) (2. Sep. 1991版)
%
% このサンプルの著作権は、Naochan!(戸塚 直哉:NIFTY-Serve GDH02225)にあります。

\font\tentmin=tmin10
\font\ninetmin=tmin9
\font\tentgoth=tgoth10
\font\ninerm=cmr9
\font\sevenrm=cmr7
\font\sixmin=min6
\font\ninemin=min9
\font\minii=tmin10 scaled \magstep2

\ninetmin \ninerm

\newskip\連数skip
\連数skip=0.1zw plus 0.1zw minus 0.1zw

\def\pTeX{p\kern-.2em\TeX}

\def\連数字#1{\hskip\連数skip\hbox to 0pt{\yoko\hss#1\hss}\nobreak\hskip\連数skip}
\def\連三字#1{{\sevenrm\hskip\連数skip\hbox to 0pt{\yoko\hss#1\hss}\nobreak\hskip\連数skip}}

\def\nidan{
  \loop\ifvbox0
    \vfill
    \vbox to \vsize{\baselineskip0mm \lineskiplimit0mm
      \vss
      \hbox to \hsize{\hss%
    \vbox{\tate
      \hrule
      \hbox to 210mm{%
        \vrule\hskip 70H%
        \setbox1=\vsplit0 to 452H
        \vtop to 148mm{\vskip 70H\unvbox1\vskip 70H}% 20*22+12
        \hskip 36H%
        \setbox1=\vsplit0 to 452H
        \vtop to 148mm{\vskip 70H\unvbox1\vskip 70H}% 20*22+12
        \hss%
	\ifodd\count0%
        \vtop{\vskip 70H\hbox to 452H{\yoko$\oldstyle\number\count0$\ 
		\sixmin 京都下宿案内\hfil}}%
	\else%
        \vtop{\vskip 70H\hbox to 452H{\yoko\hfil\sixmin 京都下宿案内\ 
		$\oldstyle\number\count0$}}%
	\fi%
        \hskip 56H\vrule%
      }
      \hrule
    }%
      \hss}\vskip3mm
    \hbox to \hsize{\hfill\rm\tengt
      Naochan!\ の『改訂版・京都下宿案内』より\hfill}
    \vss}
    \vfill
    \eject
  \repeat
}

\splittopskip=0pt
\splitmaxdepth=0pt
\tbaselineshift=2pt
\parindent=0.9zw
\baselineskip=20H plus 1H
\lineskiplimit=0H
\lineskip=0H
\leftskip=2zw

\def\半行空け{\vskip 10H plus 0.5H}

\def\一行空け{\vskip\baselineskip}

\def\科白#1{\parindent=0zw\par 「#1」\parindent=0.9zw}

\def\item#1{\leavevmode\hbox to 0pt{\kern-\leftskip #1\hss}}

\setbox0=\vbox{\tate \hsize=324H% 12[Q]x 27[字ツメ]

\line{\vrule width0pt height.5zh depth.5zh\hfill}
\hbox to 1zh{\yoko\hss%
\line{\tate\minii \hskip 3zh 改訂版・京都下宿案内\hfill}%
\hss}
\line{\vrule width0pt height.5zh depth.5zh\hfill}

\line{\vrule width0pt height.5zh depth.5zh\hfill}

東大路通沿いにも桜の花が咲く今日この頃ですが、如何お過ごしでしょうか。
早速、{\bf 「改訂版・京都下宿案内」}をお送りします。

住所も電話番号も去年と同じです。

\半行空け
〒\連三字{606}\ 京都市左京区○△□◇▽町×-\連三字{307}\par
{\hfill Tel.(xxx)xxx-xxxx(留守電有) }
\半行空け

京都にお越しの際は、{\bf 何らかの差し入れ品持参は必須}です。
案内もしてさしあげますが、何らかの見返りを求めることがあります。
(pretty ladyだったら免除しようかな)

去年の下宿案内にも書いたのですが、
今月(四月)の\連数字{15}日は{\bf My Birthday}で、
今年は二十歳になるので、それに見合うだけオトナになろうと思います。

いちおう下宿案内ですので、うちの下宿へ来るときの手順(秘儀)を公開します。

とりあえず、JR京都駅までやって来た、と仮定します。

するとあなたは、
\科白{ああ、京都に来たんだなあ}
\科白{そういえば、京都にゃ Naochan! がいるんだよね}
\科白{あいたいな}

と思います。

そういう時には、まず電話連絡を。

すると、留守電が、
\科白{はい\ \ Naochan!です\ \ ただ今留守にしております\ \ %
ピィという発信音のあとにメッセージをお願いします\ \ \ \ ピィ}

そういう場合は、お手数ですが、あなたの名前を名乗ったあとに以下のメッセージを。
\科白{せっかく京都に来たから逢いたかったのに$\cdots$留守かぁ、しょうがない$\cdots$}

こうすると、留守電の主は、留守にしていたことをきっと後悔しはじめますから。

もし、留守電ではなくて、本人が電話にでた場合。
\科白{はい、Naochan! です}
\科白{あの、○○です$\cdots$いま 京都に来てるんだけど、
せっかく京都に来たから会いたいんですが$\cdots$}

と、とりあえず言います。
(○○のところにあなたのお名前を。)



あなたが男性である場合、まず、こういう返事が返ってきます。
\科白{あ、○○(あなたの名前)?
ごめん、今日これからデートだから$\cdots$}

そういう時、差し入れ品が役に立ちます。
\科白{せっかく鳩サブレ持ってきたのになぁ$\cdots$}

きっと、「それなら$\cdots$」といって快くお会いするでしょう。



あなたが女性である場合$\cdots$
\科白{あ$\cdots$○○さん(あなたの名前)?
ごめん、今日これからデートだから$\cdots$}

あなたは、すかさずこう聞くのです。
\科白{えー?誰と?}
\科白{君とだよ}

$\cdots\cdots$あとの事は、それからふたりで考えましょう。





\一行空け

下宿を見に来たい人は$\cdots$
京都駅烏丸中央口を出ると、バスのりばがありますので、
××のりばから市バス\連三字{206}系統(東山通 北大路バスターミナルゆき)に
乗って、★★で降りてください(所要時間約\連数字{30}分)。
四月の運賃改訂で\連三字{200}円になりました。
何時何分のバスに乗る、とわかっていれば、★★のバス停で待ってます。
★★まで来て、姿が見えなかったら、
最寄りの電話ボックスから連絡してください。\par



\一行空け

それぞれの生活をエンジョイしましょう。



\一行空け

{\it When you wish upon a star \par
Makes no difference who you are \par
Anything your heart desires \par
Will come to you} (星に願いを)\par


\一行空け
\一行空け

なお、この下宿案内は\pTeX で組版しました。

縦書き可能な\pTeX を開発した アスキー と、縦書き対応の dviout/dviprt を
公開してくださった Naochan! さんに心から感謝いたします。(自画自賛)

\vfill
}

\nidan

\end
